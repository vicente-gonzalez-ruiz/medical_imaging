\chapter{Computed Tomography (CT)}

CT images are produced by passing x-rays through the body at a large
number of angles, by rotating the x-ray tube around the body. A
detector array, opposite the x-ray source, collects the transmission
projection data. The numerous data points collected in this manner are
synthesized by a computer into tomographic images of the patient. The
term tomography refers to a picture (graph) of a slice (tomo). The
advantage of CT over radiography is its ability to display
three-dimensional (3D) slices of the anatomy of interest, eliminating
the superposition of anatomical structures and thereby presenting an
unobstructed view of detailed anatomy to the physician
\cite{bushberg2011essential}.

The CT volume data set is essentially isotropic, which has led to
the increased use of coronal and sagittal CT images, in addition to
traditional axial images in CT. There are a number of different
acquisition modes available on modern CT scanners, including
dual-energy imaging, organ perfusion imaging, and prospectively gated
cardiac CT. While CT is usually used for anatomic imaging, the use of
iodinated contrast injected intravenously allows the functional
assessment of various organs as well \cite{bushberg2011essential}.

Radon transform


\section{Resources}

\bibliography{tomography}
