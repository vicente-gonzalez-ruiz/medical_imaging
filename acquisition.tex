\part{Image acquisition}

With the exception of nuclear medicine\footnote{In nuclear medicine
  imaging, radioactive substances are injected or ingested, and it is
  the physiological interactions of the agent that give rise to the
  information in the images \cite{bushberg2011essential}.}, all
medical imaging techniques require that the energy used to penetrate
the body’s tissues also interacts with those tissues.\footnote{If
  energy were to pass through the body and not experience some type of
  interaction (e.g., absorption or scattering), then the detected
  energy would not contain any useful information regarding the
  internal anatomy, and thus it would not be possible to construct an
  image of the anatomy using that information
  \cite{bushberg2011essential}.} The power, energy and time required
to acquiring medical images require a compromise between patient
safety and image quality.\footnote{Better x-ray images can be made
  when the radiation dose to the patient is high, better magnetic
  resonance images can be made when the image acquisition time is
  long, and better ultrasound images result when the ultrasound power
  levels are large \cite{bushberg2011essential}.}

Most modern medical imaging systems are digital, converting analogue
signals from detectors into digital data. This conversion, known as
digitization, involves two main steps: \emph{sampling} (selecting
specific points in time for conversion) and \emph{quantization}
(converting each analogue sample into a digital number). A
band-limited signal can be sampled without loss of
information. Quantization always loss some kind of information,
generating the so called \emph{quantization noise}. However, if the
number of bits/samples is high enought, we are replacing noise by
quantization noise, and therefore, the impact of quantization is
limited (or at least, controlled).\footnote{For instance, X-ray CT
  requires 12 bits per pixel due to its high contrast resolution,
  whereas ultrasound typically uses 8 bits due to its limited contrast
  resolution \cite{bushberg2011essential}.}
