\chapter{Medical Image Acquisition}

With the exception of nuclear medicine\footnote{In nuclear medicine
  imaging, radioactive substances are injected or ingested, and it is
  the physiological interactions of the agent that give rise to the
  information in the images.}, all medical imaging requires that the
energy used to penetrate the body’s tissues also interacts with those
tissues. If energy were to pass through the body and not experience
some type of interac- tion (e.g., absorption or scattering), then the
detected energy would not contain any useful information regarding the
internal anatomy, and thus it would not be possible to construct an
image of the anatomy using that information
\cite{bushberg2011essential}.

In most cases, the image quality that is obtained from medical
imaging devices involves compromise—better x-ray images can be made when
the radiation dose to the patient is high, better magnetic resonance images can
be made when the image acquisition time is long, and better ultrasound images
result when the ultrasound power levels are large. Of course, patient safety and
comfort must be considered when acquiring medical images; thus, excessive
patient dose in the pursuit of a perfect image is not acceptable. Rather, the power
and energy used to make medical images require a balance between patient safety
and image quality \cite{bushberg2011essential}.

\section{PET}

\section{SPECT}


