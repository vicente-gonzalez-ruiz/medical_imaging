\chapter{The \gls{WWW}}

\section{What is the Web?}
\begin{itemize}
\item The Web \popup{was created}{By Tim Berners-Lee who envisioned a
    system for automated information sharing among scientists
    worldwide. He wrote the first web server, the first web browser,
    and the first web page, and released the technology to the public
    in 1993. This decision to make the technology royalty-free and
    open was crucial to its explosive growth.} in the late 80 at the
  \gls{CERN} to share documentation.
\item It is based on the client/server model, where Web clients request
  \popup{Web objects}{A Web object is any resource (usually, a file)
    that is at the server side.} to a Web server.
\item Each object has an \popup{\gls{URL}{A unique identifier that
      specifies the Web server and the file within the server.}} that
  is used by clients to retrieve the object (from the server).
\end{itemize}

\section{The \acrentrylong{HTTP}}
\begin{itemize}
\item The client/server interaction is controlled by the \gls{HTTP}.
\item It is an application-layer protocol that runs over the \gls{TCP}.
\end{itemize}

\section{Web browsers}
\begin{itemize}
\item Most Web clients also implemen an objects-browser that \popup{can}{When the browser is not able to process an object (content), it usually offers to save the object.} render
  some standard Web contents, such as:
  \begin{enumerate}
  \item \href{}{\gls{MP3}}, \gls{AAC}, and \href{}{Ogg Vorbis} audio.
  \item \gls{PNG}, \gls{GIF}, \gls{PDF}, \gls{SVG}, \gls{JPEG}, and \gls{WebP} images.
  \item \href{}{H.264} and \href{}{VP9} encapsulated in a \gls{MP4}
    containers, and \href{}{Ogg Theora} videos.
  \item \gls{HTML}, \gls{CSS}, and \popup{\gls{JavaScript}}{} codes.
  \end{enumerate}
\end{itemize}