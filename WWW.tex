\chapter{The \gls{WWW}}

\section{What is the Web?}
\begin{itemize}
\item The Web \popup{was created}{By Tim Berners-Lee who envisioned a
    system for automated information sharing among scientists
    worldwide. He wrote the first web server, the first web browser,
    and the first web page, and released the technology to the public
    in 1993. This decision to make the technology royalty-free and
    open was crucial to its explosive growth.} in the late 80 at the
  \gls{CERN} to share documentation.
\item It is based on the client/server model, where Web clients request
  \popup{Web objects}{A Web object is any resource (usually, a file)
    that is at the server side.} to a Web server.
\item Each object has an \popup{\gls{URL}{A unique identifier that
      specifies the Web server and the file within the server.}} that
  is used by clients to retrieve the object (from the server).
\item WWW technology is used for cost-effective image distribution to
  clinicians, often via a thin client paradigm
  \cite{bushberg2011essential}.
\end{itemize}

\section{The \glsentrylong{HTTP}}
\begin{itemize}
\item The client/server interaction is controlled by the \gls{HTTP}.
\item It is an application-layer protocol that runs over the \gls{TCP}.
\end{itemize}

\section{Web browsers}
\begin{itemize}
\item Most Web clients also implemen an objects-browser that \popup{can}{When the browser is not able to process an object (content), it usually offers to save the object.} render
  some standard Web contents, such as:
  \begin{enumerate}
  \item \gls{MP3}, \gls{AAC}, and \href{https://en.wikipedia.org/wiki/Vorbis}{Vorbis} audio.
  \item \gls{PNG}, \gls{GIF}, \gls{PDF}, \gls{SVG}, \gls{JPEG}, and
    \popup{\gls{WebP}}{That also is a container for other image
      codecs.} images.
  \item H.264/\gls{AVC} (\popup{\gls{MP4}}{Usually encapsulated in a MP4
      container.}), and \popup{VP9}{Usually encapsulated in WebM or
      Matroska containers.}, \popup{\gls{AV1}}{Inside Matroska.},
    \popup{\gls{VC-1}}{Encaptulated in MP4.} and
    \popup{\href{https://en.wikipedia.org/wiki/Theora}{Theora}}{Encapsulated
      in Ogg files. videos.}.
  \item \gls{HTML}, \gls{CSS}, and
    \href{https://en.wikipedia.org/wiki/JavaScript}{JavaScript}
    \popup{source codes}{In computer science, a source code is any
      file that contains editable data that a computer can process. In
      this case, all these type of source codes are used for
      controlling the behaviour of the rendering of the Web objects.}.
  \end{enumerate}
\end{itemize}

\section{Secure transmissions: \gls{HTTPS}}
\begin{itemize}
\item \gls{HTTPS} secures data in transit through \gls{TLS}:
  \begin{enumerate}
  \item \textbf{Authentication}: HTTPS verifies that your browser is
    communicating with the correct server and not a fraudulent
    one. This is achieved through an \gls{SSL}
    \popup{certificate}{Which is a digital certificate issued by a
      trusted Certificate Authority (CA).}. The browser checks this
    certificate to \popup{confirm the identity of the Web site}{If the
      certificate is not valid, the browser will display a security
      warning.}.
  \item \textbf{Encryption}: Data between the server and the client is
    scrambled. This is performed in two steps:
    \begin{enumerate}
    \item \textbf{Handshake}:
      \begin{enumerate}
      \item The server sends its \popup{public key}{An SSL/TLS
          certificate.} to the client.
      \item The client uses this public key to encrypt a shared
        session key and sends it back to the server, which is the only
        one who can decrypt this message, using its corresponding
        private key.
      \end{enumerate}
    \item \textbf{Encripted data transfer}: A single, shared secret
      key is used for both encrypting and decrypting data (symmetric
      encryption).
    \item \textbf{Data integrity}: Special codes are interleaved with
      the data to allow the recipient to verify that the message is
      complete and unchanged.
    \end{enumerate}
  \end{enumerate}
\end{itemize}

\section{Some medical-imaging related Web sites}
\begin{enumerate}
\item \href{https://openi.nlm.nih.gov/}{Open Access Biomedical Image Search Engine}.
\item \href{https://medpix.nlm.nih.gov/home}{MedPix}.
\item The Visible Human Project:
  \href{https://www.nlm.nih.gov/research/visible/visible_human.html}{(description)},
  \href{https://visiblehumanproject.com/}{(viewer)}.
\item
  \href{https://www.flickr.com/photos/euthman/albums/72057594114099781/}{Specimens
    at flickr}.
\item
  \href{https://lane.stanford.edu/biomed-resources/bassett/index.html}{Bassett
    Collection of Stereoscopic Images of Human Anatomy}.
\item \href{https://dermnetnz.org/images}{DermNet}.
\item \href{http://imagebank.hematology.org/}{American Society of
    Hematology Image Bank}.
\item \href{http://www.med.harvard.edu/AANLIB/home.html}{The Whole
    Brain Atlas}.
\item \href{https://medpics.ucsd.edu/index.cfm}{MedPics - UC San Diego
    School of Medicine}.
\item \href{https://radiopaedia.org/}{Radiopaedia}.
\end{enumerate}
