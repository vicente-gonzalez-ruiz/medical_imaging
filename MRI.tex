\chapter{Magnetic Resonance Imaging (MRI)}

\section{Acquisition}
\gls{MRI} is a technique (usually used in medicine) to obtain detailed
views of (usually living) samples (tissues, organs, and even a
complete organism). The sample is subjected to a magnetic field (in
the order of Teslas (T), called the B0 field) which is strong enough
to align the tiny magnetic fields generated by the hydrogen atoms (in
reality, the magnetic fields are a consequence of the magnetic
activity of the protons in the nuclei) that behave as small
compasses. The scanner (that basically is a tube) has also a tubular
coil (forming another concentric tube) to generate a gradient in the
magnetic field which encodes the 3D position of the hydrogen
atoms.\footnote{The purpose of these gradients is to spatially encode
  the signal, allowing the MRI system to calculate how much signal is
  coming from each three-dimensional location (voxel) in the patient.}
Finally, closest to the sample\footnote{The antenna is typically
  placed as close to the anatomy under examination as possible to
  maximize signal reception.}, there is a third tubular concentric
structure that acts as a RF antenna, which emmit a signal to which the
hydrogen atoms are sensible\footnote{The RF excitation pulse (also
  known as B1 field), which causes hydrogen nuclei to absorb energy
  and resonate if the pulse's frequency matches their Larmor
  frequency.}, and also can receive the signal that the atoms emit
when the RF signal disapears (relaxation), until they recover their
equilibrium state (the alignment).\footnote{When the RF excitation
  pulse is switched off, the hydrogen nuclei lose the absorbed energy
  through a process called relaxation. This relaxation involves the
  recovery of longitudinal magnetization (the NMV (Net Magnetic
  Vector) realigns with B0) and the decay of coherent transverse
  magnetization.}

The signal received is an echo, which is a waveform representing
spatial frequencies from a selected slice (tomogram\footnote{A
  tomogram is an image that represents a slice or section of an 3D
  object,}). The contrast in this signal is determined by the
time-varying relaxation properties (T1 recovery and T2 decay) of the
tissues within that slice. This analog signal is digitized (converted
into binary numbers) via analog-to-digital conversion (ADC). These
digitized data points are then stored in a matrix called k-space. The
k-space data structure stores data about the frequency and phase
changes of magnetic moments over distance (spatial frequencies), i.e.,
each complex number of this matrix represents a Fourier
coefficient. In fact, the matrix if the Fourier transform of a 2D
image (a slice of the 3D MRI volume). Notice that each coefficients
``speaks'' about (the frecuencies and phases of) the complete
slice. Slice-selection is achieved by applying a slice-select gradient
simultaneously with an RF excitation pulse that has a specific center
frequency. The gradient creates a frequency slope, and only nuclei
whose precessional frequency matches the transmitted RF frequency (at
a specific location along the gradient) will resonate, thus defining
the slice \cite{westbrook2018mri}.

To reconstruct the 3D MRI volume we need to compute the inverse
Fourier transform of each slice, which is usually performed with the
inverse Fast Fourier Transform (FFT). The output of each transform is
a 2D image. The output 3D volume is the stack formed by all the 2D
images.

The signal received by the antenna is a 2D signal (with the
time-variying state of the relaxation of the hydrogen atoms of a slice
of the sample) that, when it is digitalized, generate a matrix of
complex numbers (the so called \emph{k-space}) with a magnitude and a
phase of the average oscillation (resonance) of the protons of the
hydrogen atoms in each small 3D cuve of the field of view (FOV) of
interest). These matrix represent the Fourier cofficients of a 2D
image (a slice), of the 3D MRI image. Modifiying the signal that
controls the gradient, we can select different FOVs (slices).

\section{Characteristics of the volumes}
\begin{enumerate}
\item \textbf{Signal-to-Noise Ratio (SNR)}, defined as the ratio of
  the amplitude of signal received to the average amplitude of the
  background noise. In the case of MRI depends on the strength of the
  signal received by the antenna (precession of coherent
  magnetization), and the strength of noise signal (random frequencies
  existing in space and time, primarily from thermal motion of the
  molecules (in the patient) and background electrical noise of the
  electronics). The SNR increases with the strength of B0, the proton
  density, the inverse of the diameter of the tube, the signal
  scanning times, the number of signal averages, and the size of the
  voxels.
\item \textbf{Spatial resolution}, defined as the ability to
  distinguish between two points as separate and distinct, is
  typically in the range of a fraction of milimeters (for example, 0.5
  mm). This characteristic depends fundamentally on the magnetic field
  strength (B0). The SNR increases with B0 and the voxel size, so, to
  increase the resolution (decrease the voxel size) we must keep high
  enough the SNR by increasing B0.
\item \textbf{Contrast-to-Noise Ratio (CNR)}. In general, the contrast
  in an viewed image depends on the range of intensities displayed,
  that should be as large as possible. However, a high contrast does
  not neccesaryly implies more perceived quality, because the number
  of different intensities should be also high enough (usually, at
  least different 64 intensities should be avaiable). In the case of
  MRI data, and considering that the quality is a subjective matter,
  an increase in the contrast (and therefore, a higher perceived
  quality that can help to improve the diagnostic) can be obtanied if
  we use \emph{image weighting} (for example, T2-weighted usually
  enhances pathologies), contrast agents (e.g., gadolinium) can
  selectively shorten relaxation times, increasing the CNR between
  pathology and normal anatomy (the noise), among other MRI techniques
  \cite{westbrook2018mri}.
\item 
\end{enumerate}

[MRI Physics | Magnetic Resonance and Spin Echo Sequences - Johns Hopkins Radiology(]https://www.youtube.com/watch?v=jLnuPKhKXVM)

\href{https://www.youtube.com/watch?v=TQegSF4ZiIQ}{How MRI Works - Part 1 - NMR Basics}
\href{https://www.youtube.com/watch?v=M7yh0To6Wbs}{How MRI Works - Part 2 - The Spin Echo}
\href{https://www.youtube.com/watch?v=R_4GuyJTzMo}{How MRI Works - Part 3 - Fourier Transform and K-Space}
\href{https://www.youtube.com/watch?v=vapJRr6gAds}{How MRI Works - Part 4 - The Gradient Recalled Echo (GRE)}