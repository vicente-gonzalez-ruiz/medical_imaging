\chapter{Magnetic Resonance Imaging (MRI)}

\section{Acquisition}
\gls{MRI}
\cite{westbrook2018mri,Wu2022MRI_Physics,thePIRL2018NMR_basics,thePIRL2018SpinEcho,thePIRL2018Fourier,thePIRL2018GRE}
is a technique (usually used in medicine) to obtain detailed views of
(usually living) samples (tissues, organs, and even a complete
organism). The sample is subjected to a magnetic field (in the order
of Teslas (T), called the B0 field) which is strong enough to align
the tiny magnetic fields generated by the hydrogen atoms (in reality,
the magnetic fields are a consequence of the magnetic activity of the
protons in the nuclei) that behave as small compasses. The scanner
(that basically is a tube) has also a tubular coil (forming another
concentric tube) to generate a gradient in the magnetic field which
encodes the 3D position of the hydrogen atoms.\footnote{The purpose of
  these gradients is to spatially encode the signal, allowing the MRI
  system to calculate how much signal is coming from each
  three-dimensional location (voxel) in the patient.}  Finally,
closest to the sample\footnote{The RF coil(s) is(are) typically placed
  as close to the anatomy under examination as possible to maximize
  signal reception.}, there is a third tubular concentric coil that
acts as a RF antenna, which emmit a signal to which the hydrogen atoms
are sensible\footnote{The RF excitation pulse (also known as B1
  field), which causes hydrogen nuclei to absorb energy and resonate
  if the pulse's frequency matches their Larmor frequency.}, and also
can receive the signal that the atoms emit when the RF signal
disapears (relaxation), until they recover their equilibrium state
(the alignment).\footnote{When the RF excitation pulse is switched
  off, the hydrogen nuclei lose the absorbed energy through a process
  called relaxation. This relaxation involves the recovery of
  longitudinal magnetization (the NMV (Net Magnetic Vector) realigns
  with B0) and the decay of coherent transverse magnetization.}

The signal received is an echo, which is a waveform representing
spatial frequencies from a selected slice (tomogram\footnote{A
  tomogram is an image that represents a slice or section of an 3D
  object,}). The contrast in this signal is determined by the
time-varying relaxation properties (T1 recovery and T2 decay) of the
tissues within that slice. This analog signal is digitized (converted
into binary numbers) via analog-to-digital conversion (ADC). These
digitized data points are then stored in a matrix called k-space. The
k-space data structure stores data about the frequency and phase
changes of magnetic moments over distance (spatial frequencies), i.e.,
each complex number of this matrix represents a Fourier
coefficient. In fact, the matrix if the Fourier transform of a 2D
image (a slice of the 3D MRI volume). Notice that each coefficients
``speaks'' about (the frecuencies and phases of) the complete
slice. Slice-selection is achieved by applying a slice-select gradient
simultaneously with an RF excitation pulse that has a specific center
frequency. The gradient creates a frequency slope, and only nuclei
whose precessional frequency matches the transmitted RF frequency (at
a specific location along the gradient) will resonate, thus defining
the slice \cite{westbrook2018mri}.

To reconstruct the 3D MRI volume we need to compute the inverse
Fourier transform of each slice, which is usually performed with the
inverse Fast Fourier Transform (FFT). The output of each transform is
a 2D image. The output 3D volume is the stack formed by all the 2D
images.

The signal received by the antenna is a 2D signal (with the
time-variying state of the relaxation of the hydrogen atoms of a slice
of the sample) that, when it is digitalized, generate a matrix of
complex numbers (the so called \emph{k-space}) with a magnitude and a
phase of the average oscillation (resonance) of the protons of the
hydrogen atoms in each small 3D cuve of the field of view (FOV) of
interest). These matrix represent the Fourier cofficients of a 2D
image (a slice), of the 3D MRI image. Modifiying the signal that
controls the gradient, we can select different FOVs (slices).

\section{Characteristics of the volumes}
\begin{enumerate}
\item \textbf{Signal-to-Noise Ratio (SNR)}: Defined as the ratio of
  the amplitude of signal received to the average amplitude of the
  background noise. In the case of MRI depends on the strength of the
  signal received by the antenna (precession of coherent
  magnetization), and the strength of noise signal (random frequencies
  existing in space and time, primarily from thermal motion of the
  molecules (in the patient) and background electrical noise of the
  electronics). The SNR increases with the \emph{strength of
    B0}\footnote{As the magnetic field strength increases, the Net
    Magnetic Vector (NMV) increases, leading to more available
    magnetization and consequently higher SNR. Doubling the field
    strength approximately doubles the SNR \cite{westbrook2018mri}.},
  the \emph{proton density}\footnote{Areas with a high concentration
    of MR-active protons (e.g., the pelvis) yield higher signal and
    thus higher SNR, whereas areas with low proton density (e.g., the
    lungs) result in lower signal and SNR \cite{westbrook2018mri}.},
  the \emph{coil(s) efficiency and distance to the
    sample}\footnote{Basically, the SNR is proportional to the inverse
    of the diameter of the tube. The power of the excitation RF
    signals is also proportional to the SNR.}, the \emph{signal
    scanning times}\footnote{Longer TR (Time-Repetition) allows for
    greater longitudinal magnetization recovery, making more signal
    available for conversion to transverse magnetization, which
    typically improves SNR. Shorter TE (Time-Echo) allows less
    coherent transverse magnetization to decay before the echo is
    collected, resulting in higher SNR \cite{westbrook2018mri}.}, the
  \emph{Number of Signal Averages (NSA)}\footnote{Increasing the NSA
    directly increases SNR, as correctly encoded signal is reinforced
    while random noise averages out \cite{westbrook2018mri}.}, and the
  \emph{size of the voxels}\footnote{Larger voxels contain more spins,
    which contribute to a higher signal and consequently increased SNR
    \cite{westbrook2018mri}.}.  In MRI, it depends on intrinsic signal
  intensity (T1, T2, proton density), voxel volume, number of
  excitations (NEX), receiver bandwidth, coil quality, magnetic field
  strength, and reconstruction algorithms \cite{bushberg2011essential}.
\item \textbf{Spatial resolution}: Defined as the ability to
  distinguish between two points as separate and distinct, is
  typically in the range of a fraction of milimeters (for example, 0.5
  mm). This characteristic depends fundamentally on the \emph{minimal
    slice-thickness} provided by the internal coil, the
  \emph{resolution of the k-space}, and on the \emph{magnetic field
    strength (B0)} to provide a high enought SNR.\footnote{The SNR
    increases with B0 and the voxel size, so, to increase the
    resolution (decrease the voxel size) we must keep high enough the
    SNR by increasing B0. Otherwise, the noise can make it difficult
    to recognize the pathology}
\item \textbf{Contrast}: In general, the contrast in an viewed image
  depends on the range of intensities displayed, that should be as
  large as possible.\footnote{In Magnetic Resonance Imaging (MRI),
    image contrast refers to the differences in signal intensity
    between various anatomical features, between anatomy and
    pathology, or between different tissues. This differentiation is
    crucial for identifying anatomical structures and detecting
    abnormalities within the body \cite{westbrook2018mri}.} However, a
  high contrast does not neccesaryly implies more perceived quality,
  because the number of different intensities should be also high
  enough (usually, at least different 64 intensities should be
  avaiable). In the case of MRI data, and considering that the quality
  is a subjective matter, an increase in the contrast (and therefore,
  a higher perceived quality that can help to improve the diagnostic)
  can be obtanied if we use \emph{image weighting} (for example,
  T2-weighted volumes usually enhances pathologies), \emph{contrast agents}
  (e.g., gadolinium) can selectively shorten relaxation times,
  increasing the contrast between pathology and normal anatomy, among
  \emph{other MRI contrast-enhancing techniques} (magnetization
  transfer contrast, phase-contrast MR angiography, or the use of
  presaturation pulses) \cite{westbrook2018mri}.
\end{enumerate}

\section{Image quality}

Detailed explanation of K-space filling methods (centric, keyhole, spiral, blade/propeller) and their impact on image acquisition and contrast. The diverse pulse sequences (Spin Echo, Inversion Recovery, Gradient Echo variants) are described for generating different tissue contrasts (T1, T2, proton density). Fast Spin Echo (FSE) and Echo Planar Imaging (EPI) are discussed for rapid acquisition. Parallel imaging techniques (e.g., SENSE) use multiple coils to reduce scan time or improve resolution by leveraging coil sensitivity profiles to correct aliasing. MR Angiography (MRA) using Time-of-Flight (TOF) and Phase Contrast (PC) techniques for vascular visualization. Functional MRI (fMRI) using BOLD sequences to map brain activity. Diffusion-Weighted Imaging (DWI) and Apparent Diffusion Coefficient (ADC) maps for assessing water mobility in tissues. Magnetic Resonance Spectroscopy (MRS) for "electronic biopsy" through spectral analysis of metabolites.

\section{Resources}

\bibliography{MRI}
