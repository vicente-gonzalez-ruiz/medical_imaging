\chapter{Positron Emission Tomography (PET)}

Positrons are positively charged electrons, and are emitted by some radioactive
isotopes such as fluorine-18 and oxygen-15. These radioisotopes are incorporated
into metabolically relevant compounds, such as 18 F-fluorodeoxyglucose ( 18 FDG),
which localize in the body after administration. The decay of the isotope produces
a positron, which rapidly undergoes a very unique interaction: the positron (e)
combines with an electron (e) from the surrounding tissue, and the mass of both
the e and the e is converted by annihilation into pure energy, following Einstein’s
famous equation E  mc 2 . The energy that is emitted is called annihilation radiation.
Annihilation radiation production is similar to gamma ray emission, except that
two photons are produced, and they are emitted simultaneously in almost exactly
opposite directions, that is, 180 degrees from each other. A positron emission
tomography (PET) scanner utilizes rings of detectors that surround the patient, and
has special circuitry that is capable of identifying the photon pairs produced during
annihilation. When a photon pair is detected by two detectors on the scanner, it
is assumed that the annihilation event took place somewhere along a straight line
between those two detectors. This information is used to mathematically compute
the 3D distribution of the PET agent, resulting in a set of tomographic emission
images \cite{bushberg2011essential}. 