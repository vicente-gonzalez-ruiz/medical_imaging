\chapter{Single Photon Emission Computed Tomography (SPECT)}

Single photon emission computed tomography (SPECT) is the tomographic counter-
part of nuclear medicine planar imaging, just like CT is the tomographic counterpart
of radiography. In SPECT, a nuclear camera records x- or gamma-ray emissions from
the patient from a series of different angles around the patient. These projection data
are used to reconstruct a series of tomographic emission images. SPECT images pro-
vide diagnostic functional information similar to nuclear planar examinations; how-
ever, their tomographic nature allows physicians to better understand the precise
distribution of the radioactive agent, and to make a better assessment of the function
of specific organs or tissues within the body (Fig. 1-8). The same radioactive isotopes
are used in both planar nuclear imaging and SPECT \cite{bushberg2011essential}. 