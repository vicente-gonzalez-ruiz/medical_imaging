\chapter{Single Photon Emission Computed Tomography (SPECT)}

Single photon emission computed tomography (SPECT) is the tomographic counter-
part of nuclear medicine planar imaging, just like CT is the tomographic counterpart
of radiography. In SPECT, a nuclear camera records x- or gamma-ray emissions from
the patient from a series of different angles around the patient. These projection data
are used to reconstruct a series of tomographic emission images. SPECT images pro-
vide diagnostic functional information similar to nuclear planar examinations; how-
ever, their tomographic nature allows physicians to better understand the precise
distribution of the radioactive agent, and to make a better assessment of the function
of specific organs or tissues within the body (Fig. 1-8). The same radioactive isotopes
are used in both planar nuclear imaging and SPECT \cite{bushberg2011essential}.

\section{Image quality}

Nuclear Imaging (SPECT/PET): The necessity of collimators for
projection image formation and the compromise with
sensitivity. Digital image formats and acquisition modes (frame, list,
gated). Image processing techniques like image subtraction, Regions of
Interest (ROIs), and time-activity curves (TACs) are detailed for
functional analysis. Spatial filtering (smoothing) for noise
reduction. In SPECT, the shift from filtered backprojection to
iterative reconstruction methods is noted, which can account for
attenuation, scatter, and spatial resolution degradation, leading to
higher quality images or lower dose studies. Attenuation correction in
SPECT/CT using x-ray CT data is highlighted. In PET, 3D (volume) data
acquisition without axial septa greatly increases
efficiency. Coregistered PET/CT systems combine functional and
anatomical information, which is critical for diagnosis and
attenuation correction. These topics collectively demonstrate the
sophisticated interplay of physics, computer science, and clinical
application in modern medical imaging, making them particularly
interesting from an image processing and visualization standpoint.

Iterative reconstruction in CT and SPECT can significantly reduce
image noise at lower doses \cite{bushberg2011essential}.
