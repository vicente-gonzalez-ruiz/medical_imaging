\documentclass{report}

\usepackage[screen,nopanel,sectionbreak]{pdfscreen}
\margins{0.5 cm}{0.5 cm}{0.5 cm}{1.0 cm}
%\screensize{15 cm}{20 cm} % Raz'on 3:4
%\screensize{12 cm}{16 cm}
\screensize{9 cm}{16 cm}
\overlayempty
%\overlay{overlay}
%\definecolor{mybg}{rgb}{1,0.9,0.7}
\definecolor{mybg}{rgb}{1.0,1.0,1.0}
\backgroundcolor{mybg}

\let\st\relax % Avoid -> LaTeX Error: Command \st already defined.
\usepackage{pdfcomment}
\usepackage{xcolor}
\newcommand{\popup}[2]{\pdftooltip{\textcolor{blue}{#1}}{#2}}

\usepackage{lastpage}
\usepackage{fancyhdr}
\renewcommand{\headrulewidth}{0pt}
\pagestyle{fancy} % for fancydr package
%\fancyhf{}
\pagenumbering{arabic} % Reseteamos las p'aginas
% Cabeceras no todo en may'usculas
%\renewcommand{\sectionmark}[1]{\markright{\thesection\ #1}}
% Definici'on de las cabeceras
\lhead{} \rhead{} \chead{}
%\lfoot[\fancyplain{}{\sf\thepage}]{\color{blue}\fancyplain{}{\sf\rightmark}}
%\lfoot[]{}
%\rfoot[\fancyplain{}{\sf\leftmark}]{\color{red}\fancyplain{}{\sf\thepage}}
%\rfoot[]{}
\cfoot[]{}
\rfoot{\colorbox{mybg}{\footnotesize\color{red}\thepage/\pageref{LastPage}}}
\setlength{\footskip}{12.0pt}

\usepackage{graphicx}
\graphicspath{{imgs}{figs}{notebooks}{.}{gnuplot}{modules}}

\usepackage{caption}
\captionsetup[figure]{skip=1pt}
\captionsetup[table]{skip=1pt}
\setlength{\textfloatsep}{0pt}

\usepackage[acronym]{glossaries}
\makeglossaries
\newacronym{TFT}{TFT}{{Thin-Film Transistor}}
\newacronym{GI}{GI}{{GastroIntestinal}}
\newacronym{SNR}{SNR}{{Signal-to-Noise Ratio}}
\newacronym{PNI}{PNI}{{Planar Nuclear Imaging}}
\newacronym{CT}{CT}{{Computed Tomography}}
\newacronym{LAC}{LAC}{{Linear Attenuacion Coefficient}}
\newacronym{PET}{PET}{{Positron Emission Tomography}}
\newacronym{SPECT}{SPECT}{{Single Photon Emission Computed Tomography}}
\newacronym{MRI}{MRI}{{Magnetic Resonance Imaging}}
\newacronym{RF}{RF}{{Radio Frequency}}
\newacronym{EM}{EM}{{Electro Magnetic}}
%\newacronym{CSF}{CSF}{CerebroSpinal Fluid}
\newacronym{PD}{PD}{\href{https://en.wikipedia.org/wiki/Proton}{Proton Density}}
\newacronym{FOV}{FOV}{\href{https://en.wikipedia.org/wiki/Field_of_view\#Tomography}{field of view}}

\newacronym{KB}{KB}{\href{https://en.wikipedia.org/wiki/Kilobyte}{KiloByte}}
\newacronym{MB}{MB}{\href{https://en.wikipedia.org/wiki/Megabyte}{MegaByte}}
\newacronym{GB}{GB}{\href{https://en.wikipedia.org/wiki/Gigabyte}{GigaByte}}
\newacronym{TB}{TB}{TeraByte}
\newacronym{PB}{PB}{PetaByte}

\newacronym{RAM}{RAM}{\href{https://en.wikipedia.org/wiki/Random-access_memory}{Random Access Memory}}
\newacronym{ROM}{ROM}{\href{https://en.wikipedia.org/wiki/Read-only_memory}{Read Only Memory}}
\newacronym{WORM}{WORM}{\href{https://en.wikipedia.org/wiki/Read-only_memory}{Write Once, Read Many}}
\newacronym{CD}{CD}{\href{https://en.wikipedia.org/wiki/Compact_disc}{Compact Disk}}
\newacronym{DVD}{DVD}{{Digital Versatile Disk}}
\newacronym{CD-ROM}{CD-ROM}{{Compact Disc - Read-Only Memory}}
\newacronym{DVD-ROM}{DVD-ROM}{Digital Versatile Disk - Read-Only Memory (see \gls{DVD})}
\newacronym{CD-R}{CD-R}{{Compact Disc - Recordable}}
\newacronym{CD-RW}{CD-RW}{{Compact Disc - ReWritable}}
\newacronym{RAID}{RAID}{{Redundant Array of Independent Disks}}
\newacronym{LAN}{LAN}{\href{https://en.wikipedia.org/wiki/Local_area_network}{Local Area Network}}
\newacronym{NAS}{{NAS}}{Network-Attached Storage}
\newacronym{HDD}{HDD}{{Hard Disk Drive}}
\newacronym{SSD}{SSD}{{Solid State Drive}}
\newacronym{PACS}{PACS}{\href{https://en.wikipedia.org/wiki/Picture_archiving_and_communication_system}{Picture Archiving and Communication System}}

\newacronym{PNG}{PNG}{{Portable Network Graphics}}
\newacronym{W3C}{W3C}{\href{https://en.wikipedia.org/wiki/World_Wide_Web_Consortium}{World Wide Web Consortium}}
\newacronym{RGBA}{RGBA}{\href{https://en.wikipedia.org/wiki/RGBA_color_model}{Red, Green, Blue, Alpha}}
\newacronym{APNG}{APNG}{{Animated PNG}}

\newacronym{JPEG}{JPEG}{{Joint Photographic Expert Group}}
\newacronym{ITU}{ITU}{{International Telecommunication Union}}
\newacronym{RGB}{RGB}{\href{https://en.wikipedia.org/wiki/RGB_color_model}{Red, Green, Blue}}
\newacronym{YCbCr}{YCbCr}{\href{https://en.wikipedia.org/wiki/YCbCr}{Y-luminance, blue-Crominance, red-Crominance}}
\newacronym{DCT}{DCT}{\href{https://en.wikipedia.org/wiki/Discrete_cosine_transform}{Discrete Cosine Transform}}

\newacronym{ISO}{ISO}{\href{https://www.iso.org/home.html}{International Organization for Standardization}}
\newacronym{IEC}{IEC}{\href{https://iec.ch/homepage}{International Electrotechnical Commission}}
\newacronym{CIE}{CIE}{\href{https://www.cie.co.at/}{Commission Internationale de l´Éclairage}}
\newacronym{CIELAB}{CIELAB}{\href{https://en.wikipedia.org/wiki/CIELAB_color_space}{CIE L*a*b*: L*-luminance, a*-chrominance, b*-crominance}}
\newacronym{sRGB}{sRGB}{\href{https://en.wikipedia.org/wiki/SRGB}{standard Red, Green, Blue}xs}
\newacronym{DWT}{DWT}{\href{https://en.wikipedia.org/wiki/Discrete_wavelet_transform}{Discrete Wavelet Transform}}
\newacronym{ROI}{ROI}{\href{https://en.wikipedia.org/wiki/Region_of_interest}{Region Of Interest}}
\newacronym{EBCOT}{EBCOT}{\href{https://en.wikipedia.org/wiki/JPEG_2000\#Coding}{Embedded Block Coding with Optimal Truncation}}
\newacronym{JPIP}{JPIP}{{JPeg2000 Interactive Protocol}}

\newacronym{MPEG}{MPEG}{{Moving Picture Experts Group}}
\newacronym{DVB}{DVB}{\href{https://en.wikipedia.org/wiki/DVB}{Digital Video Broadcasting}}
\newacronym{VCD}{VCD}{\href{https://en.wikipedia.org/wiki/Video_CD}{Video CD}}
\newacronym{RDO}{RDO}{\href{https://en.wikipedia.org/wiki/Rate\%E2\%80\%93distortion_optimization}{Rate/Distortion Optimization}}

\newacronym{VCEG}{VCEG}{\href{https://en.wikipedia.org/wiki/Video_Coding_Experts_Group}{Video Coding Experts Group}}
\newacronym{LPC}{LPC}{\href{https://en.wikipedia.org/wiki/Linear_predictive_coding}{Linear Predictive Coding}}
\newacronym{CABAC}{CABAC}{\href{https://en.wikipedia.org/wiki/Context-adaptive\_binary\_arithmetic\_coding}{Context-Adaptive Binary Arithmetic Coding}}

\newacronym{BER}{BER}{\href{https://en.wikipedia.org/wiki/Bit_error_rate}{Bit Error Rate}} 
\newacronym{HTTP}{HTTP}{\href{https://en.wikipedia.org/wiki/HTTP}{Hypertext Transfer Protocol}}
\newacronym{HTTPS}{HTTPS}{\href{https://en.wikipedia.org/wiki/HTTPS}{Hypertext Transfer Protocol Secure}}
\newacronym{WWW}{WWW}{\href{https://en.wikipedia.org/wiki/World_Wide_Web}{World Wide Web}} 
\newacronym{TCP}{TCP}{\href{https://en.wikipedia.org/wiki/Transmission_Control_Protocol}{Transmission Control Protocol}}
\newacronym{UDP}{UDP}{\href{https://en.wikipedia.org/wiki/User_Datagram_Protocol}{User Datagram Protocol}}
\newacronym{IP}{IP}{\href{https://en.wikipedia.org/wiki/Internet_Protocol}{Internet Protocol}}
\newacronym{NAT}{NAT}{\href{https://en.wikipedia.org/wiki/Network_address_translation}{Network Address Translation}}
\newacronym{SSL}{SSL}{\href{https://en.wikipedia.org/wiki/Transport_Layer_Security\#SSL\_1\.0,\_2.0,\_and\_3.0}{Secure Sockets Layer}}
\newacronym{TLS}{TLS}{\href{https://en.wikipedia.org/wiki/Transport_Layer_Security}{Transport Layer Security}}

\newacronym{CERN}{CERN}{\href{https://home.cern/}{Conseil Européen pour la Recherche Nucléaire}}
\newacronym{URL}{URL}{Uniform Resource Locator}
\newacronym{HTML}{HTML}{\href{https://en.wikipedia.org/wiki/HTML}{Hypertext Markup Language}}
\newacronym{GIF}{GIF}{\href{https://en.wikipedia.org/wiki/GIF}{Graphics Interchange Format}}
\newacronym{PDF}{PDF}{\href{https://en.wikipedia.org/wiki/PDF}{Portable Document Format}}
\newacronym{SVG}{SVG}{\href{https://en.wikipedia.org/wiki/SVG}{Scalable Vector Graphics}}
\newacronym{WebP}{WebP}{\href{https://en.wikipedia.org/wiki/WebP}{Web Picture}}
\newacronym{CSS}{CSS}{\href{https://en.wikipedia.org/wiki/CSS}{Cascading Style Sheets}}
\newacronym{AAC}{AAC}{\href{https://en.wikipedia.org/wiki/Advanced_Audio_Coding}{Advanced Audio Coding}}
\newacronym{MP3}{MP3}{\href{https://en.wikipedia.org/wiki/MP3}{MPEG-1 Audio Layer-3}}
\newacronym{MP4}{MP4}{MPEG-4 (video container)}
\newacronym{AVC}{AVC}{{Advanced Video Coding}}
\newacronym{HEVC}{HEVC}{{High Efficiency Video Coding}}
\newacronym{CTU}{CTU}{\href{https://en.wikipedia.org/wiki/Coding_tree_unit}{Coding Tree Unit}}
\newacronym{VVC}{VVC}{{Versatile Video Coding}}
\newacronym{VC-1}{VC-1}{\href{https://en.wikipedia.org/wiki/VC-1}{Video Codec 1}}
\newacronym{AV1}{AV1}{\href{https://en.wikipedia.org/wiki/AV1}{AOMedia Video 1}}
\newacronym{VP9}{VP9}{\href{https://en.wikipedia.org/wiki/VP9}{VP9}}

\newacronym{DICOM}{DICOM}{{Digital Imaging and Communications in Medicine}}
\newacronym{NEMA}{NEMA}{\href{https://www.nema.org/}{National Electrical Manufacturers Association}}
\newacronym{ECG}{ECG}{\href{https://en.wikipedia.org/wiki/Electrocardiography}{ElectroCardioGram}}
\newacronym{PCG}{PCG}{\href{https://en.wikipedia.org/wiki/Phonocardiogram}{PhonoCardioGram}}
\newacronym{ACR}{ACR}{\href{https://www.acr.org/}{American College of Radiology}}
\newacronym{GSDF}{GSDF}{\href{https://dicom.nema.org/medical/dicom/current/output/html/part14.html\#PS3.14}{Grayscale Standard Display Function}}
\newacronym{PCM}{PCM}{\href{https://en.wikipedia.org/wiki/Pulse-code_modulation}{Pulse-code modulation}}
\newacronym{RLE}{RLE}{\href{https://en.wikipedia.org/wiki/Run-length_encoding}{Run-Length Encoding}}
\newacronym{ULP}{ULP}{\href{https://dicom.nema.org/medical/dicom/current/output/html/part08.html\#chapter_9}{Upper Layer Protocol}}
\newacronym{JND}{JND}{\href{https://dicom.nema.org/medical/dicom/current/output/html/part14.html\#PS3.14}{Just-Noticeable Difference}}

\newacronym{CSF}{CSF}{\href{https://en.wikipedia.org/wiki/Contrast_(vision)\#Contrast_sensitivity}{Contrast Sensitivity Function}}

\newacronym{AND}{AND}{\href{https://en.wikipedia.org/wiki/Anisotropic_diffusion}{Anisotropic (Non-linear) Diffusion}}
\newacronym{NLM}{NLM}{\href{https://en.wikipedia.org/wiki/Non-local_means}{Non-Local Means}}
\newacronym{AI}{AI}{\href{https://en.wikipedia.org/wiki/Artificial_intelligence}{Artificial Intelligence}}
\newacronym{CNN}{CNN}{\href{https://en.wikipedia.org/wiki/Convolutional_neural_network}{Convolutional Neural Network}}
\newacronym{ANN}{ANN}{\href{https://en.wikipedia.org/wiki/Neural_network_(machine_learning)}{Artificial Neural Network}}
\newacronym{MNIST}{MNIST}{\href{https://en.wikipedia.org/wiki/MNIST_database}{Modified National Institute of Standards and Technology}}
\newacronym{GAN}{GAN}{\href{https://en.wikipedia.org/wiki/Generative_adversarial_network}{Generative Adversarial Network}}
\newacronym{ESRGAN}{ESRGAN}{\href{https://arxiv.org/abs/1809.00219}{Enhanced Super-Resolution Generative Adversarial Network}}
\newacronym{MSE}{MSE}{\href{https://en.wikipedia.org/wiki/Mean_squared_error}{Mean Squared Error}}

\title{\vspace{-5ex}Medical Imaging\\\href{https://www.ual.es/estudios/grados/presentacion/plandeestudios/asignatura/3321/33214301}{Informática Médica} $\star$ \href{https://www.ual.es/estudios/grados/presentacion/3321}{Grado en Medicina}}
\author{\href{https://www.ual.es/persona/515256515553484875}{Vicente González Ruiz}\\Departamento de Informática $\star$ Universidad de Almería}

\begin{document}
\maketitle
\tableofcontents
\listoffigures
\listoftables

\part{Acquisition of medical images}
%
\chapter{Basic concepts in the acquisition of medical images}

\section{Energy-tissue interaction}
\begin{itemize}
\item With the exception of \popup{nuclear medicine}{In nuclear
    medicine imaging, radioactive substances are injected or ingested,
    and it is the physiological interactions of the agent that give
    rise to the information in the images.}, all medical imaging
  techniques require that the energy used to penetrate the body's
  tissues also \popup{interacts}{If energy were to pass through the
    body and not experience some type of interaction (e.g., absorption
    or scattering), then the detected energy would not contain any
    useful information regarding the internal anatomy, and thus it
    would not be possible to construct an image of the anatomy using
    that information.} with them \cite{bushberg2011essential} (see
  Figure~\ref{fig:X-rays}).
\end{itemize}
\vspace{-5ex}
\begin{figure}[!h]
  \centering
  \includegraphics[width=12cm]{X-rays}
  \caption{X-rays machine taking an X-rays image from a
    patient \cite{CC2025Xray}.\label{fig:X-rays}}
\end{figure}

\section{Image quality VS patient safety}
\begin{itemize}
\item The power, energy and time required to acquiring medical images
  require a \popup{compromise}{Better x-ray images can be made when
    the radiation dose to the patient is high, better magnetic
    resonance images can be made when the image acquisition time is
    long, and better ultrasound images result when the ultrasound
    power levels are large.} between patient safety and image quality
  \cite{bushberg2011essential} (see
  Figure~\ref{fig:quality_vs_radiation}).
\end{itemize}
\vspace{-3ex}
\begin{figure}[!h]
  \centering
  \includegraphics[width=8cm]{X-rays__quality_vs_radiation}
  \caption{Quality VS radiation dose (measured in milliampere-second,
    i.e., the electric charge provided to the X-rays
    machine))
    \cite{huda2015radiographic}.\label{fig:quality_vs_radiation}}
\end{figure}

\section{Digital media}
\begin{itemize}
\item Most modern medical imaging systems are digital.
\item Convert \popup{continuous}{A continuous signal it defined at any
    instant of time or space.} (\popup{analog}{An analog signal can
    take an infinite number of values (for example, the amount of
    water that we can put into a glass.)}) signals from detectors into
ç  digital data. This conversion, known as digitization, involves two
  main steps:
  \begin{enumerate}
  \item \textbf{sampling}: Selection specific points in time (or
    space) for conversion.
  \item \textbf{quantization}: Converion of each analogue sample into a
    digital number.
  \end{enumerate}
\end{itemize}

\section{Noise}
\begin{itemize}
\item The processes by which radiation is emitted and interacts with
  matter are inherently random, making all radiation measurements,
  including medical imaging, subject to random error
  \cite{bushberg2011essential}.
\end{itemize}

\section{Quantization noise}
\begin{itemize}
\item A \popup{band-limited signal}{A band-limited signal requires
    only a finite interval of frequencies.} can be sampled without
  loss of information. However, quantization always loss some kind of
  information, generating the so called \emph{quantization noise}.
\item To reduce the quantization noise, \popup{the
number of bits/sample must be high enough}{For instance, X-ray CT
  requires 12 bits per pixel due to its high contrast resolution,
  whereas ultrasound typically uses 8 bits due to its limited contrast
  resolution.} \cite{bushberg2011essential}.
\item Usually, quantization \popup{replaces \emph{real}
    noise}{Physically, it is impossible to obtain a perfect digital
    signal (without any kind of noise) because the acquisition systems
    are intrinsically noisy.} by quantization noise, and therefore,
  the impact of using a finite number of bits is, in some way, limited
  (or at least, controlled).
\end{itemize}
%\chapter{Ultrasound}

Mechanical energy in the form of high-frequency (usually in the range
of 2 to 10 MHz)\footnote{The higher the frequency the better
  resolution and image detail, but lower penetration.} sound waves can
be used to generate images of the anatomy of a patient. These waves
pass through tissues, get reflected, and the returning wave (echo) is
detected and forms the image. In B-mode (B for
Brightness\footnote{There is A-Mode (A for Amplitude) that is mainly
  used in mainly in ophthalmology to investigate retinal detachment.})
imaging, the most common imaging used in medicine, the intensity of
the returning wave is represented as a level of brightness on the
monitor to give a 2D cross-sectional image on the monitor
\cite{abdulla2025ultrasound}.

\section{Acquisition details}
The echoes returned are shown on screen in a grey-scale corresponding
to their intensity. The structures are shown as a 2D image on
screen. A short-duration pulse of sound is generated by an ultrasound
transducer that is in direct physical contact with the tissues being
imaged. The sound waves travel into the tissue, and are reflected by
internal structures in the body, creating echoes. The reflected sound
waves then reach the transducer, which records the returning sound
\cite{bushberg2011essential,abdulla2025ultrasound_machine}.

\begin{figure}
  \centering
  \includegraphics{doppler}
  \caption{An ultrasound image that shows the Doppler effect
    \cite{abdulla2025ultrasound_imaging_doppler}.\label{fig:doppler}}
\end{figure}

The speed of the sound signal in the tissues are low enough to use the
Doppler effect to detect their motion. Thus, for example, using the
M-Mode (M for Motion) we can measure the blood flow displayed as color
channels (see Figure~\ref{fig:}
).\footnote{Both the speed
  and direction of blood flow can be measured, and within a subarea of
  the grayscale image, a color flow display typically shows blood flow
  in one direction as red, and in the other direction as blue
  \cite{bushberg2011essential}.}

\section{Image content}
Ultrasound imaging is basically a 2D technique (slices). However, 3D
images (volumes) can be generated by placing the known voxels in a 3D
structure and interpolating the unknown voxels. Then, using
segmentation it is possible to display surfaces to see, for example,
the face of a fetus.

\section{Speckle noise}
Ultrasound images are noisy and the predominant type of noise is
speckle noise.

Speckle is usually modeled as multiplicative noise. Amplitude
(magnitude) follows a Rayleigh distribution. Phase is uniformly
distributed. Intensity (amplitude squared) follows an exponential
distribution.

If multiple images are taken (changing the angle and orientation) and
averaged, then intensity in the resulting \emph{spatial compounding}
\cite{bushberg2011essential} follows a Gamma distribution.

%\chapter{Radiography}
\vspace{-35ex}
\begin{flushright}
  \includegraphics[width=5cm]{Knie-roentgen-r-seite} % https://upload.wikimedia.org/wikipedia/commons/thumb/8/89/Knie-roentgen-r-seite.jpg/800px-Knie-roentgen-r-seite.jpg
\end{flushright}

\section{X-ray and radiography}
\begin{itemize}
\item Radiography (X-ray 2D \popup{projection}{Radiography is also a
  projection imaging modality, meaning that each point on the image
  corresponds to information along a straight line through the
  patient}) is a transmission imaging modality where X-rays are
  emitted from a \popup{source}{An X-rays generator.}, pass through
  the patient, and are detected on the other side using a flat
  (usually \popup{digital TFT}{In digital X-ray detectors, a TFT array
    is used to read out electrical charges generated by the impact of
    the X-rays.}) detector (see
  Fig.~\ref{fig:projectional_radiography}).
\end{itemize}
\vspace{-1ex}
\begin{figure}[!h]
  \centering
  \includegraphics[width=6.5cm]{Projectional_radiography_components}
  \caption{Acquisition of projectional radiography, with an X-ray generator and a detector
    \cite{Wikipedia_X-ray_machine}.\label{fig:projectional_radiography}}
\end{figure}

\section{Transmission and attenuation}
\begin{itemize}
\item The X-ray attenuation of different tissues (e.g., bone, soft
  tissue, air) modify the homogeneous distribution of X-rays that
  enters the patient X-ray, forming the image in the detector
  \cite{bushberg2011essential} (see
  Fig.~\ref{fig:Attenuation-of-X-rays}).
\end{itemize}
\vspace{-1ex}
\begin{figure}[!h]
  \centering
  \includegraphics[width=9cm]{Attenuation-of-X-rays}
  \caption{Intensity-distance graph of X-rays for air and body
    \cite{Attenuation_X-rays}.\label{fig:Attenuation-of-X-rays}}
\end{figure}

\section{\glsentrylong{LAC} (\glsentryshort{LAC})}
\begin{itemize}
\item The X-rays are
  exponentially attenuated when they travel across the tissues \cite{wikipedia_LAC}.
\item To simplify numerical analysis and processing, we don't work
  directly with the attenuation coefficients but with the \glspl{LAC}, which
  are obtained applying a logarithm operation.
\item For example, the numerical values used in
  Section~\ref{sec:FBP_example} are the result of computing the
  logarithm of the \popup{measured}{Returned by the detector.}
  attenuation coefficients.
\end{itemize}

\section{Why are we ``transparent''?}
\begin{itemize}
\item The \popup{fast undulatory movement of X-ray-photons}{From 30
    petahertz (PHz)}{3x10$^{16}$ Hz.} to \popup{30 exahertz
    (EHz)}{3x10$^{19}$ Hz} makes them capable of \popup{penetrating
    soft tissues}{The higher the frequency of the X-rays photons, the
    higher their energy and the higher their penetration capability.}.
\end{itemize}
\vspace{-1ex}
\begin{figure}[!h]
  \centering
  \includegraphics[width=8cm]{electromagnetic-spectrum}
  \caption{The electromagnetic spectrum
    \cite{X-rays_in_spectrum}.\label{fig:X-rays_in_spectrum}}
\end{figure}

\section{Ionization and biologic damage}
\begin{itemize}
\item X-rays are a form of \popup{ionizing radiation}{An ionizing
    radiation is capable of removing electrons from atoms or
    molecules, a process known as ionization (when an atom or molecule
    loses or gains electrons, it acquires a net electrical charge and
    becomes an ion).} (see Fig.~\ref{fig:ionization}), creating
  \popup{free radicals}{A free radical is an atom or molecule that has
    an unpaired electron in its outer shell. Because electrons prefer
    to exist in pairs, this unpaired electron makes the free radical
    highly unstable and very reactive.}.
\end{itemize}
\vspace{-1ex}
\begin{figure}[!h]
  \centering
  \includegraphics[width=10cm]{ionization-diagram}
  \caption{Ionization
    \cite{Perakende_ionization}.\label{fig:ionization}}
\end{figure}

\begin{itemize}
\item Free radicals are extremely reactive and can interact with
  biomolecules. This can produce a list of effects
  \cite{bushberg2011essential}:
  \begin{enumerate}
  \item \textbf{Short-term effects} (usually under high doses):
    \href{https://en.wikipedia.org/wiki/Radiation_burn}{burns},
    sickness.
  \item \textbf{Long-term effects}:Damage to DNA, could suffer
    \popup{mutations}{Although heavily irradiated cells often die
      during mitosis, preventing the propagation of seriously
      defective cells, damage to DNA at locations responsible for
      controlling cell division (e.g., oncogenes or tumour suppressor
      genes) could potentially lead to the formation of a tumour or
      cancer}, and \popup{tissue dysfunction}{If there is cellular
      dysfunction, tissues can lose function and finally generate
      organ failure.}.
  \end{enumerate}
\end{itemize}

\section{Artifacts}
\begin{itemize}
\item In radiology, artifacts come from hardware failure, operator
  error (including the interaction/control with/of the patient) and
  software (post-processing) artifacts.
\end{itemize}

%\section*{}
\newpage
\subsection*{Motion blur}
\vspace{-1ex}
\begin{figure}[H]
  \centering
  \includegraphics[width=4.5cm]{motion_blur_2}
  \caption{The image unsharpness was caused by patient movement
    \cite{radiology_key}.\label{fig:motion_blur}}
\end{figure}

%\section*{}
\newpage
\subsection*{Static electricity}
\vspace{-1ex}
\begin{figure}[H]
  \centering
  \includegraphics[width=8cm]{static-electricity}
  \caption{Dissipation of static charges when the image is taken \cite{radiopaedia}.\label{fig:static_electricity}}
\end{figure}

%\section*{}
\newpage
\subsection*{Dead pixel}
\vspace{-1ex}
\begin{figure}[H]
  \centering
  \begin{tabular}{cc}
    \includegraphics[width=6cm]{dead-pixel-artifact_frontal} &
                                                               \includegraphics[width=6cm]{dead-pixel-artifact_oblique}
  \end{tabular}                                                
  \caption{Dead pixels are always white and always located at the same coordinates. \cite{radiopaedia}.\label{fig:dead_pixel}}
\end{figure}

%\section*{}
\newpage
\subsection*{Under- and over-exposure}
\begin{itemize}
\item In digital radiology (down), the detector always generates a good contrast.
\item The main drawback is that, under-exposure usually have more noise.
\end{itemize}
\vspace{-1ex}
\begin{figure}[H]
  \centering
    \includegraphics[width=11cm]{X-ray_exposure}
    \caption{Exposure impact (up: analog, down: digital)
      \cite{VELDKAMP2009209}.\label{fig:exposure}}
\end{figure}

\section{Mammography}
\begin{itemize}
\item Mammography is the 2D X-rays projection\\ of the breast
  \cite{bushberg2011essential}.
\item Makes use of much lower x-ray energies\\ than general purpose
  radiography.
\end{itemize}
\vspace{-24ex}
\begin{figure}[H]
  \begin{flushright}
    \includegraphics[width=5.5cm]{normal_mammogram}
    \end{flushright}
    \caption{Mammogram \cite{CDC_mammograms}.\label{fig:mamogram}}
\end{figure}

\section{Fluoroscopy}
\begin{itemize}
\item A fluoroscope produces real-time X-ray images with high temporal
  resolution (e.g., 30 frames per second), allowing continuous motion
  viewing, useful for \popup{interventional procedures}{Fluoroscopy is
    used for positioning catheters in arteries, visualizing contrast
    agents in the \gls{GI} tract, and for other medical applications
    such as invasive therapeutic proce- dures where real-time image
    feedback is necessary. It is also used to make x-ray movies of
    anatomic motion, such as of the heart or the esophagus.}
  \cite{bushberg2011essential}.
\item Compared to ``one-shot'' radiography, the images are more noisy.
\item
  \href{https://en.wikipedia.org/wiki/Fluoroscopy#/media/File:Normal_barium_swallow_animation.gif}{Swallowing
    of barium} \cite{Wikipedia_Fluoroscopy}.
\end{itemize}

\section{Quantum noise}
\label{sec:radiography_quantum_noise}
\begin{itemize}
\item Quantum noise is the most common noise in X-rays imaging.
\vspace{1ex}
\begin{figure}[H]
  \centering
    \includegraphics[width=0.9\textwidth]{quantum_noise_X-rays}
    \caption{Quantum noise in radiology
      \cite{CHANDRA2020107426}.\label{fig:quantum_noise_X-rays}}
\end{figure}
\item It is directly related to photon counting (fewer photons
  $\rightarrow$ more noise).
\item More evident (lower \gls{SNR}) at low radiation doses.
\item The amplitude of quantum noise depends on the (clean) signal amplitude.  
\newpage
\item On average, quantum noise appears on all the frequencies and is typically \href{https://en.wikipedia.org/wiki/Colors_of_noise}{colored}.
\end{itemize}
\vspace{-1ex}
\begin{figure}[H]
  \centering
    \includegraphics[width=7.5cm]{quantum_noise_mammography}
    \caption{Average quantum noise spectrum (up), some examples in mammograms (below) 
      \cite{saunders2007does}.\label{fig:quantum_noise_X-rays_spectrum}}
\end{figure}

%\chapter[\glsentrylong{CT} (\glsentryshort{CT})]{Computed\\Tomography (CT)}
\vspace{-47ex}
\begin{flushright}
\includegraphics[width=5.0cm]{CT_of_a_normal_abdomen_and_pelvis} % https://upload.wikimedia.org/wikipedia/commons/2/2b/CT_of_a_normal_abdomen_and_pelvis%2C_thumbnail.png
\end{flushright}

\section{History}
\begin{itemize}
\item \popup{Computed Tomography (CT)}{tomografía axial computarizada
    (TAC)} is a medical imaging modality that became clinically
  available in the early 1970s and was the first to be made possible
  by computers \cite{wikipedia_CT}.
\end{itemize}
\vspace{-1ex}
\begin{figure}[H]
  \centering
  \begin{tabular}{cc}
    \includegraphics[width=6.5cm]{Emi1010} &
                                           \includegraphics[width=5.0cm]{First_CT_scan_brain}
  \end{tabular}
  \caption{First CT scanner, and an example (80x80 pixels)
    \cite{Wikipedia_CT_history}.\label{fig:first_CT}}
\end{figure}

\section{The CT scanner}
\begin{itemize}
\item The images are obtained by X-ray transmission, moving an X-ray
  tube (the \emph{gantry}, see Fig.~\ref{fig:new_CT})
  and detector arrays around the patient.
\end{itemize}
\vspace{-1ex}
\begin{figure}[H]
  \centering
  \begin{tabular}{cc}
    \includegraphics[width=6.5cm]{CT_Naeotom_Alpha_Pilsen_2022} &
                                                                \includegraphics[width=6.5cm]{CT_equipment_machine}
                                                                \end{tabular}
  \caption{A recent CT scanner and its basic schema \cite{Wikipedia_CT_history}.\label{fig:new_CT}}
\end{figure}

\section{Capabilities}
\begin{itemize}
\item CT can provide \popup{isotropic}{Isotropic spatial resolution is
    a characteristic of images where the resolution is equal in all
    directions. In the case of 3D images, this means that a voxel (a
    3D pixel) is a perfect cube, with equal dimensions on all sides.:
    along the X-axis (left-right), the Y-axis (up-down), and the
    Z-axis (front-back).} 3D images (volumes), including
  \popup{axial}{An axial view divides the body horizontally into a top
    (superior) and bottom (inferior) section.}, \popup{coronal}{A
    coronal view divides the body vertically into a front (anterior)
    and back (posterior) section.}, and \popup{sagittal}{A sagittal
    view divides the body vertically into a right and left section.}
  views (see Fig.~\ref{fig:views_in_CT}).
\begin{figure}[H]
\vspace{-3ex}
  \centering
  \begin{tabular}{cc}
    \includegraphics[height=3.5cm]{CT_axis} & \includegraphics[height=4.5cm]{axial_coronal_sagittal}
  \end{tabular}
  \caption{Reference axis used in CT and views generated
    \cite{morin2025radiation}.\label{fig:views_in_CT}}
\end{figure}
\item The use of
  iodinated contrast injected intravenously allows the functional
  assessment of various organs as well \cite{bushberg2011essential}.
\end{itemize}

\section{Acquisition}
\begin{itemize}
\item The 2D data collected from each angle is called a \emph{projection}
(see Fig.~\ref{fig:projection}), consisting of multiple individual
attenuation measurements.
\end{itemize}
\vspace{-1ex}
\begin{figure}[H]
  \centering
  \includegraphics[height=4.5cm]{projection}
  \caption{A projection example \cite{takase2025CT}. $\theta$ is the
    angle used for taking the projection and $t$ the location in
    the detector.\label{fig:projection}}
\end{figure}

\begin{itemize}
\item The acquisition process can be classified as
(see Fig.~\ref{fig:CT_geometries}):
\begin{enumerate}
\item \textbf{Parallel-beam}: The X-rays are parallel and the detector
  is organized as a 1D array.
\item \textbf{Fan-beam}: The X-rays form a fan with the vertex in the
  emitter in a line and the detector is organized as a 1D array.
\item \textbf{Cone-beam}: The X-rays form a cone with the
  vertex in the emitter, and the detector is a 2D array.
\end{enumerate}
\end{itemize}
\vspace{-1ex}
\begin{figure}[H]
  \centering
  \includegraphics[height=3.5cm]{CT_geometries}
  \caption{Acquisition geometries used in CT \cite{takase2025CT}.\label{fig:CT_geometries}}
\end{figure}

\begin{itemize}
\item Depending on how the scanning is performed, we can distinguish between
(see Fig.~\ref{fig:scannings}):
\begin{enumerate}
\item \textbf{Axial (sequential) scanning}:
  The gantry completes a 360-degree rotation to acquire projection
  data while the patient table is stationary. 
\item \textbf{Helical (spiral) scanning}: The patient table moves
  at a constant speed while the gantry rotates, causing the X-rays
  source to form a helix around the patient.
\end{enumerate}
\end{itemize}
\begin{figure}[H]
  \centering
  \begin{tabular}{cc}
    \includegraphics[height=3.5cm]{axial_scanning} & \includegraphics[height=3.5cm]{helical_scanning}
  \end{tabular}
  \caption{Types of scanning used in CT \cite{abdulla2025acquiring1}.\label{fig:scannings}}
\end{figure}

\section{Sinogram}
\begin{itemize}
\item In parallel- and fan-beam scanners, the projections
  are (1D) lines and the set of all projections for the same Z-plane
  form a 2D image called \popup{\emph{sinogram}}{The Radon transform
    describes how CT projections are mathematically related to the
    scanned object. Back-projection is the reconstruction method that
    (with filtering) approximates the inverse Radon transform to
    recover the object from its projections.}. In cone-beam scanners,
  the sinogram is a 3D cube \cite{wikipedia2025radom_transform}.
\begin{figure}[H]
\vspace{-1ex}
  \centering
  \includegraphics[height=5.0cm]{sinogram_generation}
  \caption{Sinogram generation example
    \cite{abdulla2025acquiring2}.\label{fig:sinogram_generation}}
\end{figure}
\end{itemize}

\begin{itemize}
\item The CT scanner generates a \popup{collection}{usually thousands}
  of projections that need to be processed to obtain the 3D
  reconstruction. See this
  \href{https://en.wikipedia.org/wiki/Radon_transform#/media/File:Radon_transform_sinogram.gif}{GIF
    sinogram example}.
\end{itemize}
\begin{figure}[H]
  \vspace{-0ex}
  \centering
  \begin{tabular}{cc}
    \includegraphics[height=5.0cm]{Two_Squares_Phantom} & \includegraphics[height=5.0cm]{Sinogram_-_Two_Square_Indicator_Phantom}
  \end{tabular}
  \caption{a 2D phantom (left) and its sinogram (right)
    \cite{wikipedia2025radom_transform}.\label{fig:sinogram_phantom}}
\end{figure}

\section{Tomogram reconstruction}
\begin{itemize}
\item A \popup{tomogram}{A 2D structure.} is the resulting 2D
  reconstruction of an \popup{slice}{A 2D structure.} scanned
  object by means of its \popup{sinogram}{A 2D structure}.
\item Tomograms can be reconstructed in the spatial domain (using the
  Filtered Back-Projection algorithm or iterative algorithms) and in
  the frequency domain (using the Radon transform).
\end{itemize}

\section{Filtered Back-Projection (FBP)}
\begin{itemize}
\item The tomogram structure accumulates in each voxel the
  contribution of each projection \cite{abdulla2025acquiring2} (see
  Figure~\ref{fig:CT_reconstruction})), that previously has been
  \popup{filtered}{The filtering is usually performed in the Fourier
    domain, where the convolution is a linear-time operation. The FFT
    (Fast Fourier Transform, both forward and inverse) requires
    n log2(n)) operations and the multiplication requires
    n operations, where n is the number of elements to
    convolve. Convolution in the signal domain requires
    n^2. Therefore, the projections must be FFT-transformed
    (n log2(n)), perform the point-wise multiplication
    (n), and inverse FFT-transformed
    (n log2(n)). Therefore, for large enough $n$,
    convolution in the Fourier domain is faster.} using a high-pass
  filter to sharpen the textures of the reconstruction.
\end{itemize}

\label{sec:FBP_example}
\vspace{-0ex}
\begin{figure}[H]
  \centering
  \includegraphics[height=8cm]{CT_backprojection}
  \caption{An example of reconstruction with FBP using only 2
    projections
    \cite{abdulla2025acquiring2}.\label{fig:CT_reconstruction}}
\end{figure}

\section{Iterative algorithms}
\begin{itemize}
\item Handle noise and incomplete data better but require more computation.
\item Basic algorithm:
  \begin{enumerate}
  \item Start with an initial guess of the image (for example obtained with FBP).
  \item Simulate what projections this guess would produce.
  \item Compare with the measured projections.
  \item Update the guess to reduce the error.
  \item Repeat until convergence.
  \end{enumerate}
\item Examples:
  \href{https://en.wikipedia.org/wiki/Algebraic_reconstruction_technique}{ART
    (Algebraic Reconstruction Technique)},
  \href{https://tomroelandts.com/articles/the-sirt-algorithm}{SIRT
    (Simultaneous Iterative Reconstruction Technique)}, and
  \href{https://arxiv.org/pdf/1504.06889}{Maximum Likelihood
    Expectation Maximization (MLEM)}.
\end{itemize}

\section{The Radon transform}
\begin{itemize}
\item The Radon transform models how a sinogram is generated. In other
  words, the Radon transform of an \popup{image}{In the context of CT,
    the input image is the tomogram that we want to reconstruct.} is
  its sinogram.
\item The inverse Radon transform, by definition, inputs a sinogram
  and outputs the original image.
\end{itemize}

\section{FFT-based inverse Radon transform}
\begin{itemize}
\item The reconstruction of a tomogram using the Radon transform can
  run faster (using the \popup{FFT}{The FFT (Fast Fourier Transform)
    is a fast algorith to ``travel'' between the signal domain and the
    frequency domain.}) with the following algorithm:
  \begin{enumerate}
  \item Compute the FFT of each projection, resulting a new sinogram
    where each row is in the frequency domain.
  \item Distribute each row in a circle (a image), depending on the
    angle at which each projection was taken.
  \item Compute the inverse 2D-FFT of the circle.
  \end{enumerate}
  \begin{figure}[H]
    \vspace{-1ex}
    \centering
    \includegraphics[width=0.85\textwidth]{Radon_transform_via_Fourier_transform}
    \caption{The inverse Radon transform using the frequency domain
      \cite{wikipedia2025radom_transform}.\label{fig:inverse_Radon}}
  \end{figure}
\end{itemize}

\section{(3D) Volume reconstruction}
\begin{itemize}
\item The volume is form by the stacking of all 2D tomograms.
\end{itemize}
\vspace{-1ex}
\begin{figure}[H]
  \centering
  \includegraphics[height=5cm]{Stacking-2D-slices-to-create-a-3D-volume-24}
  \caption{Tomogram stacking to generate a 3D structure
    \cite{alzu2019multi}.\label{fig:stacking}}
\end{figure}

\section{Quantum noise}
\begin{itemize}
\item In CT, as it does happen with all the X-rays-based imaging
  systems, quantum noise tends to dominate at high spatial frequencies
  in the projection data, which can lead to unacceptably noisy
  reconstructed images if appropriate filtering is not applied
  \cite{bushberg2011essential}.
\item See Section~\ref{sec:radiography_quantum_noise}.
\end{itemize}

%\chapter[\glsentrylong{PNI}]{\href{https://www.radiologycafe.com/frcr-physics-notes/molecular-imaging/planar-imaging/}{Planar Nuclear\\Imaging} (\gls{PNI})}
\vspace{-51ex}
\begin{flushright}
\includegraphics[width=5.5cm]{nuclearMedicineBoneScan} % https://www.hey.nhs.uk/nuclearmedicine/scanners-cameras-and-images/
\end{flushright}


\section{Acquisition}
\begin{itemize}
\item \gls{PNI} is a medical imaging technique that
  creates two-dimensional (2D) projection images of the
  three-dimensional (3D) distribution of radioactive materials within
  a patient \cite{bushberg2011essential}.
\item A radioactive isotope, incorporated into a chemical substance
  called a radiopharmaceutical, is administered to the patient
  (orally, by injection, or inhalation).
\item Once distributed according to the patient's physiological
  status, these radioisotopes \popup{emit}{For example, Technetium-99m
    (Tc-99m) emits 140-keV gamma rays, which are commonly imaged.}
  gamma rays, X-rays, or \popup{annihilation photons}{An annihilation
    photon is a type of high-energy photon produced during a unique
    interaction called annihilation, where a positron (a positively
    charged electron) combines with an electron.}.
\item Since X-rays and gamma rays are emitted isotropically (equally
  in all directions) from the radionuclide within the patient,
  \popup{collimators}{In general, an Anger scintillation camera is
    used for this.} are necessary to define the trajectory of photons
  reaching the detector and create a projection image.
\end{itemize}

\section{Clinical applications}
\begin{itemize}
\item PNI is used for functional imaging, providing insight into the
  physiological conditions (such as hyperthyroidism
  \cite{abdulla2025molecular_imaging}) rather than just the anatomy
  \cite{bushberg2011essential}.
\item Images can reveal ``hot spots'' (areas of increased
  radiopharmaceutical concentration, e.g., stress fractures or
  metastases) or ``cold spots'' (areas where normal concentration is
  altered, e.g., \popup{perfusion defects}{A perfusion defect refers
    to an area within an organ or tissue where the normal blood flow
    or distribution of a diagnostic agent is impaired or reduced. In
    medical imaging, particularly nuclear medicine, it indicates a
    functional or physiological abnormality, such as infarction
    (tissue death due to lack of blood supply) or ischemia (reduced
    blood flow), rather than a purely anatomical one.}).
\end{itemize}
\vspace{-4ex}
\begin{figure}[!b]
  \centering
  \includegraphics[width=6.5cm]{cardiac_amyloidosis}
  \caption{Hot spots showing the effects of \popup{Cardiac
      amyloidosis}{Amyloidosis occurs when the body produces amyloid
      proteins. Unlike other proteins, amyloid does not have a
      supportive role in the body. Instead, the buildup of amyloid
      protein leads to organ damage. Cardiac amyloidosis (CA) causes
      the heart to thicken and become inflexible due to abnormal
      deposits of protein in place of healthy heart tissue.}
    \cite{MNT_effects_cardiac_amyloidosis}.\label{fig:hot_spots}}
\end{figure}

\section{Quantum mottle}
\begin{itemize}
\item Due to patient safety reasons and the \popup{statistical
    nature}{Usually modeled by a Poisson distribution.} of the
  acquisition process, PNI usually generates images with a grainy
  appearance because the number of photons detected is, in most cases,
  very small.
\end{itemize}
\vspace{-4ex}
\begin{figure}[!b]
  \centering
  \includegraphics[width=6.5cm]{PNI_noise}
  \caption{Examples of quantum noise in \gls{PNI}.
    \cite{saridin2007quantitative}.}
  \label{PNI_noise}
\end{figure}

\section{Motion blur}
\begin{itemize}
\item As a consequence of images are acquired over \popup{many seconds
    or minutes}{Basically, to gain SNR.}, it is frequent to obtain
  moved images generated by \popup{patient motion}{For example,
    respiratory motion can not be avoided.}.
\end{itemize}
\vspace{-4ex}
\begin{figure}[!b]
  \centering
  \includegraphics[width=6.5cm]{PNI_motion}
  \caption{Example of motion blur in \gls{PNI}.
    \cite{naddaf2004technical}.}
  \label{PNI_motion}
\end{figure}

%\chapter[\glsentrylong{SPECT} (\glsentryshort{SPECT})]{Single Photon Emission\\Computed Tomography (SPECT)}
\vspace{-45ex}
\begin{flushright}
\includegraphics[width=4.5cm]{SPECT_example} % https://www.cidrad.com/wp-content/uploads/2024/06/testicular-ultrasound-riverside-county-ca-usa-near-me-400-600.jpg
\end{flushright}

\section{Acquisition}
\begin{itemize}
\item \gls{SPECT} is the tomographic counterpart of nuclear medicine
  planar imaging, just like CT is the tomographic counterpart of
  radiography \cite{bushberg2011essential,wikipedia_SPECT}.
  
\item In SPECT, a nuclear camera records X- or Gamma-ray emissions
  from the patient from a series of different angles around the
  patient.
  
\item  These projection data are used to reconstruct a series of
  tomographic emission images.

\item The spatial resolution of the images is inversely proportional
  to the distance between the patient and of the \popup{camera}{The
    camera uses a collimator to determine the direction of the
    photons. This generates a low detection efficiency because the
    collimator filters out over 99.9\% of the emitted photons. The
    design of the collimator inherently compromises between spatial
    resolution and detection efficiency}.
\end{itemize}

\section{Clinical applications}
\begin{itemize}
\item SPECT images provide diagnostic functional information similar
  to nuclear planar examinations (functional information about organ
  physiology) ; however, their tomographic nature allows physicians to
  better understand the precise distribution of the radioactive agent,
  and to make a better assessment of the function of specific organs
  or tissues within the body.
\item The same radioactive isotopes are used in both planar nuclear
  imaging and SPECT \cite{bushberg2011essential}.
\end{itemize}

\section{Image quality}
\begin{itemize}
\item The resolution is limited (hundres of pixels in each dimension)
  by two reasons:
  \begin{enumerate}
  \item The detection efficiency (which is very low) depends on the
    pixel-size.
  \item Each projection requires dozens of seconds
    \cite{abdulla2025SPECT}.
  \end{enumerate}
\item Althought it is based on the same imaging technology than PNI,
  it usually provided improved contrast and reduced structural noise
  by averaging counts from overlapping structures
\end{itemize}
%\chapter[\gls{PET}]{Positron Emission\\Tomography (PET)}
\vspace{-40ex}
\begin{flushright}
\includegraphics[width=6.5cm]{PET} % https://ksimg.com/wordpress/wp-content/uploads/2022/02/pet-ct.jpg
\end{flushright}

\section{Acquisition}
\begin{itemize}
\item Such as SPECT, PET is a tomographic nuclear imaging system
  \cite{abdulla2025PET}.
\item Employs \popup{annihilation coincidence detection
    (ACD)}{Positron-emitting radioisotopes decay and release a
    positron. This positron travels a short distance, then annihilates
    with an electron from the surrounding tissue, converting their
    mass into two 511-keV photons. These two photons are emitted
    simultaneously and in nearly opposite directions (approximately
    180 degrees apart). PET scanners detect these coincident photon
    pairs using rings of detectors and specialised circuitry. ACD is
    significantly more efficient than collimation and avoids the
    degradation of spatial resolution with increasing distance from
    the detector.} instead of collimation to determine the procedence
  of the radiation that reach the detector.
\item Only uses \popup{positron-emitting radionuclides}{F-18
    fluorodeoxyglucose (FDG) is the most widely used PET
    radiopharmaceutical. Due to their generally short half-lives, PET
    radionuclides often require a cyclotron to be located nearby or
    on-site}.
\item PET scanners are more ``open'' than SPECT scanners because the
  resolution of the images is not so dependent on the distance of the
  camera to the patiend.
\item PET is \popup{more expensive}{A PET/CT system can cost
    approximately twice that of a SPECT/CT system} than SPECT.
\end{itemize}

\section{Clinical applications}
\begin{itemize}
\item As SPECT, it is mainly used to \popup{visualise physiological
    function}{The most common application of PET (especially with F-18
    FDG) is in oncology for differentiating malignant neoplasms,
    staging cancer, and monitoring treatment response.}
  \cite{bushberg2011essential}.
\end{itemize}

\section{Image quality}
\begin{itemize}
\item Compared to SPECT, PET has much higher count rate sensitivity
  and,generating less noise.
\item Therefore, PET images can be reconstructed with much higher
  spatial frequency \cite{abdulla2025NIQ}.
\end{itemize}

%\chapter[\glsentrylong{MRI} (\glsentryshort{MRI})]{Magnetic\\Resonance\\Imaging (MRI)}
\vspace{-50ex}
\begin{flushright}
\href{https://www.sciencephoto.com/media/728494/view/normal-brain-mri}{\includegraphics[width=6.5cm]{Normal_brain_MRI}}
\end{flushright}

\section{Clinical applications}
\begin{itemize}
\item \gls{MRI}
  \cite{westbrook2018mri,Wu2022MRI_Physics,thePIRL2018NMR_basics,thePIRL2018SpinEcho,thePIRL2018Fourier,thePIRL2018GRE}
  allows to obtain detailed views of (usually living) samples
  (tissues, organs, and even a complete organism) without irradiate
  them \cite{wikipedia_MRI}.
\end{itemize}

\section{MRI axes}
\begin{itemize}
\item A patient is placed in the bore of the MRI machine to be
  subjected to $B_0$, a \popup{very strong magnetic field}{In the
    order of Teslas (T). 1 Tesla = 10,000 gauss and the Earth’s
    magnetic field is approx. 0.5 gauss.}, parallel to the $Z$-axis.
\end{itemize}
\vspace{-4ex}
\begin{figure}[!b]
  \centering
  \includegraphics[width=0.8\textwidth]{mri-machine-axes}
  \caption{Axes used in an MRI machine \cite{abdulla2025MRI_machine}.}
  \label{fig:MRI_axes}
\end{figure}

\section{The magnet responsible for $B_0$}
\begin{itemize}
\item $B_0$ is a permanent magnetic field generated by \popup{toroidal
    magnet}{A superconducting electromagnet refigerated with liquid
    helium at -269°C, that can weight several tons.}. A section is
  show in Fig.~\ref{fig:MRI_machine_scheme}.
\end{itemize}
\vspace{-4ex}
\begin{figure}[!b]
  \centering
  \includegraphics[width=0.7\textwidth]{mri-machine-coils}
  \caption{A schematic section of the MRI machine \cite{abdulla2025MRI_machine}.}
  \label{fig:MRI_machine_scheme}
\end{figure}

\section{Gradient coils}
\begin{itemize}
\item There are three sets of coils orientated in the $X$, $Y$ and $Z$
  axes, used to generate a gradient in $B_0$ (see
  Section~\ref{sec:spatial_encoding}).
\item Gradient coils generate a constant variation (gradient) $B_1$ to
  $B_0$ along the $Z$-axis, generating $B_0+B_1$.
\end{itemize}

\section{\glsentrylong{RF} (\glsentryshort{RF}) (body) coil ... and others}
\begin{itemize}
\item In Fig.~\ref{fig:MRI_machine_scheme} the ``body''-coil is shown,
  which is used to image large parts of the patient.
\item This coil transmits and receives \gls{EM} signals.
\item There are other specialized coils (not shown in the figure),
  such as:
  \begin{enumerate}
  \item \textbf{Head coil} (transmit and receive): incorporated into a
    helmet and used for head scans.
  \item \textbf{Surface} (or local) \textbf{coils} (receive only):
    these are small coils applied \popup{as close to the area being
      imaged as possible}{To increase the SNR.} e.g. arm, leg,
    orbits, lumbar spine coils etc.
  \item \textbf{Arrays of coils} that only transmit or receive, and that are
    then combined to improve the SNR.
  \end{enumerate}
\item The RF \cite{wikipedia_RF} coils are \popup{switched on and off
    rapidly}{Forming train of impulses of time-variying period.}, with
  a period of 1 ms or less, and it is this that creates the loud
  noise.
\item The smaller the distance between the coils and the patient, the
  better the strength of the signals, and the higher the \gls{SNR}.
\end{itemize}

\section{Hydrogen nuclei can be considered tiny magnets}
\begin{itemize}
\item Hydrogen nuclei act like tiny \popup{magnets}{A hydrogen nucleus
    contains a single proton so it has a charge of +1. The nucleus
    also has an intrinsic “spin”. Because they have a charge and
    motion they create an electric current and this, in turn, creates
    a magnetic field.} and will be affected by any magnetic field
  applied to them.
\item Hydrogen nuclei are \popup{the most useful atoms}{Any nucleus
    with an odd number of protons can be used (an unpaired proton is
    needed to provide the magnetic moment due to the spin of the
    unpaired proton).} to use in \gls{MRI} mainly because they form the
  majority of atoms in the body.
\item Most will adopt the low energy state (the direction of $B_0$)
  but a few in the opposite direction (the high energy state),
  creating a net longitudinal magnetization (M$_Z$) in the $Z$-axis
  direction.
\end{itemize}
\vspace{-4ex}
\begin{figure}[!b]
  \centering
  \includegraphics[width=0.7\textwidth]{mri-intro-netmag}
  \caption{Net magnetization under the influence of $B_0$ \cite{abdulla2025MRI_intro}.}
  \label{fig:MRI-intro-netmag}
\end{figure}

\section{Precession}
\begin{itemize}
\item Under the influence of $B_0+B_1$, the hydrogen nuclei precess in the Z-axis.
\item The precession has a characteristic \popup{resonant}{``Larmor''.}
  frequency for each \popup{atom and $B_0+B_1$ intensity}{42 MHz a 1 Tesla
    in the case of hydrogen}.
\end{itemize}
\vspace{-4ex}
\begin{figure}[!b]
  \centering
  \includegraphics[width=0.7\textwidth]{mri-intro-precession}
  \caption{Nuclei precession \cite{abdulla2025MRI_intro}.}
  \label{fig:MRI-intro-precession}
\end{figure}

\section{Resonance magnetization}
\begin{itemize}
\item When all the nuclei are precessing, an orthogonal to $B_0+B_1$
  ($Z$-axis) time-varying magnetization oscillating at the resonance
  frequency is applied from the RF coil, making the nuclei to resonate
  in the $XY$ plane.
\end{itemize}

\section{Relaxation $T_1$ and $T_2$ times}
\begin{itemize}
\item As soon as $B_1$ is switched off, the transverse magnetization
  begins to disappear and the nuclei relax back to their resting state
  of net longitudinal magnetization. 
\item \popup{$T_1$}{Also known as spin-lattice and longitudinal
    relaxation time because T_1 depends on the surrounding molecules
    and lattice.} is the time it takes for M$_Z$ (longitudinal
  magnetization) to recover to 63\% of its equilibrium state after
  being perturbed by the $B_1$ pulse.
\item \popup{$T_2$}{Also known as spin-spin relaxation time.} is
  defined as the time required for the M$_{XY}$ (transverse magnetization) to
  decay to 37\% of its peak level after the $B_1$ pulse.
\end{itemize}

\begin{itemize}
\item Relaxation times depends on the tissue and the strength of
  $B_0$. Some examples (for 1 Tesla) are show in the Table \ref{tab:relaxation_times}:
  \begin{table}
    \begin{center}
      \begin{tabular}{r|rr}
        Tissue & $T_1$ (ms) & $T_2$ (ms) \\
        \hline
        Fat & 250 & 80 \\
        Kidney & 550 &  60 \\
        White matter & 650 & 90 \\
        Grey matter & 800 & 100 \\
        CSF & 2000 & 150 \\
        Water & 3000 & 3000 \\
        Bone, teeth & Very long & Very short
      \end{tabular}
      \caption{Typical relaxation times for different tissues
        \cite{abdulla2025MRI_T1T2}.}
      \label{tab:relaxation_times}
    \end{center}
  \end{table}
\item $T_1$ is always longer than $T_2$ except in pure water in which
  $T_1=T_2$.
\end{itemize}

\section{$T_1$-weighted imaging}
\begin{itemize}
\item The contrast of the images depends on $T_1$.
\end{itemize}
\vspace{-3ex}
\begin{figure}[!b]
  \centering
  \includegraphics[width=4.5cm]{mri-t1-brain}
  \caption{MRI slice of the brain using $T_1$-weighting \cite{abdulla2025MRI_weighting}.}
  \label{fig:MRI-T1-weighting}
\end{figure}

\section{$T_2$-weighted imaging}
\begin{itemize}
\item The contrast of the images depends on $T_2$.
\end{itemize}
\vspace{-3ex}
\begin{figure}[!b]
  \centering
  \includegraphics[width=4.5cm]{mri-t2-brain}
  \caption{MRI slice of the brain using $T_2$-weighting \cite{abdulla2025MRI_weighting}.}
  \label{fig:MRI-T2-weighting}
\end{figure}

\section{PD-weighted imaging}
\begin{itemize}
\item The contrast of the images depends on the density of hydrogen
  protons. PD = Proton Density.
\end{itemize}
\vspace{-3ex}
\begin{figure}[!b]
  \centering
  \includegraphics[width=4.5cm]{mri-pd-brain}
  \caption{MRI slice of the brain using PD-weighting \cite{abdulla2025MRI_weighting}.}
  \label{fig:MRI-PD-weighting}
\end{figure}

\section{Spatial encoding}
\label{sec:spatial_encoding}
\begin{itemize}
\item Remember: The hydrogen nuclei resonate at a
  \popup{frequency}{Larmor frequency.} that depends on the strength of
  the magnetic field, $B_0+B_1$, which is a \popup{\gls{RF}
    signal}{$B_0$ is constant but $B_1$ is a time-variying
    signal. Obviously, the sum is also variable over time.}.
\item Therefore, by modifying $B_0+B_1$ in: (1) amplitude, (2)
  frequency, and (3) phase, we can know the resonating frequency (and
  phase) of a \popup{3D section}{A cuve of tissue.} of the patient.
\end{itemize}
\vspace{-4ex}
\begin{figure}[!b]
  \centering
  \includegraphics[height=4.0cm]{mri-spatial-localisation}
  \caption{Spatial ``encoding'' \cite{abdulla2025MRI_encoding}.}
  \label{fig:MRI-encoding}
\end{figure}

\begin{itemize}
\item Therefore, using \popup{phase-sensitive band-pass filters}{Also
    called complex filters.}, we can determine the amplitude of the
  signal generated by each 3D section (also known as \gls{FOV}).
\item This information is, by definition, a \popup{matrix of coefficient in the
    Fourier domain}{Fourier coefficients are complex numbers, with
    amplitude and plase, or alternatively, real and imaginary
    parts.}.
\item Such information is received by the RF coils that, apart from
  generate the gradients in the $XY$-plane, act as a RF antenna.
\end{itemize}

\section{The K-space}
\begin{itemize}
\item In its \popup{simplest version}{The K-space can be also a 3D
    tensor, and the reconstruction can be found using the 3D inverse
    Fourier transform.}, the K-space is a 2D matrix where the
  amplitude and phase of each 3D section of a slice is stored.
\end{itemize}
\vspace{-4ex}
\begin{figure}[!b]
  \centering
  \includegraphics[height=5.5cm]{mri-kspacewavesignal}
  \caption{The K-space and some single-coefficient inverse transformations \cite{abdulla2025MRI_Kspace}.}
  \label{fig:MRI-Kspace}
\end{figure}

\section{3D reconstruction}
\begin{itemize}
\item To reconstruct a slice, we need to compute the 2D inverse
  Fourier transform of the content of the corresponding K-space.
\item To obtain the final 3D reconstruction, all the reconstructed
  (2D) slices are stacked.
\item Notice that in MRI, the tomograms are in the Fourier domain.
\end{itemize}

\section{Image quality in \acrshort{MRI}}
\begin{enumerate}
\item \gls{SNR}: In the case of MRI, it depends on the \popup{strength
    of the signal received by the antenna}{Precession of coherent
    magnetization.}, and the \popup{strength of noise signal}{Random
    frequencies existing in space and time, primarily from thermal
    motion of the molecules (in the patient) and background electrical
    noise of the electronics}.
\item The \textbf{\gls{SNR}} increases with the \popup{strength of
    B0}{As the magnetic field strength increases, the Net Magnetic
    Vector (NMV) increases, leading to more available magnetization
    and consequently higher SNR. Doubling the field strength
    approximately doubles the SNR.} \cite{westbrook2018mri}, the
  \popup{proton density}{Areas with a high concentration of MR-active
    protons (e.g., the pelvis) yield higher signal and thus higher
    SNR, whereas areas with low proton density (e.g., the lungs)
    result in lower signal and SNR.} \cite{westbrook2018mri}, the
  \popup{coil(s) efficiency and distance to the sample}{Basically, the
    SNR is proportional to the inverse of the diameter of the
    tube. The power of the excitation RF signals is also proportional
    to the SNR.}, the \popup{\emph{signal scanning times}}{Longer TR
    (Time-Repetition) allows for greater longitudinal magnetization
    recovery, making more signal available for conversion to
    transverse magnetization, which typically improves SNR. Shorter TE
    (Time-Echo) allows less coherent transverse magnetization to decay
    before the echo is collected, resulting in higher SNR.}
  \cite{westbrook2018mri}, the \popup{Number of Signal Averages
    (NSA)}{Increasing the NSA directly increases SNR, as correctly
    encoded signal is reinforced while random noise averages out.}
  \cite{westbrook2018mri}, the \popup{size of the voxels}{Larger
    voxels contain more spins, which contribute to a higher signal and
    consequently increased SNR.}  \cite{westbrook2018mri}, and depends
  on the reconstruction algorithms \cite{bushberg2011essential}.
\item \textbf{Spatial resolution}: Depends fundamentally on the \emph{minimal
    slice-thickness} provided by the internal coils, the
  resolution of the k-space, and on the magnetic field
    strength ($B_0$) to provide a high enough \popup{SNR per voxel}{The SNR
    increases with $B_0$ and the voxel size, so, to increase the
    resolution (decrease the voxel size) we must keep high enough the
    SNR by increasing $B_0$. Otherwise, the noise can make it difficult
    to recognize the pathology}.
\item \textbf{Contrast}: Image contrast refers to the differences in
  signal intensity between various anatomical features, between
  anatomy and pathology, or between different tissues
  \cite{westbrook2018mri}. Contrast can be increased if we use
  \popup{\emph{image weighting}}{For example, T2-weighted volumes
    usually enhances pathologies.}, \popup{\emph{contrast
      agents}}{E.g., gadolinium can selectively shorten relaxation
    times, increasing the contrast between pathology and normal
    anatomy-}, among \popup{\emph{other MRI contrast-enhancing
      techniques}}{Magnetization transfer contrast, phase-contrast MR
    angiography, or the use of presaturation pulses}
  \cite{westbrook2018mri}.
\end{enumerate}


\part{Storage of medical Images}
\chapter{Basic concepts}

\section{Storage charactaristics}
\begin{itemize}
\item \textbf{Capacity}: This is the total amount of data that a
  storage medium can hold (see Table~\ref{tab:memory_caps}).
  \begin{table}[!h]
    \begin{center}
      \begin{tabular}{r|l}
        Acronym & Amount of data \\
        \hline
        \popup{B}{byte}s & (8 \popup{bits}{where a bit represents a logical state with one of two possible values.})\\
        \popup{KB}{kilobyte}s & $1\text{KB} = 2^{10}\text{B}$\\
        \popup{MB}{megabyte}s & $1\text{MB} = 2^{10}\text{KB}$\\
        \popup{GB}{gigabyte}s & $1\text{GB} = 2^{10}\text{MB}$\\
        \popup{TB}{terabyte}s & $1\text{TB} = 2^{10}\text{GB}$\\
        \popup{PB}{petabyte}s & $1\text{PB} = 2^{10}\text{GB}$
      \end{tabular}
      \caption{Memory capacity measures \cite{wikipedia_KB}.}
      \label{tab:memory_caps}
    \end{center}
  \end{table}
  %\newpage
\item \textbf{Volatility}: If the storage media need to connected to a
  current supply (for example, the \gls{RAM} memory of a computer),
  the media is said \emph{volatile}.
\item \textbf{\gls{WORM}}: A \gls{CD-ROM}, for example.
\end{itemize}

\section{The Cloud}
\begin{center}
  \vspace{-4ex}
  \includegraphics[height=3.5cm]{cloud-storage}
\end{center}
\begin{itemize}
\item Data is stored in remote data-centers accessed over the Internet.
\item Fully scalable (more money, more space).
\item We don't control where the data is.
\item We don't control who has access to the data.
\end{itemize}
\begin{center}
  \vspace{-1ex}
  \includegraphics[width=3.5cm]{oracle}
\end{center}

\section{\glsentrylong{NAS} (\glsentryshort{NAS})}
\begin{itemize}
\item Data is stored in a \popup{specialized computer}{The computer
    rarely has a keyboard or monitor, and has many hard drive bays.}
  connected to the \gls{LAN} \cite{wikipedia_NAS}.
\item Usually mounts several \popup{disk}{Or disc.}s with some type of
  \glsentryshort{RAID} (\glsentrylong{RAID}) configuration.
\end{itemize}
\begin{figure}[H]
  \vspace{-1ex}
  \centering
  \includegraphics[height=4cm]{servidor-nas}
  \caption{A piece of a \gls{NAS}.}
  \label{fig:NAS}
\end{figure}

\section{\glsentrylong{RAID} (\glsentryshort{RAID})}
\begin{itemize}
\item A \gls{RAID} is a logical disk that is able to work even when
  some of the \popup{physical disks}{That actually store the data.}
  fail \cite{wikipedia_RAID}. These are some of the existing configurations:
  \begin{enumerate}
  \item RAID-0 (Striping): \popup{No redundancy}{To maximize capacity,
      splits data across drives. This means that if a physical disk
      stops working, a loss of data will happen.}.
  \item RAID-1 (Mirroring): \popup{Maximum redundancy}{All physical
      disks contain the same data. To lost data all disks must fail at
      the same time.}.
  \item RAID-5 (Striping with Parity): \popup{One disk
      redundancy}{This configuration can only tolerate the failure of
      one disk.}. When the broken disk is replaced, the RAID must
    rebuild the parity information. During this time, no other disk
    can fail.
  \item RAID-6 (Double Parity): \popup{Two disks
      redundancy}{Tolerate the failure of
      two disks at the same time.}.
  \end{enumerate}
\end{itemize}

\section{\glsentrylong{HDD} (\glsentryshort{HDD})}
\begin{itemize}
\item Electro-mechanical data storage-persistent device that uses
  magnetic storage to store and retrieve digital information
  \cite{wikipedia_NDD}.
\end{itemize}
\begin{figure}[H]
  \vspace{-1ex}
  \centering
  \includegraphics[height=5cm]{HDD}
  \caption{A hard disk.}
  \label{fig:HDD}
\end{figure}

\section{\glsentrylong{SSD} (\glsentryshort{SSD})}
\begin{itemize}
\item Fully electronic storage device based on \popup{flash
    memory}{Flash memory is a type of non-volatile computer memory
    that can be electrically erased and reprogrammed. Unlike volatile
    memory (like RAM), it retains data even when the power is turned
    off.} chips \cite{wikipedia_SSD}.
\item Compared to \gls{HDD}s, \gls{SSD}s are faster, more reliable and
  power efficient, but for the same capacity, \gls{SSD}s are more
  expensive.
\end{itemize}
\begin{figure}[H]
  \vspace{-1ex}
  \centering
  \includegraphics[height=4.5cm]{SSD}
  \caption{A solid state hard disk.}
  \label{fig:SSD}
\end{figure}

\section{\glsentrylong{CD-ROM} ({\glsentryshort{CD-ROM})}}
\begin{itemize}
\item Optical storage medium that uses a laser to read digital data
  encoded on a spinning disc \cite{wikipedia_CD-ROM}.
\item The data is stored as a spiral track of microscopic depressions
  genearted on a reflective aluminum layer encased in plastic.
\item Typical capacity: 800 GB.
\end{itemize}
\begin{figure}[H]
  \vspace{-4ex}
  \centering
  \includegraphics[height=4.5cm]{CD-ROM}
  \caption{A \gls{CD-ROM} disk.}
  \label{fig:CD-ROM}
\end{figure}
  
\section{\glsentrylong{CD-R} ({\glsentryshort{CD-R})}}
\begin{itemize}
\item A \gls{CD} that can be written once \cite{wikipedia_CD-R}.
\end{itemize}
\begin{figure}[H]
  \vspace{-1ex}
  \centering
  \includegraphics[height=5cm]{CD-R}
  \caption{A \gls{CD-R} disk.}
  \label{fig:CD-R}
\end{figure}

\section{\glsentrylong{CD-RW} ({\glsentryshort{CD-RW})}}
\begin{itemize}
\item A \gls{CD} that can be written several times \cite{wikipedia_CD-RW}.
\end{itemize}
\begin{figure}[H]
  \vspace{-1ex}
  \centering
  \includegraphics[height=5.0cm]{CD-RW}
  \caption{A \gls{CD-RW} disk.}
  \label{fig:CD-RW}
\end{figure}

\section{\glsentrylong{DVD} ({\glsentryshort{DVD})}}
\begin{itemize}
\item The same than a \gls{CD-ROM} but with a higher capacity
  \cite{wikipedia_DVD}.
\item Typical capacity: 4.7 GB.
\item As with \gls{CD}, there are -R and -RW versions.
\begin{figure}[H]
  \vspace{-14ex}
  \begin{flushright}
    \includegraphics[height=6.0cm]{DVD-ROM}
    \caption{A \gls{DVD} disk.}
    \label{fig:DVD}
  \end{flushright}
\end{figure}
\end{itemize}

\section{Blu-Ray}
\begin{itemize}
\item The same than a \gls{DVD-ROM} but with a higher capacity
  \cite{BR}.
\item Typical capacity: Up to 100 GB.
\item As with \gls{DVD}, there are -R and -RW versions.
\end{itemize}
\begin{figure}[H]
  \vspace{-16ex}
  \begin{flushright}
    \includegraphics[height=6.0cm]{Blu-ray}
    \caption{A Blu-Ray disk.}
    \label{fig:Blu-Ray}
  \end{flushright}
\end{figure}

\section{Files and streams}
\begin{itemize}
\item A \popup{file}{Also known as ''archive''.} is a collection of data \popup{stored}{Files are
persistent: once written, they stay there until deleted.} on a storage
medium (for example, a NAS) with a defined structure and a
name. Example: a DICOM file.
\item A stream is a continuous flow of data that is transmitted and
processed in real-time, often without being stored
permanently. Example: a videcon between a patient and a doctor.
\item Files can be \popup{accesed randomly}{We can move over the file
to read or modify a part of it.}. Streams cannot (only are accesed
sequentially).
\end{itemize}

\section{Formats}
\begin{itemize}
\item Files and streams must follow some predefined structure and
encodings that indicate how to recover the information contained.
\item Most of the image formats used in medicine follow some standard
which define they, such as for example, the DICOM format.
\end{itemize}

\section{Data compression}
\begin{itemize}
\item Images requires large amounts of data to be represented. Data
compression define the objective of an efficient encoding system:
reduce the lendth of files and streams.
\item Data compressors can be:
\begin{enumerate}
\item \textbf{Lossless}: If after the decompression we recover all the
compressed information, to the point that the original file/stream can
be regenerated.
\item \textbf{Lossy}: when not. The advantage is that the compression
ratios are much higher, and sometimes the loss can be aceptable.
\end{enumerate}
\end{itemize}

\section{Raster versus vectorial graphics}
\begin{itemize}
\item \textbf{Raster graphics}, also known as bitmaps, are composed of
  a fixed grid of tiny squares called pixels. Each pixel contains a
  specific color and together they form a complete image.
\item \textbf{Vector graphics} are created using mathematical formulas
  that define points, lines, curves, and shapes. The image stores
  instructions on how to draw the image rather than the image itself.
  For this reason, vector graphics are resolution-independent.
\end{itemize}
\begin{figure}[H]
  \vspace{-2ex}
  \centering
  \includegraphics[height=4.0cm]{raster-vs-vector}
  \caption{Raster and vectorial images.}
  \label{fig:raster_vs_vector}
\end{figure}

\section{Long-term persistency, data sequrity and privacy}
\begin{itemize}
\item Medical images contain important information and should be
  stored in long-term persistence storage systems.
\item Patient images are protected health information. Therefore,
  \gls{PACS} must follow laws like HIPAA (USA) or GDPR (Europe) for
  confidentiality.
\item Medical images should only be accessed by authorized healthcare
  professionals.
\end{itemize}

\chapter{PNG (Portable Network Graphics)}

\section{Main characteristics}
\begin{itemize}
\item Raster graphics file format.
\item \href{https://www.w3.org/}{W3C} standard (supported by all Web browsers).
\item Lossless compression.
\item Up to $2^{32}-1$x$2^{32}-1$ pixels.
\item RGBA and gray-scale images.
\item Up to 16 bits/channel.
\item Motion PNG mode.
\item Medical metadata not supported :-/.
\end{itemize}

\section{\href{https://www.w3.org/}{W3C} international standard}
\begin{itemize}
\item Supported by all Web browsers, and most image viewers.
\end{itemize}
%\vspace{-4ex}
\begin{center}
  \href{https://upload.wikimedia.org/wikipedia/commons/0/05/CT_of_a_normal_abdomen_and_pelvis%2C_coronal_plane_79.png}{\includegraphics[width=6cm]{CT_example}}\\
     (click on the image)
\end{center}

\section{Raster graphics file format}
\begin{itemize}
\item The image is a 2D array of pixels.
\item Pixels can be gray-scale (1 channel) or color \popup{RGBA}{Red,
    Green, Blue, and Alpha (transparency) channels. The alpha-channel
    is optional.} (4 channels).
\end{itemize}
%\vspace{-4ex}
\begin{center}
  \href{https://en.wikipedia.org/wiki/Raster_graphics}{\includegraphics[width=5cm]{RGB-raster-image}}
\end{center}

\section{8 and 16 bits/channel}
\begin{itemize}
\item In gray-scale image:
  \begin{center}
    \begin{tabular}{c|c}
      & Number of \\
      Bits/channel & gray-scale values \\
      \hline
      8 & $2^8=256$ \\
      16 & $2^{16}=65536$
    \end{tabular}
  \end{center}
\end{itemize}
\chapter{\glsentrylong{JPEG}}

\section{\glsentrylong{ITU} international standard}
\begin{itemize}
\item Developed by the \gls{JPEG} (\href{https://www.itu.int}{ITU}),
  the \gls{JPEG} standard \cite{ccitt.t81} in 1992, it is supported by
  all Web browsers, and most image viewers.
\end{itemize}
\vspace{-2ex}
\begin{center}
  \href{https://en.wikipedia.org/wiki/Magnetic_resonance_imaging_of_the_brain#/media/File:MRI_of_Human_Brain.jpg}{\includegraphics[width=4.0cm]{MRI_of_Human_Brain}}\\
  (click on the image)
\end{center}

\section{Lossy compression of color natural images}
\begin{itemize}
\item Designed for achieve high compression ratios, but at the expense of reducing quality.
  \item Raster images with up to ($2^{16}-1$)x($2^{16}-1$) pixels.
  \item Pixels must be gray-scale or \gls{RGB}, 8 bits/channel.
\end{itemize}
\vspace{-2ex}
\begin{center}
  \href{https://www.thewebmaster.com/jpeg-definitive-guide/}{\includegraphics[width=8.5cm]{JPEG_Quality_vs_Size}}
\end{center}

\section{Algorithm (1/2)}
\begin{enumerate}
\item Convert from the \gls{RGB} color space to the \gls{YCrCb} color
  space. Only if the input image is in color.
\item Subsample the crominance (Cr and Cb channels). /* Lossy step */
\item Divide each channel in blocks of size 8x8. \popup{The rest of
    steps work by blocks}{This makes it possible to work at the block
    level regardless of the image size.}.
\item Transform each block using the \gls{DCT}.
\item Quantize the \gls{DCT} coefficients. /* Lossy step */
\item Entropy encode the quantized coefficients.
\end{enumerate}

\section*{Algorithm (2/2)}
\begin{itemize}
\item A graphic description of a compression of a gray-scale image.
\end{itemize}
\vspace{-2ex}
\begin{center}
  \href{https://link.springer.com/article/10.1007/s40799-019-00358-4}{\includegraphics[width=8.5cm]{JPEG}}
\end{center}

\section{Artifacts}
\begin{itemize}
\item The human eye is sensitive to the 8x8-blockiness.
\end{itemize}
\vspace{-2ex}
\begin{center}
  \href{https://thesai.org/Publications/ViewPaper?Volume=6&Issue=4&Code=ijacsa&SerialNo=16}{\includegraphics[width=10cm]{JPEG_blocking}}
\end{center}

\section{Progressive rendering}
\begin{itemize}
\item Optinal. Blocks are reconstructed coefficient-by-coefficient following the Zig-Zag ordering.
\item During a progressive visualization, blocks display higher
  spatial frequencies.
\end{itemize}
\begin{center}
  \href{https://es.m.wikipedia.org/wiki/Archivo:Zigzag_scanning.jpg}{\includegraphics[width=8cm]{Zigzag_scanning}}
\end{center}

\section{Metadata (1/2)}
\begin{itemize}
\item \gls{JPEG} images can store metadata as
  \href{https://en.wikipedia.org/wiki/Exif}{Exif} data (see the \href{https://gitlab.com/TNThieding/exif/-/blob/master/docs/api_reference.rst?ref_type=heads}{Exif implementation}). Some fields are:
  \vspace{-2ex}
  \begin{center}
    \begin{tabular}{l|l}
      Keyword & Meaning\\
      \hline
      ImageWidth & Width of the image in pixels \\
      ImageLength & Height of the image in pixels \\
      BitsPerSample&  Number of bits per color \\
      Make & Manufacturer of the camera or scanner \\
      Model & Model of the camera or scanner \\
      Software & Software used to process the image \\
      Orientation & Of the camera when the image was taken \\
      DateTimeOriginal & Date and time the image was originally captured \\
      ExposureTime & hutter speed \\
      FNumber & Lens aperture setting \\
      FocalLength & Focal length of the lens \\
      UserComment & User's comments about the image
    \end{tabular}
  \end{center}
\end{itemize}

\section*{Metadata (2/2)}
\begin{itemize}
\item
  \href{https://github.com/vicente-gonzalez-ruiz/medical_imaging/blob/main/notebooks/JPEG_add_metadata.ipynb}{Here}
  there is an example that shows and modifies the metadata in a JPEG
  image.
\end{itemize}

\chapter{JPEG2000}

\section{\acrshort{ISO} international standard}
\begin{itemize}
\item Developed by the \gls{JPEG} (ISO/IEC 15444\href{https://www.itu.int}{ITU}),
  the JPEG2000 standard \cite{taubman2002jpeg2000} in 2000, as a successor of
  \gls{JPEG}.
\item Mainly used in medical imaging and digital cinema.
\end{itemize}
\vspace{-2ex}
\begin{center}
  \href{https://en.wikipedia.org/wiki/Magnetic_resonance_imaging_of_the_brain#/media/File:MRI_of_Human_Brain.jpg}{\includegraphics[width=4.0cm]{MRI_of_Human_Brain}}\\
  (click on the image)
\end{center}

\section{Lossless and lossy compression}
\begin{itemize}
\item Two different compression modes:
\item \textbf{Reversible}: Allows a perfect reconstruction of the raster image (like \gls{PNG}).
\item \textbf{Irreversible}: Unable to recover all the visual
  information, but offers a better rate/distortion performance than
  the reversible mode for the same bit-rate.
\end{itemize}

\section{16-bit per channel and several color spaces support}
\begin{itemize}
\item \popup{16 bits/pixel offers enough quality for specialized
    applications such as medicine, atronomy, etc.}{It is quite
    difficult to increase the SNR above of 16 bits because basically
    we register noise when the amplitude of signal is small.}.
\item Apart from gray-scale and \gls{RGB}, JPEG2000 supports
  \gls{YCrCb}, \gls{sRGB} \cite{sRGB_wikipedia}, \gls{CIELAB}
  \cite{CIELAB_wikipedia}, and \popup{custom}{It is posible to define
    a color transform.} \cite{houchin2001specification} color spaces.
\end{itemize}

\section{Superior compression efficiency than \gls{JPEG}}
\begin{itemize}
\item Better quality at the same bitrate compared to JPEG \cite{vruiz_J2K}.
\end{itemize}
\begin{center}
  \begin{tabular}{cc}
    \multicolumn{2}{c}{Lena at 0.1 bits/pixel} \\
    \includegraphics[width=5cm]{lena_01} & \includegraphics[width=5cm]{lena_01_jp2} \\
    JPEG & JPEG2000
  \end{tabular}
\end{center}

\section{Algorithm}

\section{Lossy artifacts}

\section{Scalability}
region of interest coding

\section{Error resilience}

\section{Motion JPEG2000 (digital cinema)}

\section{Metadata}
\chapter{\glsentrylong{MPEG}}

\section{A collection of standards}
\begin{itemize}
\item Developed by the \gls{MPEG} (\href{https://www.itu.int}{ITU}),
  in 1992.
\item Define different algorithms to compress sequences of raster
  images (videos) and the corresponding audio.
\item All are lossy encoders.
\end{itemize}

\section{Versions}
\begin{enumerate}
\item \href{https://en.wikipedia.org/wiki/MPEG-1}{\textbf{MPEG-1}}
  (1993): Designed to store a movie in a \gls{CD} (\gls{VCD}).
\item \href{https://en.wikipedia.org/wiki/MPEG-2}{\textbf{MPEG-2}}
  (1996): Used in \glspl{DVD}, \href{https://en.wikipedia.org/wiki/Satellite_television}{satellite TV} and \href{https://en.wikipedia.org/wiki/Digital_television}{terrestial TV} (\gls{DVB}).
\item \href{https://en.wikipedia.org/wiki/MPEG-4}{\textbf{MPEG-4}}
  (1998): Used mainly in media streaming on the Internet.
\end{enumerate}

\section{Algorithm (1/2)}
\begin{enumerate}
\item Convert to the \gls{YCbCr} color space, and
  \href{https://en.wikipedia.org/wiki/Chroma_subsampling}{subsample to
    4:2:0} (if the input is not yet in this format). /* Lossy */
\item Divide each channel in macro-blocks (MCs) of size 16x16. \popup{The
    rest of steps work by MCs}{This makes it possible to work
    at the MC level regardless of the image size.}.
\item Compensate the motion of each MC using MCs of the adjacent
  images. This step generates \popup{residue}{Residue MCs follow a
    Laplace distribution with mean 0, and its entropy is smaller than
    the original MCs (this means that a residue MCs is more
    compressible than an original MC, on average).} MCs and a motion
  vector per MC.
\item Divide each MC in 8x8 blocks.
\item Transform each block using the \gls{DCT}.
\item Quantize the \gls{DCT} coefficients. /* Lossy */
\item Entropy encode the quantized coefficients and the motion
  vectors.
\end{enumerate}

\section*{Algorithm (2/2)}
\begin{itemize}
\item A graphic description of a compression of a gray-scale sequence of images.
\end{itemize}
\vspace{-2ex}
\begin{center}
  \href{https://w3.ual.es/~vruiz/Docencia/Apuntes/Coding/Video/02-MPEG1/index.html}{\includegraphics[width=8.5cm]{MPEG-1_compressor}}
\end{center}

\section{Blocking artifacts}
\begin{center}
  \href{https://filmora.wondershare.com/video-editing/video-compression-artifacts.html}{\includegraphics[width=8.5cm]{MPEG_artifacts}}
\end{center}

\chapter{\glsentrylong{AVC} (H.264)}

\section{Basics}
\begin{itemize}
\item Lossy and \popup{lossless}{Although by a large margin, it is
    mainly used as a lossy video codec.} \cite{wikipedia_AVC}.
\item Standarized by the \gls{MPEG} and the \gls{VCEG} in 1999.
\item Is another (like \gls{MPEG}) video compression standard based on
  block-oriented, motion-compensated coding.
\end{itemize}

\section{Algorithm}
\label{sec:MPEG-4_AVC_algo}
\begin{itemize}
\item Similar to a MPEG-1 codec (see Section~\ref{sec:MPEG-1_algo}),
  but \gls{AVC}:
\begin{enumerate}
\item Can use variable MB sizes (16x16 down to 4x4, not
  necessarily squared)).
\item
  \href{https://en.wikipedia.org/wiki/Motion_compensation}{Quarter-pel
    motion compensation}.
\item Compensate the motion of each MC using MCs of the
  \popup{neighbor images}{The MCs of the frame N, are compensated
    using the MCs of a range of neighbor frames (not only the contiguous).}.
\item A \gls{RDO}-improved intra coding mode based on \gls{LPC}.
\item A \gls{RDO}-improved entropy coding using \gls{CABAC}.
\item In-Loop deblocking filter.
\end{enumerate}
\end{itemize}

\section{Tylical artifacts and effect of the deblocking filter}
\begin{center}
  \href{https://www.sciencedirect.com/science/article/pii/B9780124157606000167}{\includegraphics[width=0.9\textwidth]{AVC_artifacts}}
\end{center}

\chapter{\glsentrylong{HEVC} (H.265)}
\label{cha:HEVC}

\section{Basics}
\begin{itemize}
\item Lossy and \popup{lossless}{Although by a large margin, it is
    mainly used as a lossy video codec.} \cite{wikipedia_HEVC}.
\item Standarized by the \gls{MPEG} and the \gls{VCEG} in 2013.
\item Is another (like \gls{MPEG}) video compression standard based on
  block-oriented, motion-compensated coding.
\end{itemize}

\section{Algorithm}
\label{sec:HEVC_algo}
\begin{itemize}
\item Similar to a \gls{AVC} codec (see Section~\ref{sec:MPEG-4_AVC_algo}),
  but \gls{HEVC}:
\begin{enumerate}
\item Larger MB (\gls{CTU}) sizes (up to $64\times 64$).
\item $32\times 32$ \gls{DCT}.
\item 1/8 pel motion compensation accuracy.
\item Larger prediction neighborhoods.
\item 35 intra prediction modes (directions).
\item Improved context modeling in \gls{CABAC}.
\item Improved motion vectors entropy coding.
\item Better deblocking filters.
\end{enumerate}
\end{itemize}

\section{Compression gain}

\begin{figure}[H]
  \vspace{+2ex}
  \centering
  \href{https://www.epiphan.com/blog/h264-vs-h265/}{\includegraphics[width=\textwidth]{AVC-vs-HEVC}}
  \caption{\gls{HEVC} VS H.264/\gls{AVC}.}
  \label{fig:HEVC_vs_AVC}
\end{figure}

\begin{center}
\end{center}

\chapter{\gls{VVC} (H.266)}

\section{Basics}
\begin{itemize}
\item Lossy and \popup{lossless}{Although by a large margin, it is
    mainly used as a lossy video codec.}.
\item Standarized by the \gls{MPEG} and the \gls{VCEG} in 2020.
\item Is another (like \gls{MPEG}) video compression standard based on
  block-oriented, motion-compensated coding.
\end{itemize}

\section{Algorithm}
\begin{itemize}
\item Similar to a \gls{HEVC} codec (see Section~\ref{sec:HEVC_algo}),
  but \gls{VVC}:
\begin{enumerate}
\item \glspl{CTU} up to 128×128, allowing even triangular binary splits.
\item Affine motion compensation (including rotation and zoom).
\item Bi-directional optical flow to find the motion vectors.
\item 64x64 \gls{DCT}.
\item 67 intra prediction modes (directions) and adaptive filtering.
\item Improved context modeling in \gls{CABAC}.
\item Adaptive deblocking filters.
\end{enumerate}
\end{itemize}

\section{VVC vs HEVC vs AVC}
\begin{center}
  \href{https://thebroadcastknowledge.com/2020/11/25/video-the-new-video-codec-landscape-vvc-evc-hevc-lc-evc-av1-and-more/}{\includegraphics[width=0.9\textwidth]{CTUs}}
\end{center}

\section*{VVC vs HEVC vs AVC}
\begin{center}
  \href{https://www.linkedin.com/pulse/video-coding-standards-comparison-sraas}{\includegraphics[width=\textwidth]{AVC_HEVC_VVC}}
\end{center}

\chapter{\glsentryshort{DICOM} file formats}

\section{What is DICOM?}
\begin{itemize}
\item \popup{Open}{Openly accessible and usable by anyone.}
  \popup{standard}{Actually is a set of standards.} created in 1992 by
  the \gls{ACR} and the \gls{NEMA} \cite{DICOM2025,wikipedia_DICOM}.
\item It defines how to transfer (transport layers protocols:
  \gls{TCP} or \gls{UDP}), storage (filenames, structure of the
  filesystem, etc.), processing, display (such as \gls{GSDF}),
  perception, and use of information in medicine
  \cite{bushberg2011essential}, allowing devices and systems from
  different manufacturers to communicate and share medical image data
  seamlessly.
\end{itemize}.

\section{What is DICOM file?}
\begin{itemize}
\item \gls{DICOM} is a media container for medical signals
  (\gls{ECG} signals, \gls{PCG} audios, 2D images, 3D images, and videos).
\item A DICOM file, such as a single CT slice, consists of two
  distinct parts:
  \begin{enumerate}
  \item The \textbf{header} is a block of data that contains specific
    \popup{information}{Such as name, age, gender, date of birth, and
      technical data about the image (device used, resolution, codec,
      etc.).}  that complements the image
    (\popup{attributes}{Depending on the image and the circumstances,
      certain information is mandatory, while other attributes are
      optional.}), classified as
    \href{https://dicom.nema.org/medical/dicom/current/output/html/part06.html#PS3.6}{DICOM
      tags}.
  \item The \popup{\textbf{image} itself}{The signal in
      general.}. Notice that the metadata provided by \gls{DICOM} are
    different from the image's metadata.
  \end{enumerate}
\end{itemize}.

\section{Supported 1D-signals codecs}
\begin{itemize}
\item \gls{ECG} and \gls{PCG} signals are \popup{\gls{PCM}
    encoded}{Without any loss or data compression.} with up to 24
  bits/sample.
\end{itemize}

\section{Supported image codecs}
\begin{enumerate}
\item Lossless:
  \begin{enumerate}
  \item \popup{Raw}{As generated by the ADC (Analog Digital
      Converted.}
    \href{https://en.wikipedia.org/wiki/Endianness}{Little Endian}
    with up to 64 bits/grayscale-pixel
    (\href{https://en.wikipedia.org/wiki/Double-precision_floating-point_format}{double-precision
      floating-point format}).
  \item \gls{RLE}.
  \item \href{https://en.wikipedia.org/wiki/Lossless_JPEG}{JPEG Lossless}.
  \item \href{https://en.wikipedia.org/wiki/Lossless_JPEG\#JPEG_LS}{JPEG-LS}.
  \item \href{https://en.wikipedia.org/wiki/JPEG_2000}{JPEG 2000}
    (reversible path).
  \item \href{https://en.wikipedia.org/wiki/JPEG_XL}{JPEG XL}
    (reversible path).
  \item \href{https://en.wikipedia.org/wiki/Deflate}{Deflate}. Similar
    to \gls{PNG}.
  \end{enumerate}
\item Lossy:
  \begin{enumerate}
  \item \gls{JPEG}.
  \item \href{https://en.wikipedia.org/wiki/JPEG_2000}{JPEG 2000}
    (ireversible path).
  \item \href{https://en.wikipedia.org/wiki/JPEG_XL}{JPEG XL}
    (ireversible path).
  \end{enumerate}
\end{enumerate}

\section{Supported video codecs}
\begin{enumerate}
\item \gls{JPIP} (see Chapter~\ref{cha:JPEG2000}).
\item \gls{MPEG}-2 (see Chapter~\ref{cha:MPEG}).
\item \gls{HEVC} (see Chapter~\ref{cha:HEVC}).
\end{enumerate}

\section{Storage (example)}
\begin{verbatim}
[USB Drive Root]          <- We are using an USB drive
+-- DICOMDIR              <- Binary File
+-- DICOM                 <- Folder
    +-- STUDY01           <- Folder
    |   +-- SERIES01      <- Folder
    |   |   +-- IMAG0001  <- Binary file
    |   |   +-- IMAG0002  <- Binary file
    |   |   +-- IMAG0003  <- Binary file
    |   |-- SERIES02      <- Folder
    |   |   |-- IMAG0004  :
    |   |   |-- IMAG0005
    +-- STUDY02
        +-- SERIES01
            +-- IMAG0006
            +-- IMAG0007
            +-- IMAG0008
\end{verbatim}

Where the file \texttt{DICOMDIR} defines the content of the rest of the file system:
\begin{verbatim}
(0004,1130) DICOMDIR FILE
    (0004,1200) Patient Record: John Doe
        (0004,1200) Study Record: CT Brain 2025-09-02
            (0004,1200) Series Record: CT Scout
                (0004,1500) Image File Record: DICOM/STUDY01/SERIES01/IMAG0001
            (0004,1200) Series Record: CT Axial
                (0004,1500) Image File Record: DICOM/STUDY01/SERIES02/IMAG0002
                (0004,1500) Image File Record: DICOM/STUDY01/SERIES02/IMAG0003
            :
\end{verbatim}


\part{Transmission of medical images}
%\part{Medical Images Transmission}

WWW technology is used for cost-effective image distribution to clinicians, often via a thin client paradigm \cite{bushberg2011essential}.

\chapter{Basic concepts}

\section{Communication link characteristics}
\begin{itemize}
\item \textbf{Capacity}: This is the total amount of data that a
  data link can transmit per second (bit-rate). It's typically measured in:
  \begin{tabular}{r|l}
    Acronym & Capacity in bits/second\\
    \hline
    1 Kbps & $10^3$\\
    1 Mbps & $10^3$ Kbps\\
    1 Gbps & $10^3$ Mbps\\
    1 Tbps & $10^3$ Gbps
  \end{tabular}
\item \textbf{Reliability}: Wired networks are more reliable
(have less \popup{\gls{BER}}{Ratio of the number of bits received in
error to the total number of bits transmitted over a specific time
interval.}) than wireless networks.
\item \textbf{Security}: Wired networks are more secure than wireless
networks.
\end{itemize}

\section{Networks, links, and channels}
\begin{itemize}
\item A \textbf{channel} is a available transmission capacity in a communication link.
\item A \textbf{link} is a point-to-point (wired) or multipoint
(wireless) physical medium that is able to transmit data. A link can have several channels.
\item A \textbf{network} is a collection of links and devices to
interconnect them (usually routers).
\end{itemize}

\section{Internet and internet}
\begin{itemize}
\item A \textbf{internet} is a network of networks.
\item \textbf{Internet} is the network of networks that we use every day.
\end{itemize}

\section{Web and web}
\begin{itemize}
\item A \textbf{web} is a communication system based on the
server/client model that use the \gls{HTTP} protocol.
\item The \textbf{Web} is the communication system (called \gls{WWW})
that runs at \popup{the Internet level}{You can run a local web at
home, only for your eyes.}. \gls{WWW} and Web are synonyms.
\end{itemize}

\section{Communication protocol}
\begin{itemize}
\item Define the steps and timmings that two (or more) networked
entities must use to communicate.
\item In the Internet, the name of the protocol \popup{suite}{There
are several protocols.} is \acrshort{TCP}/\acrshort{IP}.
\item \gls{HTTP} is the protocol used in the Web (and any web).
\end{itemize}

\section{Packets}
\begin{itemize}
\item At the sender side, the data are splitted into packets.
\item Most of the links are \popup{multiplexed in time}{In an instant
of time, all the capacity of the link is used to transmit a packet.}.
\item Packets have two different parts:
\begin{enumerate}
\item A \textbf{header} that stores information for the \popup{correct
delivery}{Depending on the protocolo, even to solve the transmission
errors.}.
\item A \textbf{payload} that contains the data to transmit.
\end{enumerate}
\end{itemize}
%\chapter{\glsentrylong{TCP}/\glsentrylong{IP} (\glsentryshort{TCP}/\glsentryshort{IP})}

\section{\gls{TCP}}
\begin{itemize}
\item It is the core protocol of the Internet that \textbf{provides
  \popup{reliable}{Ordered and error-free.}, data delivery between
  \popup{networked entities}{Computer applications, ...}}
  \cite{wikipedia_TCP}.
\item \textbf{Bidirectional communication}: Both entities can send and receive
  data during a given interval or time.
\end{itemize}

\section{Steps and timings}
\begin{itemize} 
\item The entities must follow a sequence of steps and waiting times to ensure the communication.
\begin{figure}[H]
  \vspace{-0ex}
  \centering
  \href{https://www.ibm.com/support/pages/flowchart-tcp-connections-and-their-definition}{\includegraphics[height=5.5cm]{Flowchart_TCP}}
  \caption{States (and how are the transition between them) of the
    entities that use the \gls{TCP}.}
  \label{fig:TCP_states}
\end{figure}
\end{itemize}


\section{Transmission control}
\begin{itemize} 
\item The \popup{correct transmission of data is controlled}{Also
    called ``flow control''.} by the receiver.
\begin{figure}[H]
  \vspace{-0ex}
  \centering
  \href{https://ieeexplore.ieee.org/document/8668433}{\includegraphics[height=5.5cm]{TCP_timeline}}
  \caption{A time-line of a \gls{TCP} interaction.}
  \label{fig:TCP_interaction}
\end{figure}
\end{itemize}

\section{\gls{IP}}
\begin{itemize}
\item Currently, coexist two different versions of the IP
  \cite{wikipedia_IP}:
  \begin{enumerate}
  \item \textbf{Version 4}: Defined in the 70's. Uses 32-bits IP addresses.
  \item \textbf{Version 6}: Defined in the 90's. Uses 128-bits IP addresses.
  \end{enumerate}
\item Responsible for the \popup{``best effort''}{IP is classified as
    an unreliable protocol because it does not solve the transmission
    errors.} delivery of data packets.
\item The length of the packets must be $<6$4 KB (including the header).
\end{itemize}

\section{Routers}
\begin{itemize} 
\item Routers use the IP headers to \popup{compute ``on-the-fly'' the
    transmission path of each packet}{Each router is responsible for
    choosing which output link it will use to retransmit each packet
    that arrives.}. This behaviour is known as the \textbf{datagram
    transmission model}.
\begin{figure}[H]
  \vspace{1ex}
  \centering
  \includegraphics[height=0.65\textheight]{router}
  \caption{A router.}
  \label{fig:router}
\end{figure}
\end{itemize}

\section{Public and private IP addresses}
\begin{itemize}
\item Public IP addresses are used for public servers, such as the IP
  address of the Web server of the UAL.
\item Private IP addresses are used the rest of computers connected to
  the Internet. These addresses only are accessible from the local
  (private) network.
\item The router that interconnect a private network with the rest of
  the Internet is a \gls{NAT} device. All the private hosts use the
  public IP address of the \popup{NAT}{... device. It is common to say
    only ``NAT'' instead of ``NAT device'' or ``NAT box''}.
\end{itemize}


%\chapter{The \gls{WWW}}

\section{What is the Web?}
\begin{itemize}
\item The Web \popup{was created}{By Tim Berners-Lee who envisioned a
    system for automated information sharing among scientists
    worldwide. He wrote the first web server, the first web browser,
    and the first web page, and released the technology to the public
    in 1993. This decision to make the technology royalty-free and
    open was crucial to its explosive growth.} in the late 80 at the
  \gls{CERN} to share documentation \cite{wikipedia_WWW}.
\item It is based on the client/server model, where Web clients request
  \popup{Web objects}{A Web object is any resource (usually, a file)
    that is at the server side.} to a Web server.
\item Each object has an \popup{URL}{A unique identifier that
      specifies the Web server and the file within the server.} that
  is used by clients to retrieve the object (from the server).
\item WWW technology is used for cost-effective image distribution to
  clinicians, often via a thin client paradigm
  \cite{bushberg2011essential}.
\end{itemize}

\section{The \glsentrylong{HTTP} (\glsentryshort{HTTP})}
\begin{itemize}
\item The client/server interaction is controlled by the \gls{HTTP} \cite{wikipedia_HTTP}.
\item It is an application-layer protocol that runs over the \gls{TCP}.
\end{itemize}

\section{Web browsers}
\begin{itemize}
\item Most Web clients also implement an objects-browser that \popup{can}{When the browser is not able to process an object (content), it usually offers to save the object.} render
  some standard Web contents, such as:
  \begin{enumerate}
  \item \gls{MP3}, \gls{AAC}, and \href{https://en.wikipedia.org/wiki/Vorbis}{Vorbis} audio.
  \item \gls{PNG}, \gls{GIF}, \gls{PDF}, \gls{SVG}, \gls{JPEG}, and
    \popup{\gls{WebP}}{That also is a container for other image
      codecs.} images.
  \item H.264/\gls{AVC} (\popup{\gls{MP4}}{Usually encapsulated in a MP4
      container.}), and \popup{VP9}{Usually encapsulated in WebM or
      Matroska containers.}, \popup{\gls{AV1}}{Inside Matroska.},
    \popup{\gls{VC-1}}{Encaptulated in MP4.} and
    \popup{\href{https://en.wikipedia.org/wiki/Theora}{Theora}}{Encapsulated
      in Ogg files. videos.}.
  \item \gls{HTML}, \gls{CSS}, and
    \href{https://en.wikipedia.org/wiki/JavaScript}{JavaScript}
    \popup{source codes}{In computer science, a source code is any
      file that contains editable data that a computer can process. In
      this case, all these type of source codes are used for
      controlling the behaviour of the rendering of the Web objects.}.
  \end{enumerate}
\end{itemize}

\section{Secure transmissions: \gls{HTTPS}}
\begin{itemize}
\item \gls{HTTPS} secures data in transit through \gls{TLS}:
  \begin{enumerate}
  \item \textbf{Authentication}: \gls{HTTPS} verifies that your
    browser is communicating with the correct server and not a
    fraudulent one. This is achieved through an \gls{SSL}
    \popup{certificate}{Which is a digital certificate issued by a
      trusted Certificate Authority (CA).}. The browser checks this
    certificate to \popup{confirm the identity of the Web site}{If the
      certificate is not valid, the browser will display a security
      warning.}.
  \item \textbf{Encryption}: Data between the server and the client is
    scrambled. This is performed in two steps:
    \begin{enumerate}
    \item \textbf{Handshake}:
      \begin{enumerate}
      \item The server sends its \popup{public key}{An SSL/TLS
          certificate.} to the client.
      \item The client uses this public key to encrypt a shared
        session key and sends it back to the server, which is the only
        one who can decrypt this message, using its corresponding
        private key.
      \end{enumerate}
    \item \textbf{Encrypted data transfer}: A single, shared secret
      key is used for both encrypting and decrypting data (symmetric
      encryption).
    \item \textbf{Data integrity}: Special codes are interleaved with
      the data to allow the recipient to verify that the message is
      complete and unchanged.
    \end{enumerate}
  \end{enumerate}
\end{itemize}

\section{Some medical-imaging Web sites}
\begin{enumerate}
\item \href{https://openi.nlm.nih.gov/}{Open Access Biomedical Image Search Engine}.
\item \href{https://medpix.nlm.nih.gov/home}{MedPix}.
\item The Visible Human Project:
  \href{https://www.nlm.nih.gov/research/visible/visible_human.html}{(description)},
  \href{https://visiblehumanproject.com/}{(viewer)}.
\item
  \href{https://www.flickr.com/photos/euthman/albums/72057594114099781/}{Specimens
    at flickr}.
\item
  \href{https://lane.stanford.edu/biomed-resources/bassett/index.html}{Bassett
    Collection of Stereoscopic Images of Human Anatomy}.
\item \href{https://dermnetnz.org/images}{DermNet}.
\item \href{http://imagebank.hematology.org/}{American Society of
    Hematology Image Bank}.
\item \href{http://www.med.harvard.edu/AANLIB/home.html}{The Whole
    Brain Atlas}.
\item \href{https://medpics.ucsd.edu/index.cfm}{MedPics - UC San Diego
    School of Medicine}.
\item \href{https://radiopaedia.org/}{Radiopaedia}.
\end{enumerate}

%\chapter{\gls{DICOM} data transmission}

\section{Transmission using the \gls{ULP}}
\begin{itemize}
\item A pair of \gls{DICOM} devices communicate using a client-server
  model based on the \popup{DICOM \gls{ULP}}{An application-layer
    protocol.} which runs on top of standard TCP/IP.
\item Steps:
  \begin{enumerate}
  \item \textbf{Association establishment (handshake)}: the
    application entities agree about the \popup{task to be
      performed}{For example, sending an image.}, and the
    \popup{transfer syntax}{The compression method (e.g., JPEG, RLE)
      and byte order that will be used for the data.}.
  \item \textbf{Data Exchange (conversation)}: The sender sends the
    image using a C-STORE-RQ (Request) message (which the DICOM image,
    including the DICOM header) and, if the message has been
    successfully, the receiver send a C-STORE-RSP (Response).
  \item \textbf{Association Release (Hanging Up)}: After all data has
    been successfully sent and acknowledged, the devices terminate the
    association.
  \end{enumerate}
\end{itemize}



\part{Visualization of medical images}
%\chapter{Perception}

\section{Perception of lightness \cite{wikipedia_lightness} is not linear}
\begin{figure}[H]
  %\vspace{-2ex}
  \centering
  \includegraphics[width=0.9\textwidth]{horizontal}
  \caption[Perception of lightness is not linear (1).]{In each row, the \popup{gradient}{The amount of change.} is 1 (0 in each column).}
  \label{fig:HVS_no_linear}
\end{figure}

\begin{figure}[H]
  %\vspace{-2ex}
  \centering
  \includegraphics[width=0.9\textwidth]{horizontal_vertical}
  \caption[Perception of lightness is not linear (2).]{In each row and column, the \popup{gradient}{The amount of change.} 1.}
  \label{fig:HVS_no_linear}
\end{figure}

\section{Perception of the luminance VS the frequency}
\begin{itemize}
\item The perception of a change of the intensity varies with the spatial frequency.
\begin{figure}[H]
  %\vspace{-2ex}
  \centering
  \includegraphics[width=0.65\textwidth]{CSF3}
  \caption{The \gls{CSF}.}
  \label{fig:CSF}
\end{figure}
\end{itemize}

\section{Perception of the \popup{luma}{Luminance.} VS the neighborhood}
\begin{itemize}
\item The perception of the intensity depends the neighbour pixels.
\begin{figure}[H]
  %\vspace{-2ex}
  \centering
  \includegraphics[width=0.8\textwidth]{linear3}
  \caption[Effect of the neighborhood in the perception of the luminance (1).]{Effect of the neighborhood in the perception of the luminance (in each block, all the pixels have the same intensity).}
  \label{fig:luminance_vs_neighbor_1}
\end{figure}
\end{itemize}

\section*{}
\begin{itemize}
\item The perception of the intensity depends the neighbour pixels.
\begin{figure}[H]
  %\vspace{-2ex}
  \centering
  \includegraphics[width=1.0\textwidth]{contraste_simultaneo}
  \caption[Effect of the neighborhood in the perception of the luminance (2).]{Effect of the neighborhood in the perception of the luminance (all the internal squares have the same intensity).}
  \label{fig:luminance_vs_neighbor_2}
\end{figure}
\end{itemize}

\section{Perception of the luma VS the visualization time}
\begin{itemize}
\item The perception of the intensity depends the visualization time.
\begin{figure}[H]
  %\vspace{-2ex}
  \centering
  \includegraphics[width=0.4\textwidth]{punto_y_difuminado}
  \caption[Effect of visualization time in the perception of the luminance.]{Effect of the visualization time in the perception of the luminance (look to the central point for a while).}
  \label{fig:luminance_vs_visualization_time}
\end{figure}
\end{itemize}

\section{Noise masking}
\begin{itemize}
\item The perception of the structures depends on the type and intensity of the noise.
\begin{figure}[H]
  %\vspace{-2ex}
  \centering
  \includegraphics[width=1.0\textwidth]{noise_masking}
  \caption[Noise masking effect.]{Effect of the noise texture in the perception of the luminance (different effects of Gaussian noise in the wavelet domain).}
  \label{fig:noise_masking}
\end{figure}
\end{itemize}

%\part{Medical Image Visualization and Processing}

The human eye perceives brightness differences non-linearly.

At low luminance, the eye is very sensitive → small changes in brightness are perceptually big.

At high luminance, the eye is less sensitive → bigger brightness jumps are needed to recognize a change in the luminance.

Luminance is measured with a photometer
%\chapter{Visualization of \gls{DICOM} images}

\section{Display (brightness) standardization}
\begin{itemize}
\item To ensure \emph{visual consistency}, that image looks the same
  regardless of which monitor or display device it's viewed on.
\item This is achieved through the \gls{GSDF} \cite{DICOM_GSDF}, that
  expresses the \popup{luminance}{Luminosity, usually measures in
    candelas per square meter
    (\popup{cd}{Candelas.}/\popup{m}{Meter.}$^2$).} that a display
  should produce as a function of the pixels values. Thus, the
  luminance that a pixel with \popup{value}{P-Value in terms of
    the standard notation.} $x$ should have is
  \begin{equation}
    L(j) = \exp\left(\sum_{i=0}^{9}a_i\big(\ln j(x)\big)^i\right),
  \end{equation}
  where
  \begin{center}
  \begin{tabular}{l}
    $a_0 = -1.3011877$, \\
    $a_1 = -2.5840191E^{-2}$, \\
    $a_2 = 8.0242636E^{-2}$, \\
    $a_3 = -1.0320229E^{-1}$, \\
    $a_4 = 1.3646699E^{-1}$, \\
  \end{tabular}
  \begin{tabular}{l}
    $a_5 = 2.8745620E^{-2}$, \\
    $a_6 = -2.5468404E^{-2}$,\\
    $a_7 = -3.1978977E^{-3}$, \\
    $a_8 = 1.2992634E^{-4}$, \\
    $a_9 = 1.3635334E^{-3}$,
  \end{tabular}
  \end{center}
  and where
  \begin{equation}
    j(x) = j_{\text{min}} + \frac{j_{\text{max}} - j_{\text{min}}}{n(x)},
  \end{equation}
  being $j_{\text{min}}$ and $j_{\text{max}}$ are the \popup{minimum
    and maximum \gls{JND} index within the luminance range of the
    display}{That a normal human being can recognize in the display
    which is being standardized.}, and where
  \begin{equation}
    n(x) = \frac{x}{2^N-1},
  \end{equation}
  where $N$ is thee number of bits/pixel.
\end{itemize}

\begin{figure}[H]
  \vspace{-0ex}
  \centering
  \includegraphics[width=0.7\textwidth]{GSDF}
  \caption{\gls{GSDF} used in \gls{DICOM}.}
  \label{fig:GSDF}
\end{figure}

\begin{itemize}
\item This way, equal steps in pixel values correspond to equal
  perceptual differences (\glspl{JND}).
\item In other words, a \popup{step}{An increment in one in the
  integer value of ...} in \gls{JND} corresponds as the smallest
  brightness change detectable by the average human eye.
\end{itemize}


\part{Processing of medical images}
%\chapter{Image contrast enhancement}

\section{Objective}
\begin{itemize}
\item The primary objective of medical image enhancement is to improve
  the visual quality of an image, enabling physicians to more
  accurately observe and interpret critical details.
\item Enhancements such as brightness adjustment, contrast
  optimization, edge sharpening, and other visual refinements can
  significantly aid in this process.
\end{itemize}

\section{Pixel normalization}
\begin{itemize}
\item Also called contrast stretching \cite{gonzalez2009digital}, is a
  pixel-wise linear transformation that enhance the contrast by
  spreading out the intensity values over the full available range of
  pixel values (usually [0, 255]).
\item Let $x_i$ the input pixel intensity, the new value is
  \begin{equation}
    y_i = \frac{(x_i - x_{\min})}{(x_{\max} - x_{\min})} \times 255
  \end{equation}
  where $r_{\text{min}}$ and $r_{\text{max}}$ are the minimum and
  maximum input intensity values.
\end{itemize}

\begin{figure}[H]
  \vspace{-0ex}
  \centering
  \href{https://github.com/vicente-gonzalez-ruiz/medical_imaging/blob/main/notebooks/pixel_normalization.ipynb}{\includegraphics[width=10cm]{pixel_normalization}}
  \caption{Effects of pixel normalization (contrast stretching).}
  \label{fig:pixel_normalization}
\end{figure}

\section{Gamma correction}

\begin{itemize}
\item Corrects for the fact that displays (monitors, projectors, etc.) and
human vision do not respond linearly to intensity values.

\item It is defined as
\begin{equation}
s = c \cdot r^{\gamma},
\end{equation}
where:
\begin{enumerate}
\item $r$ = input pixel intensity (normalized to $[0,1]$)
\item $s$ = output pixel intensity (normalized to $[0,1]$)
\item $c$ = normalization constant (often $1$)
\item $\gamma$ = gamma value (controls brightness/contrast)
\end{enumerate}

\begin{figure}[H]
  \vspace{-0ex}
  \centering
  \href{https://github.com/vicente-gonzalez-ruiz/medical_imaging/blob/main/notebooks/gamma_correction.ipynb}{\includegraphics[width=10cm]{gamma_correction}}
  \caption{Effects of gamma correction.}
  \label{fig:gamma correction}
\end{figure}

\end{itemize}


\section{Histogram equalization}

\begin{itemize}
\item Histogram equalization \cite{gonzalez2009digital}
  redistributes the intensity values of an image so that the histogram
  (distribution of pixel intensities) becomes more uniform.
\item The goal is to increase the overall contrast, especially in
  areas that are too dark or too bright.

% Histogram Equalization Formulation
Let an image have $L$ intensity levels $0, 1, 2, \dots, L-1$.

\begin{enumerate}
  \item The probability of intensity $r_k$ is
\[
p(r_k) = \frac{n_k}{N}
\]
where 
\begin{enumerate}
\item $n_k$ = number of pixels with intensity $r_k$.
\item $N$ = total number of pixels in the image.
\end{enumerate}

\item Cumulative distribution function (CDF), defined by
\[
c(r_k) = \sum_{j=0}^{k} p(r_j).
\]

\item Mapping to new intensity, what is
\[
s_k = (L-1) \cdot c(r_k),
\]
where 
\begin{enumerate}
\item $r_k$ = original intensity.
\item $s_k$ = new intensity after histogram equalization.
\end{enumerate}

\begin{figure}[H]
  \vspace{-0ex}
  \centering
  \href{https://github.com/vicente-gonzalez-ruiz/medical_imaging/blob/main/notebooks/equalized_histogram.ipynb}{\includegraphics[width=10cm]{equalized_histogram}}
  \caption{Effects of histogram equalization.}
  \label{fig:histogram_equalization}
\end{figure}

\end{enumerate}
\end{itemize}

%section{Sharpening}

\section{Homomorphic filtering \cite{gonzalez2009digital}}

% Homomorphic Filtering Formulation
\begin{itemize}
\item An image can be modeled as the product of illumination and
  reflectance \cite{wikipedia_luminance} as
\[
f(x, y) = i(x, y) \cdot r(x, y).
\]
where
\begin{enumerate}
\item $f(x, y)$ = observed image intensity.
\item $i(x, y)$ = illumination component (slow-varying, low-frequency).
\item $r(x, y)$ = reflectance component (details, high-frequency),
\end{enumerate}

\textbf{Step 1: Logarithmic transformation}
\[
\ln f(x, y) = \ln i(x, y) + \ln r(x, y)
\]

\textbf{Step 2: Fourier transform}
\[
F(u, v) = \mathcal{F}\{\ln f(x, y)\} = I(u, v) + R(u, v)
\]

\textbf{Step 3: Apply a high-pass filter}
\[
S(u, v) = H(u, v) \cdot F(u, v)
\]
where $H(u,v)$ is the filter transfer function.

\textbf{Step 4: Inverse Fourier transform}
\[
s(x, y) = \mathcal{F}^{-1}\{S(u, v)\}
\]

\textbf{Step 5: Exponential transform to restore the image}
\[
f_{\text{enhanced}}(x, y) = \exp(s(x, y))
\]

\begin{figure}[H]
  \vspace{-0ex}
  \centering
  \href{https://github.com/vicente-gonzalez-ruiz/medical_imaging/blob/main/notebooks/homomorphic_filtering.ipynb}{\includegraphics[width=10cm]{homomorphic_filtering}}
  \caption{Effects of homomorfic enhancement.}
  \label{fig:homomorphic_filtering}
\end{figure}

\end{itemize}
 % Dealing with underexposure and overexposure.
%\chapter{Image Denoising}

\section{Thermal noise}
\begin{itemize}
\item Significative in \gls{MRI}.
\item Originated by the thermal motion of the atoms and therefore, of
  the electrons.
\item Described as a grainy, random texture.
\item Modeled as additive Gaussian noise or, in the case of \gls{MRI}
  as additive Rician noise, because the noise is captured in the
  frequency domain.
\end{itemize}

\section{Quantum mottle (quantum noise)}
\begin{itemize}
\item Appears in X-ray and \gls{CT} images.
\item Consequence of the \popup{small number
    of photons}{Remember that the radiation must be minimized and the
    number of X-ray photons is proportional to the energy of the
    radiation.} that reach the detector.
\item Looks like grainy noise. 
\item Mathematically modeled as a \popup{(multiplicative)}{By
    definition, Poisson noise is multiplicative.} Poisson
  distribution.
\end{itemize}

\section{Speckle (interference) noise}
\begin{itemize}
\item Significative in low-SNR areas of \gls{MRI} images and in ultrasound images.
\item Generated by the constructive and destructive interference of
  \popup{coherent waves}{Waves are in phase.}, such as laser light or
  radar waves, interacting with a target.
\item Described as granular, textured pattern.
\item Usually modeled as multiplicative Gamma distribution (or more
  specifically, a Rayleigh distribution for the signal amplitude).
\end{itemize}

\section{Physiological noise}
\begin{itemize}
\item Significative in \gls{MRI}, and it is independent of $B_0$ (the
  strength of the magnetic field).
\item Refers to undesired signal variations caused by the patient's
  own bodily functions, primarily cardiac (heartbeat) and respiratory
  (breathing) cycles. These processes induce changes in cerebral blood
  flow, blood volume, and cerebrospinal fluid flow, generating
  magnetic field fluctuations.
\item Non uniform (depends on the scanned area) and difficult to model.
\end{itemize}

\section{Denoising}
\begin{itemize}
\item Denoising (the removal of the noise) is carried out in
  medical imaging to improve the \gls{SNR}, with the ultimate
  objective of increase the accuracy of the diagnoses.
\item Unfortunately, it is difficult to remove only noise (some part
  of the signal, typically the high frequency components of the signal
  are also filtered-out.
\end{itemize}

\section{Gaussian filtering}
\begin{itemize}
\item Spatial (2D) filter \popup{using 1D Gaussian}{The filter is
    separable, which means that the image can be filtered by rows and
    columns, using always 1D kernels.} \popup{kernels}{Filter is
    another name for the filter structure.}.
\end{itemize}
\vspace{-4ex}
\begin{center}
  \includegraphics[width=6cm]{Gaussian_kernels}
\end{center}
\vspace{-4ex}
\begin{itemize}
\item Isotropic (it \popup{blurs pixel values equally in all directions}{Which
    softens noise but also destroys important edges.}).
\end{itemize}
\begin{center}
    \href{https://www.cloudfactory.com/blog/gaussian-noise-medical-ai}{\includegraphics[width=\textwidth]{GF_example}}
\end{center}

\section{Anisotropic diffusion filtering}
\begin{itemize}
\item Anisotropic (only blurs in the direction of the minimum
  gradient, i.e., the edged are preserved).
\end{itemize}
\begin{center}
  \href{https://dsp.stackexchange.com/questions/14606/anisotropic-diffusion}{\includegraphics[width=8cm]{anisotropic_diffusion}}\\
  (The kernel is longer in the direction of the edge.)
\end{center}
\begin{center}
    \href{https://es.mathworks.com/help/images/ref/imdiffusefilt.html}{\includegraphics[width=\textwidth]{AD_example}}
\end{center}

\section{Wiener denoising}
% Include also filtering in the wavelet domain because can be useful for removing specke noise.
\vbox{
\begin{itemize}
\item Wiener developed an adaptive filter based on a predictor capable
  of restore the \popup{image}{A signal in general.} in several
  aspects, for example,
  \href{https://docs.opencv.org/3.4/d1/dfd/tutorial_motion_deblur_filter.html}{to
    correct the blur generated by motion}.
\begin{center}
  \href{https://docs.opencv.org/3.4/white_car.jpg}{\includegraphics[width=12cm]{wiener_deconvolution}}\\
  (Motion de-blur using Wiener deconvolution.)
\end{center}
\end{itemize}
}
\vbox{
\begin{itemize}
\item Using a Wiener filter we can also \popup{remove}{The correct
    word here (and in all the denoising techniques) should be
    ``minimize''. Only if the noise signal were known, the original
    (clean) signal could be completely restored.}  \popup{additive
    noise}{A random signal that has been added to the clean signal,
    and that is independent of the clean signal.} of a image by
  estimating the amplitude of the noise. The idea is to use a
  \popup{bluring filter}{A low-pass filter, such as a Gaussian kernel}
  \popup{adapted to the energy of the noise}{The higher the amplitude
    of the noise, the longer the kernel, i.e., the higher the blur
    effect, and therefore, the higher the noise removal.} \popup{in
    each area}{For this reason, one of the parameters of the Wiener
    filter for denoising is the size of a square window.} of the
  image.
\end{itemize}
\vspace{-4ex}
\begin{center}
  \href{https://www.techscience.com/csse/v45n2/50440/html}{\includegraphics[width=10cm]{wiener_albert}}\\
  (Removal of Gaussian noise using Wiener denoising.)
\end{center}
}

\section{\gls{NLM}}

\section{\gls{AI}-based denoising}
How They Work

Unlike traditional methods that rely on fixed mathematical algorithms (like Gaussian filters or wavelet transforms), deep learning models learn how to denoise.

The typical process involves training a neural network on a massive dataset of paired data:

    Clean Data: The original, noise-free image or signal.

    Noisy Data: The same data with noise artificially added.

    The network's goal is to learn the complex mapping required to transform any noisy input back into its clean, original state. By doing this, it learns the underlying structure of the signal and can distinguish it from the characteristics of the noise.

Key Architectures Used

Several types of neural networks are particularly effective for this task:

    Denoising Autoencoders: A classic approach where a network learns to compress a noisy image into a compact representation (encoding) and then reconstruct a clean version from it (decoding).

Convolutional Neural Networks (CNNs): The most popular and powerful method for image denoising. Architectures like DnCNN (Denoising CNN) and U-Net are specifically designed to process images, effectively learning to identify and remove noise while preserving important details like edges and textures.

Generative Adversarial Networks (GANs): These models use a "generator" network to create clean images and a "discriminator" network to distinguish them from real clean images. This adversarial process pushes the generator to produce extremely realistic, high-quality denoised results.

AI vs. Traditional Denoising

Feature	Traditional Denoising	AI-Based Denoising
Principle	Relies on fixed mathematical models (e.g., averaging, transforms).	Learns from data to distinguish signal from noise.
Adaptability	Not very adaptable; optimized for specific noise types.	Highly adaptable; can handle complex and mixed noise.
Data	Requires no training data.	Requires a large dataset of noisy/clean pairs for training.
Performance	Can cause blurring and loss of fine details.	Excels at preserving details and texture, often achieving superior results.

Singh, P. (2025). Understanding Medical Image Denoising, Enhancement, and Reconstruction. Biomedical Informatics and Smart Healthcare, 1(1), 35–39. https://doi.org/10.62762/BISH.2025.966762

Gaussian,  or non-local means
%\chapter{Image super-resolution}

\section{Another ill-posed problem}
\begin{itemize}
\item Image super-resolution is basically a hallucination exercise of
  a \gls{ANN}.
\item There are infinite forms of imagining
  \href{https://en.wikipedia.org/wiki/Super-resolution_imaging}{what
    could be in an image when we zoom-in a region of it}.
\end{itemize}

\section{Objective}
\begin{itemize}
\item Super-resolution is an image processing technique that
  increases the perceived quality by means of ``fabricating''
  (possibly \popup{unreal}{Invented.}) new visual information.
\item Even knowing this fact, it can help in medical imaging
  diagnosis.
\end{itemize}

\section{Nearest-neighbor interpolation \cite{gonzalez2009digital}}
\begin{itemize}
\item For a discrete image $f: \mathbb{Z}^2 \to \mathbb{R}$, the interpolation at a point 
$(x,y) \in \mathbb{R}^2$ is given by
\begin{equation}
\hat{f}(x,y) = f\!\left( \operatorname{round}(x), \operatorname{round}(y) \right),
\end{equation}
where $\operatorname{round}(x)$ returns the nearest integer value of $x$.
\end{itemize}

\begin{figure}[H]
  \vspace{-2ex}
  \centering
  \href{https://www.mrecacademics.com/DepartmentStudyMaterials/20201220-Digital%20Image%20Processing%20Notes.pdf}{\includegraphics[width=5cm]{discrete_image}}
  \caption{Coordinates in digital images.}
  \label{fig:digital_image}
\end{figure}

\begin{figure}[H]
  \vspace{0ex}
  \centering
  \href{https://github.com/vicente-gonzalez-ruiz/medical_imaging/blob/main/notebooks/nearest_integer_interpolation.ipynb}{\includegraphics[width=12cm]{lena_nearest_integer}}
  \caption{Blocking generated by nearest integer interpolation.}
  \label{fig:pixel_blocking}
\end{figure}

\section{Bilinear interpolation \cite{gonzalez2009digital}}

\begin{itemize}
\item For an image $f: \mathbb{Z}^2 \to \mathbb{R}$, the bilinear interpolation at 
$(x,y) \in \mathbb{R}^2$ is
\begin{equation}
\hat{f}(x,y) = (1-\alpha)(1-\beta)\, f(i,j) 
+ \alpha(1-\beta)\, f(i+1,j) 
+ (1-\alpha)\beta\, f(i,j+1) 
+ \alpha\beta\, f(i+1,j+1),
\end{equation}
where
\[
i = \lfloor x \rfloor, \quad j = \lfloor y \rfloor, \quad
\alpha = x - \lfloor x \rfloor, \quad \beta = y - \lfloor y \rfloor.
\]

\newpage
\item Smooth transitions, commonly used in \gls{CT}/\gls{MRI} resampling.
\end{itemize}

\begin{figure}[H]
  \vspace{-1ex}
  \centering
  \href{https://github.com/vicente-gonzalez-ruiz/medical_imaging/blob/main/notebooks/bilinear_interpolation.ipynb}{\includegraphics[width=11cm]{lena_bilinear}}
  \caption{Bilinear interpolation example.}
  \label{fig:bilinear_interpolation}
\end{figure}

\section{\gls{ESRGAN} \cite{wang2018esrgan}}
\begin{itemize}
\item ESRGAN is a \popup{deep}{349 convolutional layers.} \gls{CNN}
  $G_\theta$ trained with adversarial and perceptual losses to
  \popup{hallucinate}{Imagine.} realistic high-frequency image
  details.
\end{itemize}

\begin{figure}[H]
  \vspace{0ex}
  \centering
  \href{https://arxiv.org/abs/2207.08036}{\includegraphics[width=11cm]{Generator-Architecture-of-Real-ESRGAN}}
  \caption{$G_\theta$, the generator used in \gls{ESRGAN}.}
  \label{fig:ESRGAN_generator}
\end{figure}

\begin{itemize}
\item $G_\theta$ performs
  \begin{equation}
    \hat{I}_{\text{HR}} = G_\theta(I_{\text{LR}}),
  \end{equation}
  where $I_{\text{LR}}$ is the low-resolution input image and
  $\hat{I}_{\text{HR}}$ is the \popup{high-resolution}{4x in the
    figure.} output image.
\end{itemize}

\section*{Adversarial training}

\begin{figure}[H]
  \vspace{0ex}
  \centering
  \href{https://semiengineering.com/knowledge_centers/artificial-intelligence/neural-networks/generative-adversarial-network-gan/}{\includegraphics[width=8cm]{GAN}}
  \caption{The \gls{ESRGAN} network.}
  \label{fig:ESRGAN}
\end{figure}

\begin{itemize}
\item \popup{Unsupervised learning}{This is true for all GANs.} that
  utilize two neural networks:
  \begin{enumerate}
  \item \textbf{Generator} ($G_\theta$): Using random data or a seed image,
    generate \popup{fake}{Unreal.} images.
  \item \textbf{Discriminator} ($D_\phi$): Try to determine if the generated
    image is real or not.
  \end{enumerate}
\item In each iteration, and independently of who wins, both, the
  discriminator and the generator are retrained (updated) considering
  the result.
\item The training ends when generator has \popup{no
    idea}{There is a 50-50 chance of success.}  about the veracity of the image
  he receives.
\item \glspl{GAN} can be trained for generate any kind of content,
  such as for example,
  \href{https://thispersondoesnotexist.com/}{faces}).
\end{itemize}

\section*{Training of the discriminator}
\begin{itemize}
\item The discriminator’s objective function
\begin{equation}
  \mathcal{L}_D(\phi) = 
  \mathbb{E}\big[ \log D_\phi(I_{\text{HR}}) \big] 
  + \mathbb{E}\big[ \log (1 - D_\phi(G_\theta(I_{\text{LR}}))) \big],
\end{equation}
is maximized when it outputs a probability close to 1 for real HR
images and close to 0 for fake HR images. We have that:
\begin{enumerate}
\item $\phi$ represents the \popup{parameters}{Weights of the ANN.} of
  the discriminator.
\item $\mathbb{E}[x]$ is the
  \popup{expectation}{Basically the mean.} of the sequence of values
  $x$ where $x_i$ is a HR image.
\item $G_\theta(I_{\text{LR}})$ is the fake image.
\item $ D_\phi(G_\theta(I_{\text{LR}}))$ is the discriminator decision for the fake image.
\item $\log(1 - D_\phi(G_\theta(I_{\text{LR}})))$ is very negative if
  the decision is one (the generated image is fake, and I guess that
  is real), and is zero if the decision is 0 (the generated image is
  fake, and I guess that is fake). Therefore, each time the
  discriminator fails, the objective is \popup{reduced}{Remember that
    we want to maximize this function.}.
\item $\log D_\phi(I_{\text{HR}})$ is \popup{zero}{Good :-)} if the
  discriminator's guess is correct (I recognized that $I_{\text{HR}}$
  is a real image), and very \popup{negative}{Bad :-(} if the guess is
  incorrect.
\end{enumerate}
\end{itemize}

\section*{Training of the generator}
\begin{itemize}
\item The generator try to minimize
\begin{equation}
\mathcal{L}_G(\theta) = 
\lambda_{\text{adv}} \, \mathcal{L}_{\text{adv}}(\theta) \;+\;
\lambda_{\text{con}} \, \mathcal{L}_{\text{con}}(\theta) \;+\;
\lambda_{\text{pix}} \, \mathcal{L}_{\text{pix}}(\theta),
\end{equation}
where
\begin{equation}
  \mathcal{L}_{\text{adv}}(\theta) = - \mathbb{E}\big[ \log D_\phi(G_\theta(I_{\text{LR}})) \big]
\end{equation}
is the adversarial loss,
\begin{equation}
\mathcal{L}_{\text{con}}(\theta) = \mathbb{E}\left[ \big\| \Phi(G_\theta(I_{\text{LR}})) - \Phi(I_{\text{HR}}) \big\|_2^2 \right]
\end{equation}
is the content (perceptual) loss, and
\begin{equation}
\mathcal{L}_{\text{pix}}(\theta) = \mathbb{E}\left[ \| G_\theta(I_{\text{LR}}) - I_{\text{HR}} \|_1 \right]
\end{equation}
is the pixel loss, where $\lambda_{\text{adv}}, \lambda_{\text{con}},$
and $\lambda_{\text{pix}}$ are three hyperparameters that control the
weight of each loss, $\theta$ represents the parameter of the
generator, $\Phi(\cdot)$ a feature extraction function (usually the
output of an intermediate layer), $\big\|\cdot\big\|_2^2$ the squared
Euclidean norm (close to the \gls{MSE}). and $\|\cdot\|_1$ is the \popup{L1 norm}{called the
  Manhattan norm or taxicab norm}.

\end{itemize}

\begin{figure}[H]
  \vspace{0ex}
  \centering
  \href{https://github.com/vicente-gonzalez-ruiz/medical_imaging/blob/main/notebooks/ESRGAN.ipynb}{\includegraphics[width=12cm]{lena_ESRGAN}}
  \caption{A super-resolution example with \gls{ESRGAN}.}
  \label{fig:ESRGAN_example}
\end{figure}


%\chapter{Segmentation}

% 3D surface reconstruction (ultrasound)

Thresholding: Based on intensity (simple, but ultrasound intensities can vary a lot).

Region Growing: Start from seed points inside the face region and expand.

Machine Learning / Deep Learning: Modern approaches use CNNs or U-Nets to segment anatomical structures directly in 3D.

Identifying specific regions such as tumors, organs, or vessels.

Can be done via thresholding, region growing, watershed, or deep learning–based methods (e.g., U-Net).

watershed segmentation

\section{U-Net segmentation}
%\chapter{Registration}

\section{Why?}
\begin{itemize}
\item Image registration \cite{Oliveira25012014} is a digital image
  processing technique that helps us align different images (for
  example, \gls{CT} and \gls{MRI} of the same scene.
\item For example, the registered images can be averaged to increase
  the \gls{SNR}.
\end{itemize}

\begin{figure}[H]
  \vspace{-0ex}
  \centering
  \begin{tabular}{ccc}
    \href{https://d2rfm59k9u0hrr.cloudfront.net/medpix/img/full/synpic50411.jpg}{\includegraphics[width=0.25\textwidth]{a1}} & \href{https://d2rfm59k9u0hrr.cloudfront.net/medpix/img/full/synpic50412.jpg}{\includegraphics[width=0.25\textwidth]{a2}} & \href{sec:ImageJ_registration}{\includegraphics[width=0.25\textwidth]{a3}} \\
    Reference & To warp & Projection
  \end{tabular}
  \caption[Image registration example.]{Image registration
    example. Images source:
    \url{https://medpix.nlm.nih.gov/case?id=414ff6ec-e857-40da-ac84-d3dba505a08a}.}
  \label{fig:image_registration}
\end{figure}
%https://medpix.nlm.nih.gov/case?id=d9f68ce4-ee14-49ea-a50a-19e70dabd239
%https://medpix.nlm.nih.gov/case?id=414ff6ec-e857-40da-ac84-d3dba505a08a

\section{Registration with ImageJ}
\label{sec:ImageJ_registration}
\begin{itemize}
  \item ImageJ
    \cite{abramoff2004image}\footnote{\url{https://imagej.net/ij}} is
    an open-source, Java-based software for processing and analyzing
    2D and 3D scientific images.
\begin{figure}[H]
  \vspace{-0ex}
  \centering
  \includegraphics[width=0.75\textwidth]{ImageJ}
  \caption{ImageJ icon and main window.}
  \label{fig:ImageJ}
\end{figure}
  \item To register 2 images there are
    \href{https://imagej.net/imaging/registration}{several}
    \popup{plugins}{A source code Java extension which incorporate
      extra functionality to the basic program}.
\end{itemize}

\begin{itemize}
\item To install a plugin\footnote{\url{https://imagej.net/plugins}} do:
  \begin{enumerate}
  \item Go to the plugins Web page: \url{https://imagej.net/plugins}.
  \item Click in \href{https://imagej.net/list-of-extensions}{List of Extensions}: \url{https://imagej.net/list-of-extensions}.
  \item Search for TurboReg: \url{https://imagej.net/plugins/turboreg}.
  \item Follow the link to the BIG website: \url{https://bigwww.epfl.ch/thevenaz/turboreg/}.
  \item Download the .zip file: \url{http://bigwww.epfl.ch/thevenaz/turboreg/turboreg.zip}.
  \item Unzip it.
  \item Open ImageJ.
  \item Go to \texttt{Plugins} $\rightarrow$ \texttt{Install ...}.
  \item Select the file \texttt{TurboReg\_.jar}.
  \end{enumerate}
  \newpage
\item To use the TurboReg plugin:
  \begin{enumerate}
  \item Load the reference image. For example, \href{https://d2rfm59k9u0hrr.cloudfront.net/medpix/img/full/synpic50411.jpg}{this image}.
  \item Load the image to register. For example, \href{https://d2rfm59k9u0hrr.cloudfront.net/medpix/img/full/synpic50412.jpg}{this image}.
  \item Go to \texttt{Pugins} $\rightarrow$ \texttt{TurboReg}.
  \item Configure and run it.
  \item Save the image, if you are happy with it: \texttt{File}
    $\rightarrow$ \texttt{Save As} $\rightarrow$ (chose the output file
    format).
  \end{enumerate}
\end{itemize}

\begin{comment}
# https://www.geeksforgeeks.org/python/image-registration-using-opencv-python/

import cv2
import numpy as np

# Open the image files.
img1_color = cv2.imread("align.jpg")  # Image to be aligned.
img2_color = cv2.imread("ref.jpg")    # Reference image.

# Convert to grayscale.
img1 = cv2.cvtColor(img1_color, cv2.COLOR_BGR2GRAY)
img2 = cv2.cvtColor(img2_color, cv2.COLOR_BGR2GRAY)
height, width = img2.shape

# Create ORB detector with 5000 features.
orb_detector = cv2.ORB_create(5000)

# Find keypoints and descriptors.
# The first arg is the image, second arg is the mask
#  (which is not required in this case).
kp1, d1 = orb_detector.detectAndCompute(img1, None)
kp2, d2 = orb_detector.detectAndCompute(img2, None)

# Match features between the two images.
# We create a Brute Force matcher with 
# Hamming distance as measurement mode.
matcher = cv2.BFMatcher(cv2.NORM_HAMMING, crossCheck = True)

# Match the two sets of descriptors.
matches = matcher.match(d1, d2)

# Sort matches on the basis of their Hamming distance.
matches.sort(key = lambda x: x.distance)

# Take the top 90 % matches forward.
matches = matches[:int(len(matches)*0.9)]
no_of_matches = len(matches)

# Define empty matrices of shape no_of_matches * 2.
p1 = np.zeros((no_of_matches, 2))
p2 = np.zeros((no_of_matches, 2))

for i in range(len(matches)):
  p1[i, :] = kp1[matches[i].queryIdx].pt
  p2[i, :] = kp2[matches[i].trainIdx].pt

# Find the homography matrix.
homography, mask = cv2.findHomography(p1, p2, cv2.RANSAC)

# Use this matrix to transform the
# colored image wrt the reference image.
transformed_img = cv2.warpPerspective(img1_color,
                    homography, (width, height))

# Save the output.
cv2.imwrite('output.jpg', transformed_img)

\end{comment}



%\input{sharpening} % Enhancement: shapering, histogram equalization, normalization
%\part{Medical Image Processing}

Deleteme

\printglossary[type=\acronymtype]


%\section*{Resources}

\bibliographystyle{plain}
\bibliography{tomography,MRI,denoising,image_formats,X-rays,ultrasound,CT,radiology,storage,JPEG,JPEG2000,color_spaces,DICOM,vruiz,local}

\end{document} 