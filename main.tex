%\documentclass{article}
%\documentclass[draft]{report}%{article}
\documentclass{report}%{article}
\usepackage{graphicx}
%\usepackage[draft]{graphicx}
\graphicspath{{figs}{notebooks}{.}}
\usepackage{hyperref}
\usepackage{amsmath}
\usepackage{amsfonts}
\usepackage{amssymb}
\usepackage[margin=1in]{geometry}
\usepackage{comment}
\usepackage{color}
\usepackage[acronym]{glossaries}
%\usepackage{alphalph} % For extended alphabetical numbering
\usepackage{appendix}

% \usepackage[%
%filename ,%
%content={no image available}
%]{draftfigure}

\makeglossaries
% \newacronym{label}{acronym}{definition}

\newacronym{LM}{LM}{Light Microscopy}
\newacronym{EM}{EM}{Electron Microscopy}
\newacronym{TEM}{TEM}{Transmission Electron Microscopy}
\newacronym{SEM}{SEM}{Scanning Electron Microscopy}
\newacronym{SPM}{SPM}{Scanning Probe Microscopy}
\newacronym{AFM}{AFM}{Atomic Force Microscopy}
\newacronym{STM}{STM}{Scanning Tunneling Microscope}

\newacronym{ET}{ET}{Electron Tomography}
\newacronym{cryo-ET}{Cryo-ET}{Cryo-Electron Tomography}
\newacronym{OPT}{OPT}{Optical Projection Tomography}
\newacronym{SXT}{STX}{Soft X-ray Tomography}
\newacronym{PAT}{PAT}{PhotoAcoustic Tomography}

\newacronym{GT}{GT}{Ground Truth}

\newacronym{DRV}{DRV}{Discrete Random Variable}

\newacronym{CC}{CC}{Cross-Correlation}
\newacronym{NCC}{NCC}{Normalized Cross-Correlation}
\newacronym{PCC}{PCC}{Pearson Correlation Coefficient}
\newacronym{AC}{AC}{Auto-Correlation}
\newacronym{NAC}{NAC}{Normalized Auto-Correlation}
\newacronym{SNR}{SNR}{Signal-to-Noise Ratio}
\newacronym{PSNR}{PSNR}{Peak \acrshort{SNR}}
\newacronym{SSNR}{SSNR}{Spectral Signal-to-Noise Ratio}
\newacronym{SRV}{SRV}{Stationary Random Variable}
\newacronym{FTDF}{FTDF}{Fourier Transform of a Discrete Function}
\newacronym{DTFT}{DTFT}{Discrete Time Fourier Transform}
\newacronym{IDTFT}{IDTFT}{Inverse Discrete Time Fourier Transform}
\newacronym{DFT}{DFT}{Discrete Fourier Transform}
\newacronym{IDFT}{IDFT}{Inverse Discrete Fourier Transform}
\newacronym{FFT}{FFT}{Fast Fourier Transform}
\newacronym{IFFT}{IFFT}{Inverse Fast Fourier Transform}
\newacronym{PS}{PS}{Power Spectrum}
\newacronym{PSD}{PSD}{Power Spectral Density}
\newacronym{CPSD}{CPSD}{Cross-Power Spectral Density}
\newacronym{MAD}{MAD}{Mean Absolute Deviation}

\newacronym{AWG}{AWG}{Additive White Gaussian}
\newacronym{PDF}{PDF}{Probability Density Function}
\newacronym{PMF}{PMF}{Probability Mass Function}
\newacronym{PMD}{PMD}{Probability Mass Distribution}
\newacronym{MPG}{MPG}{Mixed Poisson-Gaussian}

\newacronym{FSC}{FSC}{\href{https://en.wikipedia.org/wiki/Fourier_shell_correlation}
{Fourier Shell Correlation}}
\newacronym{SFRC}{SFRC}{Self Fourier Ring Correlation}
\newacronym{FRC}{FRC}{Fourier Ring Correlation}
\newacronym{FC}{FC}{Fourier Correlation}
\newacronym{SFC}{SFC}{Self Fourier Correlation}

\newacronym{DOF}{DOF}{Dense Optical Flow}
\newacronym{EOS}{EOS}{Even-Odd Splitting}
\newacronym{CBS}{CBS}{ChessBoard Splitting}
\newacronym{ICBS}{ICBS}{Interpolated \acrshort{CBS}}
\newacronym{SCBS}{SCBS}{Subsampled \acrshort{CBS}}
\newacronym{SPRS}{SPRS}{Structure Preserving Random Shuffling}
\newacronym{TRPR}{TRPR}{TriS-D Random Pixel Resampling}

\newacronym{MSE}{MSE}{Mean Square Error}
\newacronym{RMSE}{RMSE}{Root \acrshort{MSE}}
\newacronym{SSIM}{SSIM}{Structural Similarity Index Measure }
\newacronym{MS-SSIM}{MS-SSIM}{Multi-Scale \acrshort{SSIM}}
\newacronym{LPIPS}{LPIPS}{Learned Perceptual Image Patch Similarity}
\newacronym{NIQE}{NIQE}{Natural Image Quality Evaluator}
\newacronym{CS}{CS}{Cosine Similarity}

\newacronym{CLT}{CLT}{Central Limit Theorem}
\newacronym{GAT}{GAT}{Generalized Anscombe Transform}
\newacronym{VST}{VST}{Variance-Stabilizing Transform}
\newacronym{MNI}{MNI}{Mean of Noisy Instances}

\newacronym{GF}{GF}{Gaussian Filtering}
\newacronym{WF}{WF}{Wiener Filtering}
\newacronym{GD}{GD}{Gaussian Denoising}
\newacronym{TF}{TF}{Transfer Function}
\newacronym{LTI}{LTI}{Linear Time-Invariant}
\newacronym{NoO}{NoO}{Number of Operations}
\newacronym{IIR}{IIR}{Infinite Impulse Response}
\newacronym{FIR}{IIR}{Finite Impulse Response}

\newacronym{CNN}{CNN}{Convolutional Neural Network}

\newacronym{BM3D}{BM3D}{Block-Matching and 3D filtering}
\newacronym{DnCNN}{DnCNN}{Denoising \acrshort{CNN}}

\title{Denoising in Microscopy Imaging}

\author{Vicente González-Ruiz and José Jesús Fernández Rodríguez}

\begin{document}
\maketitle
\tableofcontents

\section*{Definitions and notation}
%{{{

\begin{tabular}{ll}
  $x$ & A scalar value (e.g., a value of a pixel of a grayscale image) \\
  $s(t)$ & A (continuous) signal as a function of time \\
  $s[n]$ & A discrete signal (only) defined at instants of time $tn, n\in\mathcal{Z}, t>0$ \\
  $\mathbf{s}$ & A digital (discrete and finite) signal (e.g., an image) \\
  $\mathbf{s}_{i}$ & The $i$-th element of $\mathbf{s}=\{\mathbf{s}_{i}\}_{i=0}^{N-1}=\{\mathbf{s}_{i}\}$ \\
  %$A[b]$ & The $b$-th element of the sampled version of $A(b)$ \\
  $\{i\}$ & The set $i$ \\
  $\mathbf{s}_{\{i\}}$ & The elements of $\mathbf{s}$ with indices $\{i\}$ \\
  $\mathbf{s}_{[i]}$ & A window of samples of $\mathbf{s}$ centered at the $i$-th sample \\
  $\mathbf{s}_{\href{https://numpy.org/doc/stable/user/basics.indexing.html#slicing-and-striding}{y,:}}$ & The $y$-th row of the image $\mathbf{s}$ \\
  $\mathbf{s}_{:,x}$ & The $x$-th column of the image $\mathbf{s}$ \\
  $\mathbf{s}_{y,x}$ & The pixel $(y,x)$ of the image $\mathbf{s}$ \\
  %$\mathbf{x}^{(i)}$ & The $i$-th real-noisy instance of the signal $\mathbf{x}$ \\
  %$\mathbf{s}^{()}$ & An instance of $\mathbf{s}$, possibly noisy \\
  $\mathbf{s}^{(i)}$ & The $i$-th instance of the signal $\mathbf{s}$ \\
  $\tilde{\mathbf{s}}^{(I)}$ & Approximation to $\mathbf{s}$ using $I$ instances \\ 
  $\overline{\mathbf{s}}$ & A mean of the samples of $\mathbf{s}$ \\ 
  $\href{https://docs.python.org/3/library/functions.html#len}{\text{len}}(\mathbf{s})$ & $=\mathbf{s}.\href{https://numpy.org/doc/stable/reference/generated/numpy.ndarray.size.html}{\mathsf{size}}$ Number of elements in $\mathbf{s}$ \\
  $\href{https://numpy.org/doc/stable/reference/generated/numpy.shape.html}{\text{shape}}(\mathbf{s})$ & ($=\mathbf{s}.{\mathsf{shape}}$) Shape of $\mathbf{s}$ \\
  $\text{rank}(\mathbf{s})$ & ($=\mathbf{s}.\mathsf{rank}=\text{len}(\mathbf{s}.\mathsf{shape})$) Dimensionality of $\mathbf{s}$ \\
  $\mathsf{\href{https://docs.python.org/3/library/functions.html\#func-range}{range}}(s)$ & $=\{0, 1, \cdots, s-1\}$ \\
  $\mathsf{\href{https://numpy.org/doc/stable/reference/generated/numpy.zeros_like.html}{zeros\_like}}(\mathbf{s})$ & $=\{0\}_{i=0}^{\mathbf{s}.\mathsf{size}-1}$ \\
  % $|\mathbf{X}_i|$ & The absolute value of $\mathbf{X}_i$ \\
  $\alpha\mathbf{s}$ & $=\{\alpha\mathbf{s}_i\}$ (scalar multiplication) \\
  $\mathbf{x}+\mathbf{y}$ & $=\{\mathbf{x}_i + \mathbf{y}_i\}$ (Hadamard addition) \\ 
  $\mathbf{x}\mathbf{y}$ & $=\{\mathbf{x}_i\mathbf{y}_i\}$ (Hadamard product) \\ 
  $\mathcal{N}$ & The normal distribution \\ 
  $\mathcal{P}$ & The Poisson distribution \\
  $\mathbf{x}\sim\mathcal{N}$ & The elements of $\mathbf{x}$ follows a normal distribution \\
  $\mathbf{x}_{\mathcal{N}}$ & The same as $\mathbf{x}\sim\mathcal{N}$ \\
  $\Pr(\mathbf{x}=a)$ & Probability that a $\mathbf{x}_i$ takes the value $a$ \\
  $\Pr(\mathbf{x}=a, \mathbf{y}=b)$ & $\Pr(\mathbf{x}=a)$ and $\Pr(\mathbf{y}=b)$ (joint probability)  \\
  $\text{Su}(\mathbf{x})$ & $=\{x\in\mathbb{R}|\Pr(\mathbf{x}=x)>0\}$ (support of $\mathbf{x}$)\\
  $\Pr(A|B)$ & Conditional probability of $A$ given $B$ \\
  $\mathbb{E}(\mathbf{s})$ & Expectation of $\mathbf{s}$ \\
  $\mathbb{V}(\mathbf{s})$ & Variance of $\mathbf{s}$ \\
  $||\mathbf{s}||_2$ & $L_2$ norm of $\mathbf{s}$ \\
  $f_s$ & Sampling frequency \\
  $\mathcal{F}$ & The (forward) Fourier transform ($\mathcal{F}(\mathbf{s})=\mathbf{S}$) (see Section~\ref{sec:Fourier_transform})\\
  $\mathcal{F}^{-1}$ & The inverse Fourier transform ($\mathcal{F}^{-1}(\mathbf{S})=\mathbf{s}$)  (see Section~\ref{sec:Fourier_transform})\\
  $\cdot^*$ & the complex conjugate of $\cdot$ \\
  $\mathbf{x}*\mathbf{y}$ & $=\mathcal{F}^{-1}(\mathcal{F}(\mathbf{x})\mathcal{F}(\mathbf{y}))=\mathcal{F}^{-1}(\mathbf{X}\mathbf{Y})$ (convolution) \\
  $A(b)$ & $A$ depends on (parameter) $b$ \\
  $A.b$ & The $b$ component of the data structure $A$ \\
  $(A)b$ & First $A$, then $b$ \\
  $E(\mathbf{s})$ & Energy of $\mathbf{s}$ (see Section~\ref{sec:energy_signal}) \\
  $P(\mathbf{s})$ & Power of $\mathbf{s}$ (see Section~\ref{sec:power_signal}) \\
  $\text{PS}(\mathbf{s})$ & Power spectrum of $\mathbf{s}$ (see Section~\ref{sec:power_spectrum}) \\
  $\text{PSD}(\mathbf{s})$ & Power spectral density of $\mathbf{s}$ (see Section~\ref{sec:PSD}) \\
  $\text{CC}(\mathbf{x},\mathbf{y})$ & Cross-correlation between $\mathbf{x}$ and $\mathbf{y}$ (see Section~\ref{sec:cross-correlation}) \\
  $\text{NCC}(\mathbf{x},\mathbf{y})$ & Normalized cross-correlation between $\mathbf{x}$ and $\mathbf{y}$ (see Section~\ref{sec:cross-correlation}) \\
  $\mathbf{a}\bot \mathbf{b}$ & $\mathbf{a}$ and $\mathbf{b}$ are orthogonal or independent                                
\end{tabular}

%}}}

\input{intro}
\input{discrete_signals}
\input{quantization}
\input{statistics}
\chapter{Noise}

Includes quantum mottle (statistical nature of photon detection),
grain noise (in film radiography), structured noise (due to detector
electronics or artefacts like splashes on intensifiers), and
anatomical noise (superimposition of unrelated anatomical structures)
\cite{bushberg2011essential}.


Noise Power Spectrum (NPS): A frequency-dependent measure that
describes the overall noise level and noise texture (how noise is
correlated between pixels). It's crucial for imaging system design and
comprehensive evaluation \cite{bushberg2011essential}.

High noise levels degrade contrast resolution and object conspicuity,
making lesions harder to detect. Image processing techniques like
smoothing can reduce noise but may compromise spatial
resolution. \cite{bushberg2011essential}.
\input{distortion_metrics}
\input{MNI}
\input{GF}

% \chapter{Beltrami flow}
% Parece que se usa en AND, pero no se bien cómo.

%\chapter{Median filter}

%\chapter{Bilateral filter}

\input{WF}

% \chapter{Anisotropic Non-linear Diffusion (AND)}

% \chapter{PURE-LET}
%{{{

% . Luisier, T. Blu, and M. Unser. Image denoising in mixed
% PoissonGaussian noise. IEEE Transactions on Image Pro-
% cessing, 20(3):696–708, 2011.

%}}}

%  \chapter{Non Local Means (NLM)}
%{{{

% https://biomedical-engineering-online.biomedcentral.com/articles/10.1186/1475-925X-14-2
% A. Buades, B. Coll, and J.-M. Morel. A non-local algorithm
% for image denoising. In CVPR, 2005.

%}}}

% \chapter{BM3D/BM4D}
%{{{

% K. Dabov, A. Foi, V. Katkovnik, and K. Egiazarian. Image
% denoising by sparse 3-D transform-domain collaborative
% filtering. IEEE Transactions on Image Processing, 16(8):2080–2095, 2007.
%https://pypi.org/project/bm4d/#files

%}}}

% \chapter{K-SVD}
%{{{

% M. Aharon, M. Elad, and A. Bruckstein. K-SVD: An algorithm for
% designing overcomplete dictionaries for sparse
% representation. IEEE Transactions on Signal Processing,
% 54(11):4311–4322, 2006.

%}}}

% \chapter{EPLL}
%{{{

% D. Zoran and Y. Weiss. From learning models of natural
% image patches to whole image restoration. In ICCV, 2011.

%}}}

% \chapter{WNNM}
%{{{

% S. Gu, L. Zhang, W. Zuo, and X. Feng. Weighted nuclear
% norm minimization with application to image denoising. In
% CVPR, 2014.

%}}}

% \chapter{The U-Net}
% \chapter{N2V}
%{{{

% A. Krull, T.-O. Buchholz, and F. Jug, “Noise2void-learning denoising
% from single noisy images,” in Proceedings of the IEEE/CVF conference
% on computer vision and pattern recognition, 2019, pp. 2129–2137.

%}}}

%\chapter{Pixel2Pixel}
%https://ieeexplore.ieee.org/abstract/document/10908805 

%\chapter{DnCNN}
% K. Zhang, W. Zuo, Y. Chen, D. Meng, and L. Zhang. Be-
% yond a Gaussian denoiser: Residual learning of deep CNN
% for image denoising. IEEE Transactions on Image Process-
% ing, 26(7):3142–3155, 2017.

% \chapter{FFDNet}
% K. Zhang, W. Zuo, and L. Zhang. Ffdnet: Toward a fast
% and flexible solution for CNN based image denoising. IEEE
% Transactions on Image Processing, 2018.

% \chapter{CBDNet}
% S. Guo, Z. Yan, K. Zhang, W. Zuo, and L. Zhang. Toward
% convolutional blind denoising of real photographs. In CVPR,
% 2019.

% \chapter{UDNet}
% S. Lefkimmiatis. Universal denoising networks: A novel
% cnn-based network architecture for image denoising. In
% CVPR, 2018.

% T. Pl¨otz and S. Roth. Neural nearest neighbors networks. In
% NIPS, 2018.

%\chapter{Noise2Noise}
% . Lehtinen, J. Munkberg, J. Hasselgren, S. Laine, T. Kar-
% ras, M. Aittala, and T. Aila. Noise2noise: Learning image
% restoration without clean data. In ICML, 2018.

%\chapter{Adapted RCAN (as denoiser)}

% Used in TriS-D, were the last upscaling module is removed to keep
% the spatial resolution \cite{ma2025spatiotemporal}.

% Y. Zhang, K. Li, K. Li, L. Wang, B. Zhong, and Y. Fu, “Image super-
% resolution using very deep residual channel attention networks,” in
% European Conference on Computer Vision (ECCV), 2018.

\input{SPGD}
\input{machine_learning}
\input{Cryo-CARE}

%\chapter{2D Random Shuffing Volume Denoising (2D-RSVD)}

%\chapter{3D Random Shuffling Volume Denoising (3D-RSVD)}


\input{comparative}

\appendix

% Redefine the appendix numbering to use alphalph's extended alphabet
%\makeatletter
%\renewcommand*{\thesection}{%
%  \AlphAlph{\value{section}}%
%}
%\makeatother

%{{{




%}}}

\begin{appendix}

\chapter{Images with \gls{GT}}
\label{sec:images}  
  
\begin{figure}[h]
  \centering
  \href{https://nbviewer.org/github/vicente-gonzalez-ruiz/denoising/blob/main/notebooks/Confocal_FISH.ipynb}{\includegraphics{Confocal_FISH.pdf}}
  \href{https://nbviewer.org/github/vicente-gonzalez-ruiz/denoising/blob/main/notebooks/Confocal_FISH_noisy.ipynb}{\includegraphics{Confocal_FISH_noisy.pdf}}
  \caption{Confocal FISH \cite{zhang2019poisson}.\label{fig:Confocal_FISH}}
\end{figure}

\begin{figure}[h]
  \centering
  \href{https://nbviewer.org/github/vicente-gonzalez-ruiz/denoising/blob/main/notebooks/TwoPhoton_MICE.ipynb}{\includegraphics{TwoPhoton_MICE.pdf}}
  \href{https://nbviewer.org/github/vicente-gonzalez-ruiz/denoising/blob/main/notebooks/TwoPhoton_MICE_noisy.ipynb}{\includegraphics{TwoPhoton_MICE_noisy.pdf}}
  \caption{Two-photons MICE \cite{zhang2019poisson}.\label{fig:TwoPhoton_MICE}}
\end{figure}

\begin{figure}[h]
  \centering
  \href{https://nbviewer.org/github/vicente-gonzalez-ruiz/denoising/blob/main/notebooks/Confocal_MICE.ipynb}{\includegraphics{Confocal_MICE.pdf}}
  \href{https://nbviewer.org/github/vicente-gonzalez-ruiz/denoising/blob/main/notebooks/Confocal_MICE_noisy.ipynb}{\includegraphics{Confocal_MICE_noisy.pdf}}
  \caption{Confocal MICE \cite{zhang2019poisson}.\label{fig:Confocal_MICE}}
\end{figure}

\begin{figure}[h]
  \centering
  \href{https://nbviewer.org/github/vicente-gonzalez-ruiz/denoising/blob/main/notebooks/Confocal_BPAE.ipynb}{\includegraphics{Confocal_BPAE.pdf}}
  \href{https://nbviewer.org/github/vicente-gonzalez-ruiz/denoising/blob/main/notebooks/Confocal_BPAE.ipynb}{\includegraphics{Confocal_BPAE_noisy.pdf}}
  \caption{Confocal BPAE \cite{zhang2019poisson}.\label{fig:Confocal_BPAE}}
\end{figure}

\begin{figure}[h]
  \centering
  \href{https://nbviewer.org/github/vicente-gonzalez-ruiz/denoising/blob/main/notebooks/Confocal_BPAE.ipynb}{\includegraphics{Confocal_BPAE_R.pdf}}
  \href{https://nbviewer.org/github/vicente-gonzalez-ruiz/denoising/blob/main/notebooks/Confocal_BPAE.ipynb}{\includegraphics{Confocal_BPAE_noisy_R.pdf}}
  \caption{Confocal BPAE (red channel) \cite{zhang2019poisson}.\label{fig:Confocal_BPAE_R}}
\end{figure}

\begin{figure}[h]
  \centering
  \href{https://nbviewer.org/github/vicente-gonzalez-ruiz/denoising/blob/main/notebooks/Confocal_BPAE.ipynb}{\includegraphics{Confocal_BPAE_G.pdf}}
  \href{https://nbviewer.org/github/vicente-gonzalez-ruiz/denoising/blob/main/notebooks/Confocal_BPAE.ipynb}{\includegraphics{Confocal_BPAE_noisy_G.pdf}}
  \caption{Confocal BPAE (green channel) \cite{zhang2019poisson}.\label{fig:Confocal_BPAE_G}}
\end{figure}

\begin{figure}[h]
  \centering
  \href{https://nbviewer.org/github/vicente-gonzalez-ruiz/denoising/blob/main/notebooks/Confocal_BPAE.ipynb}{\includegraphics{Confocal_BPAE_B.pdf}}
  \href{https://nbviewer.org/github/vicente-gonzalez-ruiz/denoising/blob/main/notebooks/Confocal_BPAE.ipynb}{\includegraphics{Confocal_BPAE_noisy_B.pdf}}
  \caption{Confocal BPAE (blue channel) \cite{zhang2019poisson}.\label{fig:Confocal_BPAE_B}}
\end{figure}

\begin{figure}[h]
  \centering
  \href{https://nbviewer.org/github/vicente-gonzalez-ruiz/denoising/blob/main/notebooks/TwoPhoton_BPAE.ipynb}{\includegraphics{TwoPhoton_BPAE.pdf}}
  \href{https://nbviewer.org/github/vicente-gonzalez-ruiz/denoising/blob/main/notebooks/TwoPhoton_BPAE.ipynb}{\includegraphics{TwoPhoton_BPAE_noisy.pdf}}
  \caption{TwoPhoton BPAE \cite{zhang2019poisson}.\label{fig:TwoPhoton_BPAE}}
\end{figure}

\begin{figure}[h]
  \centering
  \href{https://nbviewer.org/github/vicente-gonzalez-ruiz/denoising/blob/main/notebooks/TwoPhoton_BPAE.ipynb}{\includegraphics{TwoPhoton_BPAE_R.pdf}}
  \href{https://nbviewer.org/github/vicente-gonzalez-ruiz/denoising/blob/main/notebooks/TwoPhoton_BPAE.ipynb}{\includegraphics{TwoPhoton_BPAE_noisy_R.pdf}}
  \caption{TwoPhoton BPAE (red channel) \cite{zhang2019poisson}.\label{fig:TwoPhoton_BPAE_R}}
\end{figure}

\begin{figure}[h]
  \centering
  \href{https://nbviewer.org/github/vicente-gonzalez-ruiz/denoising/blob/main/notebooks/TwoPhoton_BPAE.ipynb}{\includegraphics{TwoPhoton_BPAE_G.pdf}}
  \href{https://nbviewer.org/github/vicente-gonzalez-ruiz/denoising/blob/main/notebooks/TwoPhoton_BPAE.ipynb}{\includegraphics{TwoPhoton_BPAE_noisy_G.pdf}}
  \caption{TwoPhoton BPAE (green channel) \cite{zhang2019poisson}.\label{fig:TwoPhoton_BPAE_G}}
\end{figure}

\begin{figure}[h]
  \centering
  \href{https://nbviewer.org/github/vicente-gonzalez-ruiz/denoising/blob/main/notebooks/TwoPhoton_BPAE.ipynb}{\includegraphics{TwoPhoton_BPAE_B.pdf}}
  \href{https://nbviewer.org/github/vicente-gonzalez-ruiz/denoising/blob/main/notebooks/TwoPhoton_BPAE.ipynb}{\includegraphics{TwoPhoton_BPAE_noisy_B.pdf}}
  \caption{TwoPhoton BPAE (blue channel) \cite{zhang2019poisson}.\label{fig:TwoPhoton_BPAE_B}}
\end{figure}

\begin{figure}[h]
  \centering
  \href{https://nbviewer.org/github/vicente-gonzalez-ruiz/denoising/blob/main/notebooks/WideField_BPAE.ipynb}{\includegraphics{WideField_BPAE.pdf}}
  \href{https://nbviewer.org/github/vicente-gonzalez-ruiz/denoising/blob/main/notebooks/WideField_BPAE.ipynb}{\includegraphics{WideField_BPAE_noisy.pdf}}
  \caption{WideField BPAE \cite{zhang2019poisson}.\label{fig:WideField_BPAE}}
\end{figure}

\begin{figure}[h]
  \centering
  \href{https://nbviewer.org/github/vicente-gonzalez-ruiz/denoising/blob/main/notebooks/WideField_BPAE.ipynb}{\includegraphics{WideField_BPAE_R.pdf}}
  \href{https://nbviewer.org/github/vicente-gonzalez-ruiz/denoising/blob/main/notebooks/WideField_BPAE.ipynb}{\includegraphics{WideField_BPAE_noisy_R.pdf}}
  \caption{WideField BPAE (red channel) \cite{zhang2019poisson}.\label{fig:WideField_BPAE_R}}
\end{figure}

\begin{figure}[h]
  \centering
  \href{https://nbviewer.org/github/vicente-gonzalez-ruiz/denoising/blob/main/notebooks/WideField_BPAE.ipynb}{\includegraphics{WideField_BPAE_G.pdf}}
  \href{https://nbviewer.org/github/vicente-gonzalez-ruiz/denoising/blob/main/notebooks/WideField_BPAE.ipynb}{\includegraphics{WideField_BPAE_noisy_G.pdf}}
  \caption{WideField BPAE (green channel) \cite{zhang2019poisson}.\label{fig:WideField_BPAE_G}}
\end{figure}

\begin{figure}[h]
  \centering
  \href{https://nbviewer.org/github/vicente-gonzalez-ruiz/denoising/blob/main/notebooks/WideField_BPAE.ipynb}{\includegraphics{WideField_BPAE_B.pdf}}
  \href{https://nbviewer.org/github/vicente-gonzalez-ruiz/denoising/blob/main/notebooks/WideField_BPAE.ipynb}{\includegraphics{WideField_BPAE_noisy_B.pdf}}
  \caption{WideField BPAE (blue channel) \cite{zhang2019poisson}.\label{fig:WideField_BPAE_B}}
\end{figure}

\end{appendix}

\printglossary[type=\acronymtype]

\bibliographystyle{plain}
\bibliography{signal_processing,image_processing,microscopy,denoising,motion_estimation,image_compression,statistics}

\end{document}