\chapter{Fluoroscopy}

Fluoroscopy refers to the continuous acquisition of a sequence of x-ray images over
time, essentially a real-time x-ray movie of the patient. It is a transmission projection
imaging modality, and is, in essence, just real-time radiography. Fluoroscopic systems
use x-ray detector systems capable of producing images in rapid temporal sequence.
Fluoroscopy is used for positioning catheters in arteries, visualizing contrast agents
in the GI tract, and for other medical applications such as invasive therapeutic proce-
dures where real-time image feedback is necessary. It is also used to make x-ray movies
of anatomic motion, such as of the heart or the esophagus \cite{bushberg2011essential}.

\section{Resources}

\bibliography{tomography}
