\chapter{Fluoroscopy}

Produces real-time X-ray images with high temporal resolution (e.g.,
30 frames per second) but at a low dose (and therefore low \gls{SNR})
per image compared to radiography. This allows for continuous motion
viewing, useful for interventional procedures.\footnote{Fluoroscopy is
  used for positioning catheters in arteries, visualizing contrast
  agents in the \gls{GI} tract, and for other medical applications
  such as invasive therapeutic proce- dures where real-time image
  feedback is necessary. It is also used to make x-ray movies of
  anatomic motion, such as of the heart or the esophagus
  \cite{bushberg2011essential}.}

When possible, frame averaging is used to increase the
SNR.\footnote{Fluoroscopy systems provide excellent temporal
  resolution, a feature that is the basis of their clinical
  utility. However, fluoroscopy images are also relatively noisy, and
  under certain circumstances it is appropriate and beneficial to
  (reduce) temporal resolution for lower quantum noise. This can be
  accomplished by averaging a series of images. Appreciable frame
  averaging can cause notice-able image lag with reduced temporal
  resolution. The compromise depends on the specific fluoroscopic
  application and the preferences of the user
  \cite{bushberg2011essential}.}
