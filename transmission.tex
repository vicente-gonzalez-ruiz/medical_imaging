\part{Medical Images Transmission}

WWW technology is used for cost-effective image distribution to clinicians, often via a thin client paradigm \cite{bushberg2011essential}.

\chapter{Basic concepts}

\section{Communication link characteristics}
\begin{itemize}
\item \textbf{Capacity}: This is the total amount of data that a
  data link can transmit per second (bit-rate). It's typically measured in:
  \begin{tabular}{r|l}
    Acronym & Capacity in bits/second\\
    \hline
    1 Kbps & $10^3$\\
    1 Mbps & $10^3$ Kbps\\
    1 Gbps & $10^3$ Mbps\\
    1 Tbps & $10^3$ Gbps
  \end{tabular}
\item \textbf{Reliability}: Wired networks are more reliable
(have less \popup{\gls{BER}}{Ratio of the number of bits received in
error to the total number of bits transmitted over a specific time
interval.}) than wireless networks.
\item \textbf{Security}: Wired networks are more secure than wireless
networks.
\end{itemize}

\section{Networks, links, and channels}
\begin{itemize}
\item A \textbf{channel} is a available transmission capacity in a communication link.
\item A \textbf{link} is a point-to-point (wired) or multipoint
(wireless) physical medium that is able to transmit data. A link can have several channels.
\item A \textbf{network} is a collection of links and devices to
interconnect them (usually routers).
\end{itemize}

\section{Internet and internet}
\begin{itemize}
\item A \textbf{internet} is a network of networks.
\item \textbf{Internet} is the network of networks that we use every day.
\end{itemize}

\section{Web and web}
\begin{itemize}
\item A \textbf{web} is a communication system based on the
server/client model that use the \gls{HTTP} protocol.
\item The \textbf{Web} is the communication system (called \gls{WWW})
that runs at \popup{the Internet level}{You can run a local web at
home, only for your eyes.}. \gls{WWW} and Web are synonyms.
\end{itemize}

\section{Communication protocols}
\begin{itemize}
\item Define the steps and timmings that two (or more) networked
entities must use to communicate.
\item In the Internet, the name of the protocol \popup{suite}{There
are several protocols.} is \acrshort{TCP}/\acrshort{IP}.
\item For real-time transmissions, the suite also defines the
\gls{UDP}.
\item \gls{HTTP} is the protocol used in the Web (and any web).
\end{itemize}

\section{Packets}
\begin{itemize}
\item At the sender side, the data are splitted into packets.
\item Most of the links are \popup{multiplexed in time}{In an instant
of time, all the capacity of the link is used to transmit a packet.}.
\item Packets have two different parts:
\begin{enumerate}
\item A \textbf{header} that stores information for the \popup{correct
delivery}{Depending on the protocolo, even to solve the transmission
errors.}.
\item A \textbf{payload} that contains the data to transmit.
\end{enumerate}
\end{itemize}