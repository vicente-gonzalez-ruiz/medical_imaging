\part{Medical Images Transmission}

WWW technology is used for cost-effective image distribution to clinicians, often via a thin client paradigm \cite{bushberg2011essential}.

\chapter{Basic concepts}

\section{Communication link characteristics}
\begin{itemize}
\item \textbf{Capacity}: This is the total amount of data that a
  data link can transmit per second (bit-rate). It's typically measured in:
  \begin{tabular}{r|l}
    Acronym & Bit-rate \\
    \hline
    \popup{B}{byte}s & (8 \popup{bits, where a bit represents a logical state with one of two possible values.})\\
    \popup{KB}{kilobyte}s & $1\text{KB} = 2^{10}\text{B}$\\
    \popup{MB}{megabytes}s & $1\text{MB} = 2^{10}\text{KB}$\\
    \popup{GB}{gigabyte}s & $1\text{GB} = 2^{10}\text{MB}$\\
    \popup{TB}{terabyte}s & $1\text{TB} = 2^{10}\text{GB}$\\
    \popup{PB}{petabyte}s & $1\text{PB} = 2^{10}\text{GB}$
  \end{tabular}
\item \textbf{Reliability}: If the storage media need to connected to a
  current supply (for example, the \gls{RAM} memory of a computer),
  the media is said \emph{volatile}.
\item \textbf{Wired VS wireless}: A \gls{CD-ROM}, for example.
\end{itemize}
