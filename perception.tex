\chapter{Perception}

\section{Perception of the luminance VS the luminance}
\begin{itemize}
\item The perception of the intensity is not linear.
\end{itemize}
\begin{center}
  \includegraphics[width=0.9\textwidth]{linear2}\\
  (In each row and column, the gradient is linear.)
\end{center}

\section{Perception of the luminance VS the frequency}
\begin{itemize}
\item The perception of a change of the intensity varies with the spatial frequency.
\end{itemize}
\begin{center}
  \includegraphics[width=0.65\textwidth]{CSF3}\\
  (\gls{CSF}.)
\end{center}

\section{Perception of the \popup{luma}{Luminance.} VS the neighborhood (1/2)}
\begin{itemize}
\item The perception of the intensity depends the neighbour pixels.
\end{itemize}
\begin{center}
  \includegraphics[width=0.8\textwidth]{linear3}\\
  (In each block, all the pixels have the same intensity.)
\end{center}

\section{Perception of the \popup{luma}{Luminance.} VS the neighborhood (2/2)}
\begin{itemize}
\item The perception of the intensity depends the neighbour pixels.
\end{itemize}
\begin{center}
  \includegraphics[width=1.0\textwidth]{contraste_simultaneo}\\
  (All the internal squares have the same intensity.)
\end{center}

\section{Perception of the luma VS the visualization time}
\begin{itemize}
\item The perception of the intensity depends the visualization time.
\end{itemize}
\begin{center}
  \includegraphics[width=0.4\textwidth]{punto_y_difuminado}\\
  (Look to the central point for a while.)
\end{center}

\section{Noise masking}
\begin{itemize}
\item The perception of the structures depends on the type and intensity of the noise.
\end{itemize}
\begin{center}
  \includegraphics[width=1.0\textwidth]{noise_masking}\\
  (Different effects of Gaussian noise in the Wavelet domain.)
\end{center}

\section{Retinex theory}

Retinex theory was introduced by Edwin Land (the founder of Polaroid)
to explain human color perception. The name comes from retina + cortex
— highlighting that perception is not purely a retinal process but
involves cortical processing too.
