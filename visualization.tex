\chapter{Visualization}

\section{Keypoints}
\begin{itemize}
\item Visualization should allow a physician to the interpret the
  images for accurate diagnosis.
\item Nowadays, the previous statement should be also true for
  algorithms (including IAs).
\item Visualization should allow zooming, panning, windowing, and
  annotation.
\end{itemize}

\section{Display characteristics}
\begin{enumerate}
\item \textbf{(Spatial) resolution}: Although zooming can help, the
  higher the resolution, the better.
\item \textbf{Brightness}: Again, the higher, the better, although
  this is a feature that depends on the ambient light
  conditions. Brightness (luminance) is measured with a photometer.
\item \textbf{Contrast ratio}: Defined by
  \begin{equation}
    \text{CR} = \frac{\text{Intensity~of~the~pure~white}}{\text{Intensity~of~the~pure~black}},
  \end{equation}
  the higher, the better. OLED/MicroLED technology is better than TFT/LCD.
\item \textbf{Color depth}: Medical displays should be able to display
  at least $2^{10}$ different shades of gray.
\item \textbf{Viewing angle}: The higher, the better. OLED is better than TFT/LCD. 
\item \textbf{Refresh rate}: This feature is important if the image is
  not static (for example, in interventional imaging). Notice,
  however, that none of the current technologies (LED or LCD) generate
  flickering.
\item \textbf{Pixel size}: Large displays can be viewed from a greater distance.
\item \textbf{Stereo} \cite{wikipedia_stereoscopy}: Necessary if the
  display is used in remote interventions.
\item \textbf{Calibrated and DICOM compliant}.
\end{enumerate}

