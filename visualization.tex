\chapter{Visualization}
Medical image visualization encompasses the entire process of displaying and presenting medical images and related information to facilitate diagnosis and clinical decision-making. It's a multidisciplinary field drawing on science and engineering, including computer sciences, medical physics, and perceptual psychology

Visualization should allow a physician to the interpret the images for accurate diagnosis

Contrast Resolution: This is the ability to detect very subtle changes in grayscale and distinguish them from image noise. It's primarily characterized by the signal-to-noise ratio (SNR) in an image

Quantum noise is common in X-ray or gamma ray images because relatively few quanta are used to limit patient radiation dose

Display should be calibrated.

The human eye perceives brightness differences non-linearly.

At low luminance, the eye is very sensitive → small changes in brightness are perceptually big.

At high luminance, the eye is less sensitive → bigger brightness jumps are needed to recognize a change in the luminance.

Luminance is measured with a photometer

The Barten model of human visual contrast sensitivity


\href{https://en.wikipedia.org/wiki/Stereoscopy}{Stereoscopy using a single image}.