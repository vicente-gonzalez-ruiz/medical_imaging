\chapter{Quality Metrics in Medical Imaging}

In medical imaging, the most important image characteristics for
improving diagnostic utility are those that directly enable the
interpreting physician to make an accurate (correct) diagnosis, rather
than focusing on aesthetic appeal. This involves a careful balance and
optimisation of several key technical parameters
\cite{bushberg2011essential}.

\section{Spatial resolution}
This refers to the level of detail visible on an image, or how small
an object an imaging system can distinguish. Higher spatial resolution
means the ability to detect smaller objects or finer anatomical
details \cite{bushberg2011essential}.

\section{Blur level}
Even if the spatial resolution is high, blurring (or smoothing) produces
a loss of high frequency information (the edges) in the image,
dicreasing the accuracy the diagnosis.

Blurring is due to several factors:
\begin{enumerate}
\item \textbf{Imperfections in the optical of the imaging system}, which can be
  cuantified checking Point Spread Function (PSF)
  \cite{bushberg2011essential}.
\item \textbf{Patient motion}, that usually depends on the adquisition
  time.
\item \textbf{Reconstruction algorithms}, which must be implemented
  with the required accuracy (probably requiring more computation
  resources and time).a
\end{enumerate}

\section{Signal-to-Noise Ratio (SNR)}
In the context of medical imaging, the SNR is a metric that compares
how many useful information (which helps to the correctness of the
diagnosis) there is in the image compared to how many noise (which
difficults the diagnostic and therefore, affect to the accuracy of the
diagnosis) there is in the image.

Notice that to measure objectively the SNR, both, the signal and the
noise must be known. In practice, in medical imaging, neither the
signal nor the noise are usually known, only estimated. Therefore,
when we speak about SNR we are using a perceived (subjective) metric.

\section{Contrast}

Contrast can be controller using Look-Up Tables (LUTs).

\section{Depth}

Depth refers to the number of bits/pixel (in the case of the images)
and the number of bits/voxel (in the case of the volumes). This number
usually ranges between 8 and 16 (per channel). To perceive a good
contrast, the number of bits must be high enough (usually bigger than
6 -- $2^6=64$ --).


\textbf{Signal-to-Noise Ratio (SNR)}: Defined as the ratio of
  the amplitude of signal received to the average amplitude of the
  background noise.

  \section{Spatial resolution}
  Defined as the ability to
  distinguish between two points as separate and distinct, is
  typically in the range of a fraction of milimeters (for example, 0.5
  mm).

  \section{Contrast}
  In general, the contrast in an viewed image
  depends on the range of intensities displayed (in general, above 64), that should be as
  large as possible.