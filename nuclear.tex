\chapter{Nuclear Medicine Imaging}

Nuclear medicine is the branch of radiology in which a chemical or
other substance containing a radioactive isotope is given to the
patient orally, by injection or by inhalation. Once the material has
distributed itself according to the physiological status of the
patient, a radiation detector is used to make projection images from
the x- and/or gamma rays emitted during radioactive decay of the
agent. Nuclear medicine produces emission images (as opposed to
transmission images), because the radioisotopes emit their energy from
inside the patient. \cite{bushberg2011essential}

Nuclear medicine planar images are projection images, since each point
on the image is representative of the radioisotope activity along a
line projected through the patient. Planar nuclear images are
essentially 2D maps of the 3D radioisotope distri- bution, and are
helpful in the evaluation of a large number of
disorders. \cite{bushberg2011essential}


\section{Resources}

\bibliography{tomography}
