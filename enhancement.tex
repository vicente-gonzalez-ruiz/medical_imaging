\chapter{Medical Image Enhancement}

The primary objective of medical image enhancement is to improve the visual quality of an image, enabling physicians to more accurately observe and interpret critical details. Enhancements such as brightness adjustment, contrast optimization, edge sharpening, and other visual refinements can significantly aid in this process. For example, in modalities like X-rays, CT scans, MRIs, and ultrasounds, enhancement techniques can improve the visibility of tumors, blood vessels, bones, and other anatomical structures.

Singh, P. (2025). Understanding Medical Image Denoising, Enhancement, and Reconstruction. Biomedical Informatics and Smart Healthcare, 1(1), 35–39. https://doi.org/10.62762/BISH.2025.966762

\section{Histogram equalization}

Roy, S., Bhalla, K., & Patel, R. (2024). Mathematical analysis of histogram equalization techniques for medical image enhancement: A tutorial from the perspective of data loss. Multimedia Tools and Applications, 83(5), 14363–14392.

\section{Deep learning enhancement?}

Balaji, V., Song, T. A., Malekzadeh, M., Heidari, P., & Dutta, J. (2024). Artificial intelligence for PET and SPECT image enhancement. Journal of Nuclear Medicine, 65(1), 4–12.

\section{}
RetiNex algorithm, the Weber-Fechner method, the Fourier transform, histogram equalization, and linear regression algorithms.

\section{Edge sharpening}

\section{Contrast modification}