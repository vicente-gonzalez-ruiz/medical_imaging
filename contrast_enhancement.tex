\chapter{Image contrast enhancement}

\section{Objective}
The primary objective of medical image enhancement is to improve the
visual quality of an image, enabling physicians to more accurately
observe and interpret critical details. Enhancements such as
brightness adjustment, contrast optimization, edge sharpening, and
other visual refinements can significantly aid in this process.

\section{Contrast stretching (normalization)}
\begin{itemize}
\item Pixel-wise linear transformation that enhance the contrast by spreading out the
  intensity values over the full available range of pixel values (usually [0, 255]).
\item Let $x_i$ the input pixel intensity, the new value is
  \begin{equation}
    y_i = \frac{(x_i - x_{\min})}{(x_{\max} - x_{\min})} \times 255
  \end{equation}
  where $r_{\text{min}}$ and $r_{\text{max}}$ are the minimum and
  maximum input intensity values.
\end{itemize}

\section{Gamma correction}

corrects for the fact that displays (monitors, projectors, etc.) and human vision do not respond linearly to intensity values.

\begin{equation}
s = c \cdot r^{\gamma}
\end{equation}

Where:
\begin{itemize}
    \item $r$ = input pixel intensity (normalized to $[0,1]$)
    \item $s$ = output pixel intensity (normalized to $[0,1]$)
    \item $c$ = normalization constant (often $1$)
    \item $\gamma$ = gamma value (controls brightness/contrast)
\end{itemize}


\section{Histogram equalization}

Histogram equalization redistributes the intensity values of an image so that the histogram (distribution of pixel intensities) becomes more uniform. The goal is to increase the overall contrast, especially in areas that are too dark or too bright.

% Histogram Equalization Formulation
Let an image have $L$ intensity levels $0, 1, 2, \dots, L-1$.

1. Probability of intensity $r_k$:
\[
p(r_k) = \frac{n_k}{N}
\]
where 
\begin{itemize}
    \item $n_k$ = number of pixels with intensity $r_k$
    \item $N$ = total number of pixels in the image
\end{itemize}

2. Cumulative distribution function (CDF):
\[
c(r_k) = \sum_{j=0}^{k} p(r_j)
\]

3. Mapping to new intensity:
\[
s_k = (L-1) \cdot c(r_k)
\]
where 
\begin{itemize}
    \item $r_k$ = original intensity
    \item $s_k$ = new intensity after histogram equalization
\end{itemize}

%section{Sharpening}

\section{Homomorphic filtering}

% Homomorphic Filtering Formulation

An image can be modeled as the product of illumination and reflectance:
\[
f(x, y) = i(x, y) \cdot r(x, y)
\]
where
\begin{itemize}
    \item $f(x, y)$ = observed image intensity
    \item $i(x, y)$ = illumination component (slow-varying, low-frequency)
    \item $r(x, y)$ = reflectance component (details, high-frequency)
\end{itemize}

\textbf{Step 1: Logarithmic transformation}
\[
\ln f(x, y) = \ln i(x, y) + \ln r(x, y)
\]

\textbf{Step 2: Fourier transform}
\[
F(u, v) = \mathcal{F}\{\ln f(x, y)\} = I(u, v) + R(u, v)
\]

\textbf{Step 3: Apply a high-pass filter}
\[
S(u, v) = H(u, v) \cdot F(u, v)
\]
where $H(u,v)$ is the filter transfer function.

\textbf{Step 4: Inverse Fourier transform}
\[
s(x, y) = \mathcal{F}^{-1}\{S(u, v)\}
\]

\textbf{Step 5: Exponential transform to restore the image}
\[
f_{\text{enhanced}}(x, y) = \exp(s(x, y))
\]
