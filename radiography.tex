\chapter{Radiography}

\section{Acquisition}
Radiography (projection (2D) X-ray radiography) is a transmission
imaging modality where X-rays are emitted from a source, pass through
the patient, and are detected on the other side using a flat (usually
digital TFT\footnote{In digital X-ray detectors, a TFT array is used
  to read out electrical charges generated by the impact of the
  X-rays.}) detector. The attenuation properties of different tissues
(e.g., bone, soft tissue, air) modify the homogeneous distribution of
X-rays that enters the patient X-ray, forming the image in the
detector \cite{bushberg2011essential}.

Radiography is also a projection imaging modality, meaning that each
point on the image corresponds to information along a straight line
through the patient \cite{bushberg2011essential}.

