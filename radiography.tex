\chapter{Radiography}

Radiography is performed with an x-ray source on one side of the
patient and a (typically flat) x-ray detector on the other side. A
short-duration (typically less than ½ second) pulse of x-rays is
emitted by the x-ray tube, a large fraction of the x-rays interact in
the patient, and some of the x-rays pass through the patient and reach
the detector, where a radiographic image is formed. The homogeneous
distribution of x-rays that enters the patient is modified by the
degree to which the x-rays are removed from the beam (i.e.,
attenuated) by scattering and absorption within the tissues. The
attenua- tion properties of tissues such as bone, soft tissue, and air
inside the patient are very different, resulting in a heterogeneous
distribution of x-rays that emerges from the patient. The radiographic
image is a picture of this x-ray distribution. The detector used in
radiography can be photographic film (e.g., screen-film radiography)
or an electronic detector system (i.e., digital radiography)
\cite{bushberg2011essential}.

\section{Resources}

\bibliography{tomography}
