\chapter{\gls{DICOM}}

\section{What is DICOM?}
\begin{itemize}
\item \popup{Open}{Openly accessible and usable by anyone.}
  \popup{standard}{Actually is a set of standards.} created in 1992 by
  the \gls{ACR} and the \gls{NEMA} \cite{DICOM2025}.
\item It defines how to transfer (transport layers protocols:
  \gls{TCP} or \gls{UDP}), storage (filenames, structure of the
  filesystem, etc.), processing, display (such as \gls{GSDF}),
  perception, and use of information in medicine
  \cite{bushberg2011essential}.
\end{itemize}.

\section{What is DICOM file?}
\begin{itemize}
\item \gls{DICOM} is a media container for medical signals
  (\gls{ECG} signals, \gls{PCG} audios, 2D images, 3D images, and videos).
\item A DICOM file, such as a single CT slice, consists of two
  distinct parts:
  \begin{enumerate}
  \item The \textbf{header} is a block of data that contains specific
    \popup{information}{Such as name, age, gender, date of birth, and
      technical data about the image (device used, resolution, codec,
      etc.).}  that complements the image
    (\popup{attributes}{Depending on the image and the circumstances,
      certain information is mandatory, while other attributes are
      optional.}), classified as
    \href{https://dicom.nema.org/medical/dicom/current/output/html/part06.html#PS3.6}{DICOM
      tags}.
  \item The \popup{\textbf{image} itself}{The signal in
      general.}. Notice that the metadata provided by \gls{DICOM} are
    different from the image's metadata.
  \end{enumerate}
\end{itemize}.

\section{Encoding of 1D signals}
\begin{itemize}
\item \gls{ECG} and \gls{PCG} signals are \popup{\gls{PCM}
    encoded}{Without any loss or data compression.} with up to 24
  bits/sample.
\end{itemize}

\section{Supported image codecs}
\begin{enumerate}
\item Lossless:
  \begin{enumerate}
  \item \popup{Raw}{As generated by the ADC (Analog Digital Converted.}
    \href{https://en.wikipedia.org/wiki/Endianness}{Little Endian} with
    up to XXXX bits/pixel.
  \item \gls{RLE}.
  \item \href{https://en.wikipedia.org/wiki/Lossless_JPEG}{JPEG Lossless}.
  \item \href{https://en.wikipedia.org/wiki/Lossless_JPEG\#JPEG_LS}{JPEG-LS}.
  \item \href{https://en.wikipedia.org/wiki/JPEG_2000}{JPEG 2000}
    (reversible path).
  \item \href{https://en.wikipedia.org/wiki/JPEG_XL}{JPEG XL}
    (reversible path).
  \item \href{https://en.wikipedia.org/wiki/Deflate}{DEflate}. Similar
    to \gls{PNG}.
  \end{enumerate}
\item Lossy:
  \begin{enumerate}
  \item \gls{JPEG}.
  \item \href{https://en.wikipedia.org/wiki/JPEG_2000}{JPEG 2000}
    (ireversible path).
  \item \href{https://en.wikipedia.org/wiki/JPEG_XL}{JPEG XL}
    (ireversible path).
  \end{enumerate}
\end{enumerate}

\section{Supported video codecs (all lossy)}
\begin{enumerate}
\item \href{https://en.wikipedia.org/wiki/MPEG-2}{\gls{MPEG}-2}.
\end{enumerate}
  
DICOM is a technical standard for the digital storage and transmission
of medical images and related information.[1] It includes a file
format definition, which specifies the structure of a DICOM file, as
well as a network communication protocol that uses TCP/IP to
communicate between systems \cite{wikipedia2025DICOM}.

allowing devices and systems from different manufacturers to communicate and share medical image data seamlessly.

combines the image data with a wealth of metadata (name, ID, age, the type of imaging modality used (e.g., MRI, CT, X-ray), and acquisition parameters)

Strategies for image storage (hierarchical, prefetching, on-demand)
are covered \cite{bushberg2011essential}.

Standardisation of Display: The DICOM Grayscale Standard Display Function (GSDF) is highlighted as a critical standard for ensuring predictable transformation of pixel values to luminance and similar contrast display across devices, aiming for perceptual linearization \cite{bushberg2011essential}.

DICOM ULP
TCP
IP


UPL (Upper Layer Protocol) operates on a client-server model, where devices, known as Application Entities (AEs), communicate with each other. The protocol defines a set of Protocol Data Units (PDUs), which are the messages exchanged during these steps:

    Association Establishment (A-ASSOCIATE-RQ/AC/RJ PDU): Before any data is sent, two AEs must establish an "association." This is like a handshake where they negotiate and agree on the types of information and services they will exchange. This includes details like image compression, transfer syntax (byte ordering), and the specific DICOM services they will use (e.g., C-STORE for storing an image).

Data Transfer (P-DATA-TF PDU): Once the association is established, the actual DICOM messages, containing both command sets and image data, are transferred using this PDU. This is where the bulk of the medical image and patient data is transmitted.

Association Termination (A-RELEASE-RQ/RP/ABORT PDU): After the data transfer is complete, the association is properly released. This ensures an orderly end to the communication session. An "abort" PDU is used to terminate a session abruptly in case of an error.

The DICOM ULP essentially provides a "language" for medical devices to negotiate, send, and receive data, abstracting away the complexities of the underlying network and ensuring that different vendors' equipment can communicate effectively.
