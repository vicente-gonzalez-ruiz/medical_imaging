\chapter{Ultrasound Imaging}

Mechanical
energy in the form of high-frequency (“ultra”) sound can be used to generate images
of the anatomy of a patient. A short-duration pulse of sound is generated by an
ultrasound transducer that is in direct physical contact with the tissues being imaged.
The sound waves travel into the tissue, and are reflected by internal structures in the
body, creating echoes. The reflected sound waves then reach the transducer, which
records the returning sound \cite{bushberg2011essential}.

Doppler ultrasound makes use of a phenomenon familiar to train enthusiasts.
For the observer standing beside railroad tracks as a rapidly moving train goes
by blowing its whistle, the pitch of the whistle is higher as the train approaches
and becomes lower as the train passes by the observer and speeds off into the
distance. The change in the pitch of the whistle, which is an apparent change in
the frequency of the sound, is a result of the Doppler effect. The same phenom-
enon occurs at ultrasound frequencies, and the change in frequency (the Doppler
shift) is used to measure the motion of blood. Both the speed and direction of
blood flow can be measured, and within a subarea of the grayscale image, a color
flow display typically shows blood flow in one direction as red, and in the other
direction as blue \cite{bushberg2011essential}.

\section{Resources}

\bibliography{tomography}
