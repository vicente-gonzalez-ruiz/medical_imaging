% Ultrasound Imaging

Mechanical energy in the form of high-frequency (``ultra'') sound can
be used to generate images of the anatomy of a patient. A
short-duration pulse of sound is generated by an ultrasound transducer
that is in direct physical contact with the tissues being imaged. The
sound waves travel into the tissue, and are reflected by internal
structures in the body, creating echoes. The reflected sound waves
then reach the transducer, which records the returning sound
\cite{bushberg2011essential}.

The speed of the sound signal in the tissues are low enough to use the
Doppler effect to detect their motion. Thus, for example, we can
measure the blood flow displayed as color channels.\footnote{Both the
  speed and direction of blood flow can be measured, and within a
  subarea of the grayscale image, a color flow display typically shows
  blood flow in one direction as red, and in the other direction as
  blue \cite{bushberg2011essential}.}

Ultrasound imaging is basically a 2D technique (slices). However, 3D
images (volumes) can be generated by placing the known voxels in a 3D
structure and interpolating the unknown voxels. Then, using
segmentation it is possible to display surfaces to see, for example,
the face of a fetus.

Ultrasound images are noisy and the predominant type of noise is
speckle noise.
