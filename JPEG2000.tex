\chapter{JPEG2000}

\section{\acrshort{ISO} international standard}
\begin{itemize}
\item Developed by the \gls{JPEG} (ISO/IEC 15444\href{https://www.itu.int}{ITU}),
  the JPEG2000 standard \cite{taubman2002jpeg2000} in 2000, as a successor of
  \gls{JPEG}.
\item Mainly used in medical imaging and digital cinema.
\end{itemize}
\vspace{-2ex}
\begin{center}
  \href{https://en.wikipedia.org/wiki/Magnetic_resonance_imaging_of_the_brain#/media/File:MRI_of_Human_Brain.jpg}{\includegraphics[width=4.0cm]{MRI_of_Human_Brain}}\\
  (click on the image)
\end{center}

\section{Lossless and lossy compression}
\begin{itemize}
\item Two different compression modes:
\item \textbf{Reversible}: Allows a perfect reconstruction of the raster image (like \gls{PNG}).
\item \textbf{Irreversible}: Unable to recover all the visual
  information, but offers a better rate/distortion performance than
  the reversible mode for the same bit-rate.
\end{itemize}

\section{16-bit per channel and several color spaces support}
\begin{itemize}
\item \popup{16 bits/pixel offers enough quality for specialized
    applications such as medicine, atronomy, etc.}{It is quite
    difficult to increase the SNR above of 16 bits because basically
    we register noise when the amplitude of signal is small.}.
\item Apart from gray-scale and \gls{RGB}, JPEG2000 supports
  \gls{YCbCr}, \gls{sRGB} \cite{sRGB_wikipedia}, \gls{CIELAB}
  \cite{CIELAB_wikipedia}, and \popup{custom}{It is posible to define
    a color transform.} \cite{houchin2001specification} color spaces.
\end{itemize}

\section{Superior compression efficiency than \gls{JPEG}}
\begin{itemize}
\item Better quality at the same bitrate compared to JPEG \cite{vruiz_J2K}.
\end{itemize}
\begin{center}
  \begin{tabular}{cc}
    \multicolumn{2}{c}{Lena at 0.1 bits/pixel} \\
    \includegraphics[width=5cm]{lena_01} & \includegraphics[width=5cm]{lena_01_jp2} \\
    JPEG & JPEG2000
  \end{tabular}
\end{center}

\section{Baseline algorithm (1/2)}
\begin{enumerate}
\item Convert from the \gls{RGB} color space to the \gls{YCbCr} color
  space. Only if the input image is in color and not in \gls{YCbCr}.
\item Transform each \gls{YCbCr} component using the 2D-\gls{DWT}.
\item If we are using the irreversible mode, quantize the \gls{DWT}
  coefficients. /* Lossy step */
\item Entropy encode the quantized coefficients with \gls{EBCOT}.
\end{enumerate}

\section{Structure of an image in the DWT domain}
\begin{itemize}
\item The \gls{DWT} represents signals as a multiresolution structure
  \cite{vruiz_J2K}.
\end{itemize}
\vspace{-2ex}
\begin{center}
  \includegraphics[width=6cm]{2-level_wavelet_transform-lichtenstein}
\end{center}

\section{Spatial scalability}
\begin{itemize}
\item When the image is rendered by resolution levels.
\end{itemize}
\vspace{-2ex}
\begin{center}
  \resizebox{11cm}{!}{
    \begin{tabular}{ccc}
      \includegraphics{lena_128x128_rgb} & \includegraphics{lena_256x256_rgb} & \includegraphics{lena_512x512_rgb}
    \end{tabular}
  }
\end{center}

\section{Quality scalability}
\begin{itemize}
\item When the image is rendered by \popup{coefficient amplitude}{DWT
    coefficients are actually decoded based on their contribution to
    the rate/distortion (R/D) curve. An R/D curve relates (X-axis) the
    amount of decompressed data versus (Y-axis) the achieved
    distortion. Depending on the distortion metric used, the curve may
    decrease with increasing number of decompressed bits (e.g., when
    measuring MSE) or increase with increasing number of decompressed
    bits (e.g., when measuring SNR).} \cite{vruiz_J2K}.
\end{itemize}
\vspace{-2ex}
\begin{center}
  \begin{tabular}{ccc}
    \includegraphics[width=4.0cm]{lena_01_jp2} & \includegraphics[width=4.0cm]{lena_02_jp2} & \includegraphics[width=4.0cm]{lena_05_jp2} \\
    0.1 bits/pixel & 0.2 bits/pixel & 0.5 bits/pixel
  \end{tabular}
\end{center}

\section{\acrshort{ROI} scalability}
\begin{itemize}
\item It is possible to dedicate more bit-rate to a \gls{ROI}, that
  can be defined interactively \cite{vruiz_J2K}.
\end{itemize}
\vspace{-2ex}
\begin{center}
   \includegraphics[width=11cm]{ROI}
\end{center}
  
\section{Error resilience}
\begin{itemize}
\item An error in JPEG 2000 tends to affect only localized image areas
  rather than corrupting the entire picture, due to the entropy coding
  of data in relatively small independent blocks
  \cite{wikipedia_J2K,brahimi2021efficient}.
\end{itemize}
\vspace{-2ex}
\begin{center}
  \href{https://flylib.com/books/en/2.537.1.37/1}{\includegraphics[width=11cm]{J2K_error_resilience}}
\end{center}

\section{3D support}
\begin{itemize}
\item Based on the 3D-DWT \cite{Bruylants_J2K_3D}.
\end{itemize}
\vspace{-2ex}
\begin{center}
  \href{https://spie.org/images/Graphics/Newsroom/Imported/0779/0779_fig1.jpg}{\includegraphics[width=11cm]{J2K_3D}}
\end{center}

\section{Motion JPEG2000 (digital cinema)}
\begin{itemize}
\item The movies played in digital cinemas leverage the spatial
  scalability provided by JPEG2000 to distribute a single DCP file
  that can be played in different resolution-capability beamers
  \cite{wikipedia_DCP}.
\item Each piture of the movie is compressed independently
  \cite{DigitalCinema}.
\end{itemize}
\begin{center}
  \href{}{\includegraphics[width=8cm]{MJ2K}}
\end{center}

\section{Metadata}