\chapter{Medical Image Reconstruction}

Medical imaging modalities such as computed-tomography, magnetic -resonance imaging, positron-emission tomography (PET), and others engage in reconstruction in medical imaging. This procedure entails creating visual images from raw data. Projections, signals, or frequency data obtained from these devices must be converted into a comprehensible and useful image for diagnostic purposes. Computed Tomography Scan: Useful for identifying infections, malignancies, and fractures via the reconstruction of cross-sectional images. By using iterative reconstruction, the radiation dosage may be decreased. To detect neurological and musculoskeletal diseases, a Magnetic Resonance Imaging is used to get a highly detailed image of soft tissues. fMRI reveals neural activity. Brain and cancer patients may benefit from re-creating metabolic activity images using Positron Emission Tomography scans. Assesses the foetal heart's shape and development by ultrasound. Optical Coherence Tomography: Uses retinal images to diagnose macular degeneration and glaucoma.

\section{Super resolution??}

\section{Stereoscopic systems}